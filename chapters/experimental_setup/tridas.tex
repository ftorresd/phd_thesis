
\section{Trigger and Data Acquisition}

The LHC collides protons at 40 MHz. To process and save this amount of information would be unmanageable. To deal with the high rate of readouts generated by the collisions and nuclear background (around 950 TB/s) CMS uses a two-tiered trigger system~\cite{Khachatryan:2016bia}. The first level (L1), composed of custom hardware processors, uses information from the calorimeters and muon detectors, in the form of the so called trigger-primitives, to select events at a rate of around 100\unit{kHz} within a time interval of less than 4\mus. The L1 trigger relies on the processing of the optical links, coming CMS subdetector by FPGAs (Field Programmable Gate Array) processors. This combination of technologies allows the maximum speed in the readout information processing.

The second level, known as the high-level trigger (HLT), consists of a farm of processors running a version of the full event reconstruction software optimized for fast processing, and reduces the event rate to around 1\unit{kHz} before data storage.

Both triggers systems are designed to quickly identify the events~\footnote{A Event can be understood the set of information from the detector channels, extracted in one readout cycle.} that have a specific set of signatures of interesting physics, to the context of CMS. As an example, events with characteristics of the historically widely studied soft-diffraction, are mostly (but not fully) discarded. 

Once a event is read by CMS, it is categorized in one or many of the defined "triggers". Each trigger is composed by a minimum sets of requirements, e.g. a single isolated muon trigger is defined as "at least one muon, well isolated from any other detector relevant activity, above a minimum transverse momentum threshold". If a event falls into a L1 trigger definition and passes the prescaling~\footnote{Each trigger has its prescaling. For example, a prescaling 30 means that only once every 30 times that this trigger is activated, the event will in be processed and forwarded into the data acquisition chain.} of that trigger, a "L1 Accept" (L1A) optical signal is propagated to all subdetectors readout hardware and the information is injected into the Data Acquisition (DAQ) system and saved at the local computing cluster, the HLT. This decision process takes around 3.2 $\mu$s. Saved events are processed by an optimized version of the Particle-Flow algorithm and if it again falls into one the HLT triggers paths (definitions) it is saved for future analysis.