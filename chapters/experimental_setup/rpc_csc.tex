
\subsection{CSC}

The Cathode Strip Chamber (CSC) is also a gaseous detector (50\% CO2, 40\% Ar, and 10\% CF4) of the Muon System which covers the endcap region, up to $|\eta| < 2.4$ composed by wires perpendicular to $\eta$ (radial measurement) and strips along $\eta$, the former operating at 3.9 to 3.6 kV. With 8.4 to 16 mm strip width and a wire-distance of 2.5 to 3.16 mm depending on their location, they provide a 75 to 150 $\mu$m resolution.

They are installed in four layers (or disk) on each side of CMS, with each disk divided in up to three rings.


\subsection{RPC}

The Resistive Plate Chambers (RPC) is the only muon detection technology present in both barrel and endcap. It has very good timing resolution and it is used mostly for triggering.

Due to the particularities of the study, especially the contributions given to the RPC project of CMS, Chapter~\ref{chapter_rpc} is devoted exclusively to this subdetector.

