%%%%%%%%%%%%%%%%%%%%%%%%%%%%%%%%%%%%%%%%%%
\chapter{Results and conclusion}
\label{chaper_results}

A two-dimensional (2D) unbinned maximum-likelihood fit to the $m_{\mu^{+}\mu^{-}\gamma}$ and $m_{\mu^{+}\mu^{-}}$ distributions was used to compare the data with background and signal predictions. Search has been performed for a SM Higgs and $\mathrm{Z}$ boson decaying into a $\Upsilon(1S,2S,3S)\gamma$, with $\Upsilon(1S,2S,3S)$ subsequently decaying into $\mu^{+}\mu^{-}$ using data obtained from $35.9~fb^{-1}$ of $pp$ colisions at $\sqrt{s}=13~\mathrm{TeV}$. 

Since no excess has been observed above the background, the \CLs formalism is applied, in order to establish an upper limit in the braching fractions for each channel.

\section{The $CL_{s}$ formalism for upper limits setting at CMS}
\label{sec:cls}

The \CLs formalism~\cite{cls_Read_2002} consist in a modified frequentist approach to obtain an upper limit for a certain parameter of a model, with respect to the data. It is based on the profile-likelihood-ratio test statistic~\cite{profiled_lh} and asymptotic approximations~\cite{asymptotic_cls}. It is a standard upper limit setting procedure for the LHC experiments~\cite{CMS-NOTE-2011-005}.

When searching for non-observed phenomena, it is often usual to derive the results as a function of the signal strength modifier $\mu$, which is a free parameter of the full model (signal + background). It can be define such as, the expectation value for the number of events in a bin~\footnote{A set of common analysis criteria.} is:


\begin{equation}
\label{eqn:signal_strength}
E[n] = \mu s + b,
\end{equation}
where, $s$ and $b$ are the expected number of signal e background events, respectively.

The Neyman–Pearson lemma~\cite{profiled_lh} states the the optimal 


\todo{Terminar a descrição do $CL_{s}$ formalism.}

\section{Branching fraction upper limits}
\label{sec:results}
The result are summarized on table \ref{tab:UpperLimits_Cat123}.

\begin{table}[ht]
\begin{center}
%\resizebox{.5\width}{!}{\begin{tabular}{l|llll}
\multicolumn{4}{c}{95\% C.L. Upper Limit} \\
\hline
\hline
& \multicolumn{3}{c}{$\mathcal{B}(Z \rightarrow \Upsilon\gamma)$ $[\times10^{-6}]$}      \\
\cline{2-4}
&  $\Upsilon(1S)$ & $\Upsilon(2S)$ & $\Upsilon(3S)$  \\
\hline
Expected     & $6.4^{+3.1}_{-2.0}$ &  $8.3^{+4.0}_{-2.5}$  & $8.0^{+3.9}_{-2.4}$            \\
Observed     & 9.0 &  12.3  & 11.4      \\
\hline
SM Prediction $[\times10^{-8}]$ & 4.8  &  2.4  & 1.9      \\
\hline
\hline
& \multicolumn{3}{c}{$\mathcal{B}(H \rightarrow \Upsilon\gamma)$ $[\times10^{-4}]$}       \\
\cline{2-4}
&  $\Upsilon(1S)$ & $\Upsilon(2S)$ & $\Upsilon(3S)$ &   \\
\hline
Expected     & $12.5^{+6.1}_{-3.9}$ &  $14.6^{+7.1}_{-4.5}$  & $13.6^{+6.6}_{-4.2}$        \\
Observed     & 11.5 &  13.6  & 12.7     \\
\hline
SM Prediction $[\times10^{-9}]$ & 5.2  &  1.4  & 0.9      \\
\hline
\hline
\end{tabular}

}
% \begin{tabular}{l|llll}
\multicolumn{4}{c}{95\% C.L. Upper Limit} \\
\hline
\hline
& \multicolumn{3}{c}{$\mathcal{B}(Z \rightarrow \Upsilon\gamma)$ $[\times10^{-6}]$}      \\
\cline{2-4}
&  $\Upsilon(1S)$ & $\Upsilon(2S)$ & $\Upsilon(3S)$  \\
\hline
Expected     & $6.4^{+3.1}_{-2.0}$ &  $8.3^{+4.0}_{-2.5}$  & $8.0^{+3.9}_{-2.4}$            \\
Observed     & 9.0 &  12.3  & 11.4      \\
\hline
SM Prediction $[\times10^{-8}]$ & 4.8  &  2.4  & 1.9      \\
\hline
\hline
& \multicolumn{3}{c}{$\mathcal{B}(H \rightarrow \Upsilon\gamma)$ $[\times10^{-4}]$}       \\
\cline{2-4}
&  $\Upsilon(1S)$ & $\Upsilon(2S)$ & $\Upsilon(3S)$ &   \\
\hline
Expected     & $12.5^{+6.1}_{-3.9}$ &  $14.6^{+7.1}_{-4.5}$  & $13.6^{+6.6}_{-4.2}$        \\
Observed     & 11.5 &  13.6  & 12.7     \\
\hline
SM Prediction $[\times10^{-9}]$ & 5.2  &  1.4  & 0.9      \\
\hline
\hline
\end{tabular}



\begin{tabular}{l|llll}
\multicolumn{4}{c}{95\% C.L. Upper Limit} \\
\hline
\hline
& \multicolumn{3}{c}{$\mathcal{B}(Z \rightarrow \Upsilon\gamma)$ $[\times10^{-6}]$}      \\
\cline{2-4}
&  $\Upsilon(1S)$ & $\Upsilon(2S)$ & $\Upsilon(3S)$  \\
\hline
Expected     & $1.6^{+0.8}_{-0.5}$ &  $2.0^{+1.0}_{-0.6}$  & $1.8^{+1.0}_{-0.6}$            \\
Observed     & 2.9 &  2.7  & 1.4      \\
\hline
SM Prediction $[\times10^{-8}]$ & 4.8  &  2.4  & 1.9      \\
\hline
\hline
& \multicolumn{3}{c}{$\mathcal{B}(H \rightarrow \Upsilon\gamma)$ $[\times10^{-4}]$}       \\
\cline{2-4}
&  $\Upsilon(1S)$ & $\Upsilon(2S)$ & $\Upsilon(3S)$ &   \\
\hline
Expected     & $7.3^{+4.0}_{-2.4}$ &  $8.1^{+4.6}_{-2.8}$  & $6.8^{+3.9}_{-2.3}$        \\
Observed     & 6.9 &  7.4  & 5.8     \\
\hline
SM Prediction $[\times10^{-9}]$ & 5.2  &  1.4  & 0.9      \\
\hline
\hline
\end{tabular}
	
\caption{Summary table for the limits on branching ratio of $\mathrm{Z}\to\Upsilon(1S,2S,3S)\gamma$ and $\mathrm{H}\to\Upsilon(1S,2S,3S)\gamma$ decays.}
\label{tab:UpperLimits_Cat123}
\end{center}
\end{table}

The observed(expected) exclusion limit at $95\%$ confidence level on the $\mathcal{B}(\mathrm{Z}\to\Upsilon(1S,2S,3S)\gamma)=$ 2.9, 2.7, 1.4 ($1.6^{+0.8}_{-0.5}$,  $2.0^{+1.0}_{-0.6}$, $1.8^{+1.0}_{-0.6}$)$\times 10^{-6}$, and on the $\mathcal{B}(\mathrm{H}\to\Upsilon(1S,2S,3S)\gamma)=$ 6.9, 7.4, 5.8 ($7.3^{+4.0}_{-2.4}$,  $8.1^{+4.6}_{-2.8}$, $6.8^{+3.9}_{-2.3}$)$\times 10^{-4}$.

As stated before, this analysis was done, for the Z decay, taking into account a mutually excludent categorization of events, based on the reconstructed photon properties ($\eta_{SC}$ and R9 value), as described in section~\ref{sec:categorization}. 

At table~\ref{tab:UpperLimits_Cat0} we present the results obtained when there is no categorization of events (Inclusive category).

\begin{table}[ht]
\begin{center}
%\resizebox{.5\width}{!}{\begin{tabular}{l|llll}
\multicolumn{4}{c}{95\% C.L. Upper Limit} \\
\hline
\hline
& \multicolumn{3}{c}{$\mathcal{B}(Z \rightarrow \Upsilon\gamma)$ $[\times10^{-6}]$}      \\
\cline{2-4}
&  $\Upsilon(1S)$ & $\Upsilon(2S)$ & $\Upsilon(3S)$  \\
\hline
Expected     & $6.4^{+3.1}_{-2.0}$ &  $8.3^{+4.0}_{-2.5}$  & $8.0^{+3.9}_{-2.4}$            \\
Observed     & 9.0 &  12.3  & 11.4      \\
\hline
SM Prediction $[\times10^{-8}]$ & 4.8  &  2.4  & 1.9      \\
\hline
\hline
& \multicolumn{3}{c}{$\mathcal{B}(H \rightarrow \Upsilon\gamma)$ $[\times10^{-4}]$}       \\
\cline{2-4}
&  $\Upsilon(1S)$ & $\Upsilon(2S)$ & $\Upsilon(3S)$ &   \\
\hline
Expected     & $12.5^{+6.1}_{-3.9}$ &  $14.6^{+7.1}_{-4.5}$  & $13.6^{+6.6}_{-4.2}$        \\
Observed     & 11.5 &  13.6  & 12.7     \\
\hline
SM Prediction $[\times10^{-9}]$ & 5.2  &  1.4  & 0.9      \\
\hline
\hline
\end{tabular}

}
% \begin{tabular}{l|llll}
\multicolumn{4}{c}{95\% C.L. Upper Limit} \\
\hline
\hline
& \multicolumn{3}{c}{$\mathcal{B}(Z \rightarrow \Upsilon\gamma)$ $[\times10^{-6}]$}      \\
\cline{2-4}
&  $\Upsilon(1S)$ & $\Upsilon(2S)$ & $\Upsilon(3S)$  \\
\hline
Expected     & $6.4^{+3.1}_{-2.0}$ &  $8.3^{+4.0}_{-2.5}$  & $8.0^{+3.9}_{-2.4}$            \\
Observed     & 9.0 &  12.3  & 11.4      \\
\hline
SM Prediction $[\times10^{-8}]$ & 4.8  &  2.4  & 1.9      \\
\hline
\hline
& \multicolumn{3}{c}{$\mathcal{B}(H \rightarrow \Upsilon\gamma)$ $[\times10^{-4}]$}       \\
\cline{2-4}
&  $\Upsilon(1S)$ & $\Upsilon(2S)$ & $\Upsilon(3S)$ &   \\
\hline
Expected     & $12.5^{+6.1}_{-3.9}$ &  $14.6^{+7.1}_{-4.5}$  & $13.6^{+6.6}_{-4.2}$        \\
Observed     & 11.5 &  13.6  & 12.7     \\
\hline
SM Prediction $[\times10^{-9}]$ & 5.2  &  1.4  & 0.9      \\
\hline
\hline
\end{tabular}



\begin{tabular}{l|llll}
\multicolumn{4}{c}{95\% C.L. Upper Limit - $\mathcal{B}(Z \rightarrow \Upsilon\gamma)$ $[\times10^{-6}]$} \\
\hline
\hline
& \multicolumn{3}{c}{without categorization}      \\
\cline{2-4}
&  $\Upsilon(1S)$ & $\Upsilon(2S)$ & $\Upsilon(3S)$  \\
\hline
Expected     & $1.7^{+0.9}_{-0.5}$ &  $2.1^{+1.1}_{-0.7}$  & $1.9^{+1.0}_{-0.6}$            \\
Observed     & 2.6 &  2.3  & 1.2      \\
\hline
\hline
& \multicolumn{3}{c}{with categorization}      \\
\cline{2-4}
&  $\Upsilon(1S)$ & $\Upsilon(2S)$ & $\Upsilon(3S)$  \\
\hline
Expected     & $1.6^{+0.8}_{-0.5}$ &  $2.0^{+1.0}_{-0.6}$  & $1.8^{+1.0}_{-0.6}$            \\
Observed     & 2.9 &  2.7  & 1.4      \\
\hline
\hline
\end{tabular}
	
\caption{Summary table for the limits on branching ratio of $\mathrm{Z}\to\Upsilon(1S,2S,3S)\gamma$, for the two possible categorization scenarios.}
\label{tab:UpperLimits_Cat0}
\end{center}
\end{table}


It is worth to remember that the categorization takes places only for the Z decay. For the Higgs decay, no categorization is imposed.

By taking, or not, into account any categorization, the numbers presented in both tables (\ref{tab:UpperLimits_Cat123} and \ref{tab:UpperLimits_Cat0}), are compatible within themselves and with the results published by the ATLAS collaboration~\cite{Aaboud_2018}.


\clearpage
