%%%%%%%%%%%%%%%%%%%%%%%%%%%%%%%%%%%%%%%%%%
\section{Results for the inclusive category}

As stated before, this analysis was done, for the Z decay, taking into account a mutually excludent categorization of events, based on the reconstructed photon properties ($\eta_{SC}$ and R9 value), as described in section~\ref{sec:categorization}. 

For a matter of documentation, in table~\ref{tab:UpperLimits_Cat0} we present the results obtained when there is no categorization of events (inclusive category - Cat0).

\begin{table}[ht]
\begin{center}
%\resizebox{.5\width}{!}{\begin{tabular}{l|llll}
\multicolumn{4}{c}{95\% C.L. Upper Limit} \\
\hline
\hline
& \multicolumn{3}{c}{$\mathcal{B}(Z \rightarrow \Upsilon\gamma)$ $[\times10^{-6}]$}      \\
\cline{2-4}
&  $\Upsilon(1S)$ & $\Upsilon(2S)$ & $\Upsilon(3S)$  \\
\hline
Expected     & $6.4^{+3.1}_{-2.0}$ &  $8.3^{+4.0}_{-2.5}$  & $8.0^{+3.9}_{-2.4}$            \\
Observed     & 9.0 &  12.3  & 11.4      \\
\hline
SM Prediction $[\times10^{-8}]$ & 4.8  &  2.4  & 1.9      \\
\hline
\hline
& \multicolumn{3}{c}{$\mathcal{B}(H \rightarrow \Upsilon\gamma)$ $[\times10^{-4}]$}       \\
\cline{2-4}
&  $\Upsilon(1S)$ & $\Upsilon(2S)$ & $\Upsilon(3S)$ &   \\
\hline
Expected     & $12.5^{+6.1}_{-3.9}$ &  $14.6^{+7.1}_{-4.5}$  & $13.6^{+6.6}_{-4.2}$        \\
Observed     & 11.5 &  13.6  & 12.7     \\
\hline
SM Prediction $[\times10^{-9}]$ & 5.2  &  1.4  & 0.9      \\
\hline
\hline
\end{tabular}

}
% \begin{tabular}{l|llll}
\multicolumn{4}{c}{95\% C.L. Upper Limit} \\
\hline
\hline
& \multicolumn{3}{c}{$\mathcal{B}(Z \rightarrow \Upsilon\gamma)$ $[\times10^{-6}]$}      \\
\cline{2-4}
&  $\Upsilon(1S)$ & $\Upsilon(2S)$ & $\Upsilon(3S)$  \\
\hline
Expected     & $6.4^{+3.1}_{-2.0}$ &  $8.3^{+4.0}_{-2.5}$  & $8.0^{+3.9}_{-2.4}$            \\
Observed     & 9.0 &  12.3  & 11.4      \\
\hline
SM Prediction $[\times10^{-8}]$ & 4.8  &  2.4  & 1.9      \\
\hline
\hline
& \multicolumn{3}{c}{$\mathcal{B}(H \rightarrow \Upsilon\gamma)$ $[\times10^{-4}]$}       \\
\cline{2-4}
&  $\Upsilon(1S)$ & $\Upsilon(2S)$ & $\Upsilon(3S)$ &   \\
\hline
Expected     & $12.5^{+6.1}_{-3.9}$ &  $14.6^{+7.1}_{-4.5}$  & $13.6^{+6.6}_{-4.2}$        \\
Observed     & 11.5 &  13.6  & 12.7     \\
\hline
SM Prediction $[\times10^{-9}]$ & 5.2  &  1.4  & 0.9      \\
\hline
\hline
\end{tabular}



\begin{tabular}{l|llll}
\multicolumn{4}{c}{95\% C.L. Upper Limit - $\mathcal{B}(Z \rightarrow \Upsilon\gamma)$ $[\times10^{-6}]$} \\
\hline
\hline
& \multicolumn{3}{c}{without categorization}      \\
\cline{2-4}
&  $\Upsilon(1S)$ & $\Upsilon(2S)$ & $\Upsilon(3S)$  \\
\hline
Expected     & $1.7^{+0.9}_{-0.5}$ &  $2.1^{+1.1}_{-0.7}$  & $1.9^{+1.0}_{-0.6}$            \\
Observed     & 2.6 &  2.3  & 1.2      \\
\hline
\hline
& \multicolumn{3}{c}{with categorization}      \\
\cline{2-4}
&  $\Upsilon(1S)$ & $\Upsilon(2S)$ & $\Upsilon(3S)$  \\
\hline
Expected     & $1.6^{+0.8}_{-0.5}$ &  $2.0^{+1.0}_{-0.6}$  & $1.8^{+1.0}_{-0.6}$            \\
Observed     & 2.9 &  2.7  & 1.4      \\
\hline
\hline
\end{tabular}
	
\caption{Summary table for the limits on branching ratio of $\mathrm{Z}\to\Upsilon(1S,2S,3S)\gamma$, for the two possible categorization scenarios.}
\label{tab:UpperLimits_Cat0}
\end{center}
\end{table}


It is worth to remember that the categorization takes places only for the Z decay. For the Higgs decay, no categorization is imposed.


\clearpage
