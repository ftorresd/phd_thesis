\chapter{Conclusion and perspectives}
\label{chapter_conclusion_and_perspectives}

In this document it was presented an analysis of the $H/Z \rightarrow \Upsilon + \gamma$, with 2016 data sample of the CMS detector, at center-of-mass energy $\sqrt{s}=13$ TeV. The obtained upper limits (Table~\ref{UpperLimits_Cat123_conclusion}), show good agreement with the Standard Model predictions and are compatible with previous measurements from other LHC experiments. Future developments of this analysis would be the measurement of the same upper limits considering the fully available statistics of CMS Run2 (2016, 2017 and 2018), the extrapolation of these results to the expected full CMS luminosity (3000 $fb^-1$) and an evaluation, using DELPHES~\cite{delphes},  of the sensitivity of future colliders, such as the International Linear Collider (ILC)~\cite{ilc} or the FCC~\cite{fcc}, to this decay.

\begin{table}[ht]
    \begin{center}
    \caption{Summary table for the limits on branching ratio of $\mathrm{Z}\to\Upsilon(1S,2S,3S)\gamma$ and $\mathrm{H}\to\Upsilon(1S,2S,3S)\gamma$ decays.}
    %\resizebox{.5\width}{!}{\begin{tabular}{l|llll}
\multicolumn{4}{c}{95\% C.L. Upper Limit} \\
\hline
\hline
& \multicolumn{3}{c}{$\mathcal{B}(Z \rightarrow \Upsilon\gamma)$ $[\times10^{-6}]$}      \\
\cline{2-4}
&  $\Upsilon(1S)$ & $\Upsilon(2S)$ & $\Upsilon(3S)$  \\
\hline
Expected     & $6.4^{+3.1}_{-2.0}$ &  $8.3^{+4.0}_{-2.5}$  & $8.0^{+3.9}_{-2.4}$            \\
Observed     & 9.0 &  12.3  & 11.4      \\
\hline
SM Prediction $[\times10^{-8}]$ & 4.8  &  2.4  & 1.9      \\
\hline
\hline
& \multicolumn{3}{c}{$\mathcal{B}(H \rightarrow \Upsilon\gamma)$ $[\times10^{-4}]$}       \\
\cline{2-4}
&  $\Upsilon(1S)$ & $\Upsilon(2S)$ & $\Upsilon(3S)$ &   \\
\hline
Expected     & $12.5^{+6.1}_{-3.9}$ &  $14.6^{+7.1}_{-4.5}$  & $13.6^{+6.6}_{-4.2}$        \\
Observed     & 11.5 &  13.6  & 12.7     \\
\hline
SM Prediction $[\times10^{-9}]$ & 5.2  &  1.4  & 0.9      \\
\hline
\hline
\end{tabular}

}
    % \begin{tabular}{l|llll}
\multicolumn{4}{c}{95\% C.L. Upper Limit} \\
\hline
\hline
& \multicolumn{3}{c}{$\mathcal{B}(Z \rightarrow \Upsilon\gamma)$ $[\times10^{-6}]$}      \\
\cline{2-4}
&  $\Upsilon(1S)$ & $\Upsilon(2S)$ & $\Upsilon(3S)$  \\
\hline
Expected     & $6.4^{+3.1}_{-2.0}$ &  $8.3^{+4.0}_{-2.5}$  & $8.0^{+3.9}_{-2.4}$            \\
Observed     & 9.0 &  12.3  & 11.4      \\
\hline
SM Prediction $[\times10^{-8}]$ & 4.8  &  2.4  & 1.9      \\
\hline
\hline
& \multicolumn{3}{c}{$\mathcal{B}(H \rightarrow \Upsilon\gamma)$ $[\times10^{-4}]$}       \\
\cline{2-4}
&  $\Upsilon(1S)$ & $\Upsilon(2S)$ & $\Upsilon(3S)$ &   \\
\hline
Expected     & $12.5^{+6.1}_{-3.9}$ &  $14.6^{+7.1}_{-4.5}$  & $13.6^{+6.6}_{-4.2}$        \\
Observed     & 11.5 &  13.6  & 12.7     \\
\hline
SM Prediction $[\times10^{-9}]$ & 5.2  &  1.4  & 0.9      \\
\hline
\hline
\end{tabular}


    
\begin{tabular}{l|llll}
\multicolumn{4}{c}{95\% C.L. Upper Limit} \\
\hline
\hline
& \multicolumn{3}{c}{$\mathcal{B}(Z \rightarrow \Upsilon\gamma)$ $[\times10^{-6}]$}      \\
\cline{2-4}
&  $\Upsilon(1S)$ & $\Upsilon(2S)$ & $\Upsilon(3S)$  \\
\hline
Expected     & $1.6^{+0.8}_{-0.5}$ &  $2.0^{+1.0}_{-0.6}$  & $1.8^{+1.0}_{-0.6}$            \\
Observed     & 2.9 &  2.7  & 1.4      \\
\hline
SM Prediction $[\times10^{-8}]$ & 4.8  &  2.4  & 1.9      \\
\hline
\hline
& \multicolumn{3}{c}{$\mathcal{B}(H \rightarrow \Upsilon\gamma)$ $[\times10^{-4}]$}       \\
\cline{2-4}
&  $\Upsilon(1S)$ & $\Upsilon(2S)$ & $\Upsilon(3S)$ &   \\
\hline
Expected     & $7.3^{+4.0}_{-2.4}$ &  $8.1^{+4.6}_{-2.8}$  & $6.8^{+3.9}_{-2.3}$        \\
Observed     & 6.9 &  7.4  & 5.8     \\
\hline
SM Prediction $[\times10^{-9}]$ & 5.2  &  1.4  & 0.9      \\
\hline
\hline
\end{tabular}
	
    \label{tab:UpperLimits_Cat123_conclusion}
    \end{center}
\end{table}

For the Resistive Plate Chambers, it was presented contributions given to the RPC system of CMS, during the development of this study, including its maintenance and R\&D. The main challenge for the next generation of detector based on this technology is research on new gas mixtures that do not included in its composition, green houses gases. There are already developments in this direction~\cite{eco_gas}. 