% Escolha: Portugues ou Ingles ou Espanhol.
% Para a versão final do texto, acrescente a palavra "Final",
% como na segunda linha abaixo da próxima.
%\documentclass[Portugues]{tese-FT}
\documentclass[Ingles,Final]{tese-IFGW}
%\documentclass[Ingles]{tese-FT}
%\documentclass[Espanhol,Final]{tese-FT}

%Adicione seu arquivo com as referências bibliográficas
% \usepackage[sorting=none,style=nature,natbib=true]{biblatex} % Use the bibtex backend with the authoryear citation style (which resembles APA)
\addbibresource{thesis.bib}

%O pacote a seguir gera um dummy text. Elimine a linha quando
% for editar seu texto.
\usepackage{lipsum}

%----------------------------------------------------------------------------------------
%	CMS DEFS
%----------------------------------------------------------------------------------------

%\DeclareUnicodeCharacter{U+9A}{o} % to solve char problem in bib

\usepackage{lipsum}

% \usepackage{subfig}
\usepackage{caption}
\usepackage{subcaption}


% \usepackage[para,online,flushleft]{threeparttable}



% \usepackage{lineno}
% \linenumbers

% \usepackage{draftwatermark}
% \SetWatermarkLightness{0.9}
% \SetWatermarkScale{4}

\usepackage{url}
\usepackage{hyperref}
\usepackage{multirow}

\usepackage{tensor}
\newcommand{\levicivita}{}% initialize
\def\levicivita#1#{\tensor#1{\epsilon}}

\usepackage{scalerel}

\usepackage{slashed}

% \renewcommand\thefootnote{\fnsymbol{footnote}}

\usepackage{rotating}

% colored rounded box
\usepackage{tcolorbox}

% \makeatletter
% \@namedef{ver@transparent.sty}{}
% \makeatother
% \usepackage{ctable} % for \specialrule command

%\usepackage{supertabular} % for \topcaption command
\usepackage{topcapt}

% ident first paragraph of chapter or section
\usepackage{indentfirst}

% CMS definitions
\usepackage{ifthen}
% switches to handle document styles: one of these must be specified.
% pas is the same as as.
\newboolean{cms@tdr}
\setboolean{cms@tdr}{false}
\newboolean{cms@note}
\setboolean{cms@note}{false}
\newboolean{cms@an}
\setboolean{cms@an}{false}
\newboolean{cms@in}
\setboolean{cms@in}{false}
\newboolean{cms@cr}
\setboolean{cms@cr}{false}
\newboolean{cms@pas}
\setboolean{cms@pas}{false}
\newboolean{cms@dn}
\setboolean{cms@dn}{false}
\newboolean{cms@paper}
\setboolean{cms@paper}{false}
% indicates final version. remove any draft/internal use markers.
\newboolean{cms@final}
\setboolean{cms@final}{false}
% to use a footnote for the collaboration list
\newboolean{cms@collab}
\setboolean{cms@collab}{false}
%to allow external commands
\newboolean{cms@external}
\setboolean{cms@external}{false}
%to allow for italic formatting
\newboolean{cms@italic}
\setboolean{cms@italic}{false}
% to use formatting for a supplement
\newboolean{cms@supplement}
\setboolean{cms@supplement}{false}
% to remove linenumbers for a non @final version
\newboolean{cms@nolineno}
\setboolean{cms@nolineno}{false}

% double hat
\usepackage{amsmath}
\usepackage{accents}
\newlength{\dhatheight}
\newcommand{\doublehat}[1]{%
    \settoheight{\dhatheight}{\ensuremath{\hat{#1}}}%
    \addtolength{\dhatheight}{-0.35ex}%
    \hat{\vphantom{\rule{1pt}{\dhatheight}}%
    \smash{\hat{#1}}}}


\newcommand{\cPZJpsill}{\ensuremath{\cPZ\to\cPJgy\,\ell^+\ell^-}\xspace}
\newcommand{\cPZJpsimm}{\ensuremath{\cPZ\to\cPJgy\,\Pgmp\Pgmm}\xspace}
\newcommand{\cPZJpsiee}{\ensuremath{\cPZ\to\cPJgy\,\Pep\Pem}\xspace}
\newcommand{\cPZpsill}{\ensuremath{\cPZ\to\psi\,\ell^+\ell^-}\xspace}
\newcommand{\cPZpsimm}{\ensuremath{\cPZ\to\psi\,\Pgmp\Pgmm}\xspace}
\newcommand{\cPZpsiee}{\ensuremath{\cPZ\to\psi\,\Pep\Pem}\xspace}
\newcommand{\psill}{\ensuremath{\psi\,\ell^+\ell^-}\xspace}
\newcommand{\cPZfourmu}{\ensuremath{\cPZ\to\Pgmp\Pgmm\Pgmp\Pgmm}\xspace}
\newcommand{\psitomm}{\ensuremath{\psi\to\Pgmp\Pgmm}\xspace}
\newcommand{\Jpsitomm}{\ensuremath{\cPJgy\to\Pgmp\Pgmm}\xspace}
\newcommand{\lplm}{\ensuremath{\ell^+\ell^-}\xspace}
\newcommand{\mpmm}{\ensuremath{\Pgmp\Pgmm}\xspace}
\newcommand{\epem}{\ensuremath{\Pep\Pem}\xspace}

% \newcommand{\todo}[1]{
% \textcolor{red}{\textbf{#1}}
% }
% \usepackage[colorinlistoftodos]{todonotes}



% \newcommand{\CLs}[0]{
% $CL_{s}$
% }

\usepackage{ptdr-definitions}

%\usepackage{siunitx}
%\newcommand{\pt}{p_{T}}
%\newcommand{\PT}{p_{T}}
%\newcommand{\MADGRAPH}{MADGRAPH}
%\newcommand{\POWHEG}{POWHEG}
%%\newcommand{\Z}{Z}
%\def\Z{Z}
%\def\unit#1{ #1}



%% The amssymb package provides various useful mathematical symbols
\usepackage{amssymb}
%----------------------------------------------------------------------------------------
% Adicione o pacote a seguir, se seu texto tiver uma lista de 
% símbolos. Não confunda com a lista de abreviaturas.
%\usepackage[symbols,nogroupskip,sort=none]{glossaries-extra}

\begin{document}

% Escolha entre autor ou autora:
\autor{Felipe Torres da Silva de Araujo}
%\autora{Nome da Autora}

% Sempre deve haver um título em português:
\titulo{Estudo sobre decaimentos dos bósons Z e Higgs em $\Upsilon + \gamma$ para colisões pp no CMS/LHC}

% Se a língua for o inglês ou o espanhol defina:
\title{Search for Z and Higgs boson decaying into $\Upsilon + \gamma$ in pp collisions at CMS/LHC}

% Escolha entre orientador ou orientadora e inclua os títulos:
\orientador{Prof. Dr. José Augusto Chinellato}
% \orientadora{Profa. Dr. Alberto Franco de Sá Santoro}

% Escolha entre coorientador ou coorientadora, se houver, 
% e inclua os títulos:
\coorientador{Profa. Dr. Alberto Franco de Sá Santoro}
% \coorientadora{Prof. Dra. Eng. Lic. Nome da Co-Orientadora}

% Escolha entre uma das quatro opções a seguir (comente as demais):
%\bsi         % para Trabalho de Conclusão de Curso em BSI
%\tads       % para Trabalho de Conclusão de Curso em TADS
%\qualificacaoMestrado  % Para textos de qualificação de mestrado.
%\qualificacaoDoutorado % Para textos de qualificação de doutorado.
% \mestrado   % para Dissertação de Mestrado em Tecnologia
\doutorado  % para Tese de Doutorado em Tecnologia

%Defina a área de concentração. Se for TCC, deixe comentado
\areaConcentracao{Física}
%\areaConcentracao{Ambiente}
%\areaConcentracao{Ciência dos Materiais}

% Se houve cotutela, defina:
%\cotutela{Universidade Nova de Plutão}

%Defina a data da defesa no formato {Dia}{Mês}{Ano}
\datadadefesa{20}{04}{2021}

% Para a versão final defina:
% Repita o nome do Orientador(a) no primeiro avaliador
\avaliadorA{Prof. Dr. Nome do Orientador}{IFCH/UNICAMP}
\avaliadorB{Profa. Dra. Segunda Avaliadora}{Instituição da segunda avaliadora}
\avaliadorC{Dr. Terceiro Avaliador}{Instituição do terceiro avaliador}
% \avaliadorD{Prof. Dr. Quarto Avaliador}{Instituição do quarto avaliador}
% \avaliadorE{Prof. Dr. Quinto Avaliador}{Instituição do quinto avaliador}
% \avaliadorF{Prof. Dr. Sexto Avaliador}{Instituição do sexto avaliador}
% \avaliadorG{Prof. Dr. Sétimo Avaliador}{Instituição do sétimo avaliador}
% \avaliadorH{Prof. Dr. Oitavo Avaliador}{Instituição do oitavo avaliador}


% Para incluir a ficha catalográfica em PDF na versão final, 
% copie o arquivo PDF, descomente e ajuste a linha a seguir:
%\fichacatalografica{arquivo.pdf}

% Este comando deve ficar aqui:
\paginasiniciais

% Se houver dedicatória, descomente a linha a seguir:
%\prefacesection{Isso é um teste}
%A dedicatória deve ocupar uma única página.
%
%
% Se houver epígrafe, descomente e edite as linhas a seguir:
\begin{epigrafe}
    \noindent{\itshape Sometimes science is a lot more art than science. A lot of people don't get that.}\bigbreak

    \hfill Rick Sanchez
    
    \vspace*{0.05\textheight}
    
    \noindent{\itshape Então, que seja doce. Repito todas as manhãs, ao abrir as janelas para deixar entrar o sol ou o cinza dos dias, bem assim, que seja doce. Quando há sol, e esse sol bate na minha cara amassada do sono ou da insônia, contemplando as partículas de poeira soltas no ar, feito um pequeno universo; repito sete vezes para dar sorte: que seja doce que seja doce que seja doce e assim por diante. Mas, se alguém me perguntasse o que deverá ser doce, talvez não saiba responder. Tudo é tão vago como se fosse nada.}\bigbreak
    
    \hfill Caio Fernando Abreu
\end{epigrafe}

% dedicatoria
% \prefacesection{Dedication}

\begin{dedicatoria}
  \hfill {\itshape Para minha mãe\ldots}
\end{dedicatoria}

%
%
% Adicione no arquivo "agradecimentos.tex" os seus agradecimentos
% Caso prefira omitir os agradecimentos, comente a linha a seguir.
\prefacesection{Acknowledgements}
I would like to thank:
\begin{itemize}
  \item the Campinas State University for providing the institutional support for this study;
  \item the Rio de Janeiro State University for the cooperation with Campinas State University in their high-energy physics program. This was a key factor for this study;
  \item the HEPGRID - High Energy Physics GRID (CMS computing cluster of Rio de Janeiro State University) for providing the computing resources;
  \item the Ministry of Science, Technology and Innovation” and the “National Council for Scientific and Technological Development – CNPq”;
  \item the European Laboratory for Particle Physics (CERN) for the construction and operation of the Large Hadron Collider (LHC);
  \item the Compact Muon Solenoid (CMS) collaboration for the construction, operation and provision of the instrumental means for this study;
  \item my colleagues from CMS-RPC, specially Mehar, Jan, Andres, Nicolas, Gabriella, Mapse and Kevin. Such welcome environment, created a lot of learning opportunities, which I will take for life;
  \item my colleagues from Rio de Janeiro University, specially my friends, Eliza, Sandro, Dilson and Sheila. This work is gladly shared with them;
  \item my advisor, Prof. José Augusto Chinellato. His kindness, guidance and trust in my work made this a reality. I will thankfully take his lessons to my carrier;
  \item my co-advisor, Prof. Alberto Santoro, with all my admiration for his successful carrier and achievements as scientist. My pleasure to be his friend;
  \item my wife, Yasmin, for her love, unrestricted support and dedication. Without her, this study would never be possible;
  \item my mother, Doralice, for her unconditional love, support and inspiration.
\end{itemize}



% Sempre deve haver um resumo em português:

\begin{resumo}
    Esse trabalho apresenta o estudo sobre decaimentos raros de bósons do Modelo Padrão em quarkonia. O mesmo é feito em dados coletados em 2016 pelo detector CMS, com energia de centro de massa de $\sqrt{s}=13$ TeV. Decaimentos de bósons Z e Higgs em $\Upsilon(1S,2S,3S)$ e um fóton, com o subsequente decaimento do $\Upsilon(1S,2S,3S)$ em $\mu^{+} \mu^{-}$ são estudados utilizando uma luminosidade integrada de 35.86 $fb^{-1}$ em colisões próton-próton. Nenhum excesso significativo foi observado além da suposição do modelo de somente-fundo. Um limite em $95\%$ de nível de confiança é colocado na razão de ramificação dos decaimentos $Z \rightarrow  \Upsilon(1S,2S,3S) + \gamma$ de 2.9, 2.7, 1.4 $\times 10^{-6}$ e em $H\rightarrow  \Upsilon(1S,2S,3S) + \gamma$ de 6.9, 7.4, 5.8 $\times 10^{-4}$, usando o método de $CL_s$. Contribuições dadas de 2016 até 2018 para a operação, manutenção e P\&D da Fase-2 de melhorias do sistema de Câmaras de Placas Resistivas do CMS também são apresentadas. Incluindo plantões para operação do sistema, certificação de dados para controle de qualidade, melhorias e manutenção do sistema \textit{online} e manutenção do detector durante o período de manutenção conhecido como LS2.
\end{resumo}
    

% Sempre deve haver um abstract:
\input{resumo_eng}

% Se houver um resumo em espanhol, descomente as linhas a seguir:
%\begin{resumen}
% A mesma regra aplica-se.
%\end{resumen}

% A lista de figuras:
\listoffigures

% A lista de tabelas:
\listoftables

% A lista de símbolos é opcional. Não confunda a lisa de símbolos
% com a lista de abreviaturas.
% \input{listaSimbolos}

% A lista de abreviações e siglas vem a seguir.
% Dê uma olhada no pacote nomencl para ver os comandos para 
% adicionar abreviações e siglas no texto.
\renewcommand{\nomname}{Lista de Abreviações e Siglas}
\printnomenclature[3cm]

%----------------------------------------------------------------------------------------
%	ABBREVIATIONS
%----------------------------------------------------------------------------------------

\begin{abbreviations} % Include a list of abbreviations (a table of two columns)

    \textbf{CERN} & European Laboratory for Particle Physics \\
    \textbf{LHC} & Large Hadron Collider \\
    \textbf{CMS} & Compact Muon Solenoid \\
    \textbf{SM} & Standard Model \\
    \textbf{R9} & Photon R9 is a shower shape variable defined as the fraction of \\
                       & energy deposited in the 5x5 square surrounding the Super Cluster \\
                       & seed of the reconstructed photon.   \\
    \textbf{LS1, LS2} & Long-Shutdown 1 and 2. Long periods of maintenance and upgrade \\
                       & (spread over few year), in between data taking periods (Run). \\
                      &  The LHC timescale is: Run1, LS1, Run2, LS2, Run3, ...  \\
    \textbf{ECAL} & Electromagnetic Calorimeter \\
    \textbf{HCAL} & Hadronic Calorimeter \\
    \textbf{FEWZ} & Fully Exclusive W and Z Production \\
    
\end{abbreviations}

% O sumário vem aqui:
\tableofcontents

% E esta linha deve ficar bem aqui:
\fimdaspaginasiniciais

% O corpo da dissertação ou tese começa aqui:
% introduction
%==============================================================================
\chapter{Introduction}

The Standard Model (SM) have been proven successful over the last decades by its accordance with results from many collider experiments, the Large Electron–Positron Collider (LEP)~\cite{lep_tdr} and its experiments created the experimental conditions to the discovery of the electroweak bosons, $W^{\pm}$ and $Z$. The Tevatron experiments (D0 and CDF) allowed the discovery of the top quark. These were 3 of the four heaviest components of the SM. The missing piece was the, so called, Higgs Boson, or any other explanation to the mass of the other SM particles. 

In 2012, during CMS' Run1, at center-of-mass energy $\sqrt{s} = 13$ TeV, researchers from CMS~\cite{Chatrchyan:2008zzk} and ATLAS~\cite{atlas_collaboration_2008}, two collaborations with experiments located at the Large Hadron Collider (LHC), a 27 km long circular proton-proton collider build and operated by CERN, announced the discovery a new particle~\cite{higgs_discovery_cms, higgs_discovery_atlas}, with characteristics compatibles with the Brout-Englert-Higgs boson, completing the SM picture proposed up to fifty years ago. In 2013, Francois Englert and Peter Higgs were awarded with the Noble Prize for \textit{"for the theoretical discovery of a mechanism that contributes to our understanding of the origin of mass of subatomic particles, and which recently was confirmed through the discovery of the predicted fundamental particle, by the ATLAS and CMS experiments at CERN's Large Hadron Collider"~\cite{noble_prize}.}

On top of the success of the Higgs program at CMS, there is much to be understood, e.g. pin down the coupling constants of the Higgs boson with all three generations of quarks and leptons, its mass and its full width, evaluate non-zero CP-odd components in Higgs interactions, investigate double Higgs production and its self-coupling constant and possible extensions of the SM close to the Higgs sector and explore rare decays of Higgs. The former one, specially rare decays involving quarkonia, ,such as $H \rightarrow M \gamma$, where $M$ is a meson state, are a very good scenario to investigate the Higgs interaction with other SM particles other than the direct decay. This one would be overwhelmed by the immense background coming from QCD events. The same analogy can be extended to the Z boson, which also serves as a benchmark for the Higgs study.

The present study corresponds to 35.86 $fb^{-1}$ of data taken by CMS during 2016, during the Run2, at center-of-mass energy $\sqrt{s} = 13$ TeV, in which an upper limit on the branching fraction for $H/Z \rightarrow \Upsilon(1S, 2S, 3S) (\rightarrow \mu\mu) + \gamma $ is determined.

Because of its narrow resolution, muons play a special role not only for this study, but for CMS, in general. Not only the Higgs studies heavily depends of muonic final states (for decay channels, such as $H \rightarrow \mu\mu$ and $H \rightarrow ZZ \rightarrow 4l$ and identification of the production modes), but also muon final states are very important to a whole broad of physics process accessible at CMS/LHC. The Figure~\ref{dimuon_invariant_mass} presents the the distribution of dimuon invariant mass reconstructed from different double muon triggers, with different requirements in pseudorapidity and transverse momentum. It is clear how the muons at CMS broader the set of interesting process giving access to light quark hadrons to high transverse momentum phenomena.

% dimuon invariant mass
\begin{figure}[htbp]
    \centering
    \includegraphics[width=0.7\textwidth]{figures_and_tables/introduction/dimuon_inv_mass.pdf}
    \caption{Dimuon mass distribution collected with various dimuon triggers. The light gray continuous distribution represents events collected wit inclusive dimuon triggers with high $p_T$ thresholds. The dataset corresponding to an integrated luminosity of 13.1 $fb^{-1}$ was collected during the 25 ns LHC  running period at 13 TeV in 2016. Source:~\cite{dimuon_inv_mass}.}
    % \caption{Dimuon mass distribution collected with various dimuon triggers. The light gray continuous distribution represents events collected wit inclusive dimuon triggers with high $p_T$ thresholds. The dark gray band is collected by a trigger with low-mass non-resonant dimuon plus a track. The other colored spectra are acquired using specialized triggers which require a pair of muons with opposite charge, a vertex-fit probability greater than 0.5\%, and specific dimuon invariant mass and $p_T$ regions: Magenta: dimuon mass within (0.85, 1.2) GeV, dimuon $p_T > 0$ GeV, both muons $|\eta| < 1.6$. Red: dimuon mass within (2.95, 3.3) GeV, dimuon $p_T > 16$ GeV; or dimuon mass within (2.95, 3.3) GeV, dimuon $p_T > 10$ GeV, both muons $|\eta| < 1.6 $. Blue: dimuon mass within (3.4, 4) GeV, dimuon $pT > 13$ GeV; or dimuon mass within (3.4, 4) GeV, dimuon $p_T > 8$ GeV, both muons $|\eta| < 1.6$. Cyan: dimuon mass within (4.5, 6) GeV, the leading muon $p_T > 4$ GeV and the sub-leading muon $p_T > 3$ GeV. Green: dimuon mass within (8.5, 11) GeV, dimuon $p_T > 13$ GeV; or dimuon mass within (8.5, 11) GeV, dimuon $p_T > 8$ GeV, both muons $|\eta| < 1.6$. The dataset corresponding to an integrated luminosity of 13.1 $fb^{-1}$ was collected during the 25 ns LHC  running period at 13 TeV in 2016. Source:~\cite{dimuon_inv_mass}.}
    \label{dimuon_invariant_mass}
\end{figure}

In this scenario, a contribution to the muon system of CMS is a meaningfully one to the collaboration. In this document it is described the contributions given to Resistive Plate Chamber (RPC) subdetector, including its commissioning, instrumentation for its upgrade, operation and maintenance.

This document is organized as follows: Chapter 1 is this introduction. Chapter 2 is devoted to a review of the theoretical foundations of this study and the motivations for the study of Rare Z and Higgs decays involving quarkonia. Chapter 3 is a review of the collider and experimental setup, LHC and CMS respectively. Chapter 4 is a detailed description of the data sample and the applied analysis procedure, as well as the statistical modeling and the branching fraction upper limit extraction. Chapter 5 is a reviews of the Resistive Plate Chamber technology for muon detection at CMS and the details of the contributions given to this subdetector.

Wherever figures and tables' sources are not provided, the source is the author himself.

In this document, the convention of natural units is implicitly used: the vacuum speed of light ($c$), the reduced Planck constant ($\hbar$) and electric permittivity ($\epsilon_{0}$) are normalized to unity. In this way, SI units are:
\begin{itemize}
    \setlength\itemsep{-0.5em}
    \item mass ($[m]$) = GeV,
    \item energy ($[E]$) = GeV,
    \item momemtum ($[p]$) = GeV,
    \item mass ($[m]$) = GeV,
    \item time ($[t]$) = 1/GeV,
    \item length ($[s]$) = 1/GeV.
\end{itemize}
 
% theory
%%%%%%%%%%%%%%%%%%%%%%%%%%%%%%%%%%%%%
\chapter{Standard Model and rare Z and Higgs decays to quarkonia}
\label{chaptertheory}

\section{Standard Model and Local Gauge Invariance}
\label{section_sm}

% The human quest for understanding the Universe, goes through taking a complex problem and break it down into smaller (simpler) ideas, that, when stacked together, gives rise to a proper explanation of a phenomenon and, in the best scenario, allows us to make predictions about unexpected or less known aspects of subject. The Standard Model (SM) embodies this idea and attempts not only to explain the Universe in fundamental process (interactions), but also in terms of fundamentals components (particles). To what it proposes to explain, the Standard Model have been proven very effective.
 
Physics understands matter and how it interacts in terms of two components: fundamentals forces and elementary particles. From the weakest to the strongest, the fundamental forces are: Gravitational, Weak, Electromagnetic and Strong. All share common characteristics like, being mediated by particles~\footnote{There is no evidence of the existence of the Graviton (force carrier associated to the gravitational force), even though, it is theorized by models that wish to comprehend gravity in a quantum perspective.}, being relevant within some effective range and have a associate a charge-like quantity (i.e. an intrinsic characteristic of the object) that defines whether or not, particles might be subjected to a specific interaction.

Along with the fundamental interactions, the Standard Model~\cite{burgess_moore_2006, oguri_qcd, Halzen:1984mc, Aitchison:2004cs} (or simply \textit{SM}) defines every existing matter in the Universe as a set of fundamental quantum objects, with properties that prescribes their interaction. Those objects are said to be fundamental since, in the context of the SM, they are the smallest possible components of matter. We shall refer to them as \textit{Fundamental Particles}. There four of those mediating particles (force carriers), gluon ($g$ - for the strong interaction), photon ($\gamma$ - for the electromagnetic interaction), Z and W (for weak interaction), all of them being vector bosons (spin 1). Besides the interaction mediators, described at Table~\ref{fundamental_forces}, the fundamental particles are divided in two groups (\textit{quarks} and \textit{leptons}), with three generations, each. These are not force carriers, but elementary particles, endowed with charge-like characteristics that allow them to interact by exchange the vector bosons. Those are the building blocks of Matter in our Universe.

Figure~\ref{sm_summary} summarizes their properties. Table~\ref{fundamental_forces} presents the relative strength and effective range, for each on of the four fundamental interactions. It is important to stress that, the gravitational force is not study subject of the Standard Model.

\begin{figure}[!htbp]
  \begin{center}
  \includegraphics[width=0.6\textwidth ]{figures_and_tables/theory/sm.png}
  \end{center}\vspace*{-.5cm}
  \caption{Elementary particles of the Standard Model, with their masses charges and spin. Those particles can be divided in two classes: boson (the interaction/force carriers) and the fermions, which are divided in three generations. Source:~\cite{fig_sm_summary}.}
  \label{sm_summary}
  \end{figure}
 

\begin{table}[htp]
  \begin{center}
      \caption{Relative strength (with respect to the strong force) and effective range of action for the four fundamentals interactions.}
    \begin{tabular}{ cccc }
       & Mediator & Relative Strength & Effective Range \\ \hline
      Gravitational & Graviton & $10^{-41}$ & $\infty$ \\ 
      Weak & W and Z & $10^{-16}$ & $10^{-18}$ m \\ 
      Electromagnetic & Photon & $10^{-3}$ & $\infty$ \\ 
      Strong & Gluon & $1$ & $10^{-15}$ m\\ \hline
      \end{tabular}
  \label{fundamental_forces}
  \end{center} 
  \end{table}


There are six quarks, up and down ($u$ and $d$ - first generation), charm and strange ($c$ and $s$ - second generation), top and bottom ($t$ and $b$ - third generation), in increasing invariant mass order of the generations. Since they interact through all the three fundamental forces of the SM, they are said to possess electrical charge (for the electromagnetic interaction), flavour (for the weak interaction) and color (for the strong). Their generational counterparts, the leptons, don't interact via strong force, that is why they are said to have only flavour and electric charge. The leptons are electron and electron neutrino ($e$ and $\nu_e$ - first generation), muon and muon neutrino ($\mu$ and $\nu_{\mu}$ - second generation) and tau and tau neutrino ($\tau$ and $\nu_{\tau}$ - third generation). The neutrinos, within the SM, are massless, even though, experimental measurements have shown that they actually have mass~\cite{pdg_2020}. Neutrinos are also electrically neutral, meaning that they only interact through weak interactions.

Figure~\ref{sm_summary} also presents the Higgs Boson ($H$) which is part of the SM and shall be discussed later.

\subsection{Local Gauge Invariance}

Within the Standard Model, the theoretical basis that describe the fundamental interactions are derived from a common principle: the local gauge invariance. According to Salam and Ward~\cite{ward_salam}:

\begin{quote}
  "Our basic postulate is that it should be possible to generate strong, weak and electro-magnetic interaction terms [...], by making local gauge transformations on the kinetic-energy terms in the free Lagrangian for all particles."
\end{quote}

Taking the Quantum Electrodynamics (QED) as an example: the quantum field theory that describes the electromagnetic interactions, consider the Dirac equation, in the covariant form, for a particle with mass $m$, charge $-e$ and spin $1/2$, i.e. a electron:

\begin{equation}
    (i \gamma^\mu \partial_\mu + m)\psi(x) = 0,
    \label{dirac_equation}
\end{equation}
where $\psi(x)$ is a spinor, describing the wave-function and $\gamma^\mu$ are gamma-matrices. This equation can be obtained from the lagrangian $\mathcal{L}$~\footnote{Even though, the $\mathcal{L}$ actually represents the lagrangian density, in this document we shall refer to it as simply lagrangian.} of a free particle, in the form of 

\begin{equation}
    \mathcal{L_{\text{0}}} = i\bar{\psi}(x)\gamma^\mu\partial_\mu\psi(x)-m\bar{\psi}\psi(x),
    \label{lagrangian_free_particle}
\end{equation}
when applied to the Euler-Lagrange equation.

It is clear that, the Dirac Equation (\ref{dirac_equation}) and its lagrangian (\ref{lagrangian_free_particle}) are invariant under a global phase transformation.

\begin{equation}
    \psi(x) \rightarrow \psi'(x) = \exp{(-ie\alpha)}\psi(x),
    \label{global_phase_transformation}
\end{equation}
where is a constant (global phase shift).

The same is not true when $\alpha$ is not a constant, but actually a local phase transformation, a gauge transform.

\begin{equation}
    \psi(x) \rightarrow \psi'(x) = \exp{(-ie\alpha(x))}\psi(x)
    \label{local_phase_transformation}
\end{equation}

In this case, the derivative of $\alpha(x)$ will introduce a new term that would break the invariance. To recover it, the covariant derivative operator should be modified as follows:

\begin{equation}
    \partial_\mu \rightarrow D_\mu = \partial_\mu - ieA_\mu.
    \label{covariant_derivative_modification}
\end{equation}

This modification introduces the concept of the gauge field $A_\mu$, associated to a particle of spin 1 and zero mass, the photon. This term should transform under gauge, in the following manner:

\begin{equation}
    A_\mu \rightarrow A'_\mu = A_\mu - \partial_\mu\alpha(x).
    \label{covariant_gauge_field}
\end{equation}

Modifications \ref{covariant_derivative_modification} and \ref{covariant_gauge_field} are sufficient not only to make the free particle Dirac Equation and its lagrangian gauge transformation invariant (Equations \ref{invariant_dirac_equation} and \ref{invariant_lagrangian} ), but also it naturally gives rise to an interaction term associated to the gauge field $A_\mu$.

\begin{equation}
        (i \gamma^\mu \partial_\mu + m)\psi(x) = -e\gamma_\mu A_\mu(x) \psi(x) 
    \label{invariant_dirac_equation}
\end{equation}
\begin{equation}
    \begin{split}
        \mathcal{L} \rightarrow \mathcal{L'} &= i\bar{\psi'}(x)\gamma^\mu\ D_\mu\psi'(x)-m\bar{\psi'}\psi'(x) \\
        \mathcal{L'} &= \mathcal{L_{\text{0}}} + e\bar{\psi}(x)\gamma^\mu A_\mu \psi(x) = \mathcal{L}
    \end{split}
    \label{invariant_lagrangian}
\end{equation}

Interesting to notice that the $\mathcal{L_{\text{0}}}$, on \ref{invariant_lagrangian} term corresponds to the electron kinetic energy plus the mass contribution (the free particle lagrangian), while the second corresponds to the interaction of the electron ($\psi(x)$) and the electromagnetic field. One could add the energy contribution of the electromagnetic field itself, by adding a term like:

\begin{equation}
    \mathcal{L_{\text{EM}}} = - \frac{1}{4} F_{\mu\nu} F^{\mu\nu},
\label{lagragian_em}
\end{equation}
where:
\begin{equation}
    F_{\mu\nu} = \partial_\mu A_\nu - \partial_\nu A_\mu.
\label{f_munu_definition}
\end{equation}

It can be proven that applying \ref{f_munu_definition} on the Euler-Lagrange equations, this will give us the Maxwell's Equations for the vacuum, $\partial_{\mu} F^{\mu\nu} = 0$~\footnote{A non-vacuum covariant form of the Maxwell's Equations would be $\partial_{\mu} F^{\mu\nu} = j^{\nu}$.}. One could also expect that a field mass contribution in as below, could be introduced, as well.

\begin{equation}
    \frac{1}{2}m A_\mu A_\mu
\label{photon_mass_term}
\end{equation}

This one would break the gauge invariance, therefor we could imply that the photon should be massless.

\subsection{The Standard Model}

Taking profit of the Local Gauge Invariance as path to introduce interactions in a quantum field theory, such as for the QED, the Standard Model can be defined as 



\todo[inline]{SM as a gauge theory} 
\todo[inline]{Spontaneous Symmetry break and the Higgs Boson} 

\section{SM and Higgs results}
\label{section_sm_higgs}

The Standard Model have been proven extremely successful in describing what it is proposed to do. The discovery of the two highest invariant mass particles of the SM, the top quark~\cite{top_discovery_cdf,top_discovery_d0}, by the CDF and 0 collaboration, at FERMILAB, and the Higgs Boson~\cite{higgs_discovery_cms,higgs_discovery_atlas}, by CMS and ATLAS, at CERN, fill the two missing pieces of the SM puzzle, presented at Figure~\ref{sm_summary}. In general, SM measurements presents very good agreement between theory and experiment, even when the Higgs boson is taken into account, once it mass has been established, the subsequent results tend to be found restricted within the expectations and constrained by the statistics and experimental sensitivity.  

In this section, we shall briefly review some of the most relevant SM results from LHC, with special focus to $Z$ and Higgs boson, subjects of the study. 

\subsection{Standard Model vector bosons at CMS}
\label{section_sm_vb_results}

The success of the Standard Model relies mostly on its excellent agreement between its predictions and the measurements, even though there are still many open questions on fundamental particle physics~\cite{open_questions}, such as: How cna we explain the number of fundamental particles known so far? Why matter and antimatter appear in the Universe in different proportions? What is the astrophysical dark matter? How could we unify the fundamental interactions? How to quantize gravity? 

The Figures~\ref{sm_ewk_results},~\ref{sm_vbf_results} and~\ref{cms_sm_xsec} presents a summary of relevant CMS results on SM measurements. The former one presents the ratio between the observed and expected cross section ($\sigma_{exp}/\sigma_{theo}$) for different di-boson production at NNLO calculations and pure electroweak processes, while the later have a summary of cross section measurements made by CMS. When theory and experiment agreement is not exact, one has to take into account the experimental limitations of one experiment, such as CMS and the many possible electroweak phenomena to be studied.


\begin{figure}[htbp]
  \centering
  \begin{subfigure}[htbp]{0.48\textwidth}
    \centering
    \includegraphics[width=\textwidth]{figures_and_tables/theory/sm_vbf_results.pdf}
    \caption{}
    \label{sm_vbf_results}
  \end{subfigure}
  \hfill
  \begin{subfigure}[htbp]{0.48\textwidth}
    \centering
    \includegraphics[width=\textwidth]{figures_and_tables/theory/sm_ewk_results.pdf}
    \caption{}
    \label{sm_ewk_results}
  \end{subfigure}
  \caption{(a) Di-boson cross section ratio comparison to theory: Theory predictions updated to latest NNLO calculations where available compared to predictions in the CMS papers and preliminary physics analysis summaries. Source:~\cite{cms_sm_xsec_summary}. (b) Summary of the cross sections of pure Electroweak (EWK) interactions among the gauge bosons presented as a ratio compared to theory. Source:~\cite{cms_sm_xsec_summary}.}
\end{figure}

\begin{sidewaysfigure}[htbp]
  \centering
  \includegraphics[width=\textwidth]{figures_and_tables/theory/cms_sm_xsec.pdf}
  \caption{Summary of the cross section measurements of Standard Model processes at CMS. Source:~\cite{cms_sm_xsec_summary}.}
  \label{cms_sm_xsec}
\end{sidewaysfigure}

The open questions above are not subjected to the SM scope, but even within the SM there still relevant precision measurements~\cite{sm_global_fit} that are important to understand the validity of the SM and what other questions lies about the SM, at the threshold of the LHC experiments precision.

\clearpage 
\subsection{Higgs boson at CMS}
\label{section_sm_vb_results}

The Higgs may be produced at LHC proton-proton collisions by the following process, called \mbox{\textbf{Production Modes}}. \textit{state-of-art} SM cross section predictions were computed by the "LHC Higgs Cross Section Working Group"\cite{deFlorian:2016spz} and are presented as a function oftheHiggs mass is presented at Figure~\ref{higgs_prod_modes} and examples of leading order Feynmann diagrams of them are presented at Figure~\ref{fig_diagrams_production_modes}, for the highest cross section production modes.

The \textbf{Gluon Fusion - ggF} - is the result of a gluon-gluon interaction which is mediated by a heavy quark loop. Each quark contributing is suppressed by $1/m_{q}^{2}$. It is by far the one with highest cross section. Its final state is composed only by a Higgs boson, which makes it harder to identify, since there are no other auxiliary final state particle to tag it. In this decay, QCD radiactive corrections are very important and have been in included in the results of Figure~\ref{higgs_prod_modes} up to N3LO (next-to-next-to-next-to-leading order, while electroweak corrections are computed up to NNLO. The \textbf{Associated Vector Boson Production - VH} - a SM vector boson (Z or W) irradiate a Higgs. Due to its clear electroweak signature (a final state with a Higgs and a vector boson), this production mode enhances the signal, when the Higgs decay has a large contribution from QCD background, e.g. $H \rightarrow b\bar{b}$. This process is also called Higgs-Strahlung.

The third process is the \textbf{Vector Boson Fusion - VBFH} - in which the two quarks from the initial state scatter by the emission a pair of vector bosons (ZZ or $W{\pm}W{\mp}$). Those would interact (fuse) and produce a Higgs in the final, associated with two back-to-back jets, from the initial state quarks. The \textbf{Associated $t\bar{t}$ Production - ttH} - and \textbf{Associated $b\bar{b}$Production - bbH} are very similar process (especially in the scale of $\sqrt{s} = 13$ TeV, where their cross sections almost match), where the coupling of the heavy quark to the Higgs boson, contrary to what happens in the ggF production, it is not with a virtual state of then. 

The \textbf{Associated Single Top Production - tH} - is the production mode with the smallest cross section, due to its destructive interference with other process. Without loss of generality, it is not considered in this study.


\begin{figure}[htbp]
  \centering
  \begin{subfigure}[htbp]{0.48\textwidth}
    \centering
    \includegraphics[width=\textwidth]{figures_and_tables/theory/higgs_prod_modes.pdf}
    \caption{}
    \label{higgs_prod_modes}
  \end{subfigure}
  \hfill
  \begin{subfigure}[htbp]{0.48\textwidth}
    \centering
    \includegraphics[width=\textwidth]{figures_and_tables/theory/higgs_decays.pdf}
    \caption{}
    \label{higgs_decays}
  \end{subfigure}
  \caption{(a) Standard Model Higgs boson production cross sections at $\sqrt{s}=13$ TeV as a function of Higgs boson mass. The tH production cross section accounts for $t$-channel and $s$-channel only (no $tWH$ production). The VBF process is indicated here as $qqH$. The theoretical uncertainties are indicated as shaded bands around the lines. Source:~\cite{deFlorian:2016spz}. (b) Standard Model Higgs boson decay branching ratios for different decay channels. The theoretical uncertainties are indicated as shaded bands around the lines. Source:~\cite{deFlorian:2016spz}.}
\end{figure}

% production modes
\begin{figure}[htbp]
  \centering
  \begin{subfigure}[htbp]{0.48\textwidth}
    \centering
    \includegraphics[width=\textwidth]{figures_and_tables/theory/higgs_prod_and_decays/ggf.pdf}
    \caption{Gluon Fusion - ggF}
  \end{subfigure}
  \hfill
  \begin{subfigure}[htbp]{0.48\textwidth}
    \centering
    \includegraphics[width=\textwidth]{figures_and_tables/theory/higgs_prod_and_decays/vh.pdf}
    \caption{Associated Vector Boson Production - VH}
  \end{subfigure}
  \begin{subfigure}[htbp]{0.48\textwidth}
    \centering
    \includegraphics[width=\textwidth]{figures_and_tables/theory/higgs_prod_and_decays/vbf.pdf}
    \caption{Vector Boson Fusion - VBFH}
  \end{subfigure}
  \hfill
  \begin{subfigure}[htbp]{0.48\textwidth}
    \centering
    \includegraphics[width=\textwidth]{figures_and_tables/theory/higgs_prod_and_decays/tth.pdf}
    \caption{Associated $t\bar{t}$ Production - ttH}
  \end{subfigure}
  \caption{Example of leading order Standard Model Higgs boson production model diagrams. Source:~\cite{higgs_diagrams}.}
  \label{fig_diagrams_production_modes}
\end{figure}

The Higgs allowed \mbox{\textbf{Decay Channel}}, in the context of the Standard Model, is also a closet set, which have also been subject of study of the "LHC Higgs Cross Section Working Group"~\cite{deFlorian:2016spz}. Figure~\ref{higgs_decays} presents their expected branching ratios.

The largest branching fraction is the decay to a $b\bar{b}$ pair, which is, at $\sqrt{s} = 13$ TeV, more than the double of the next channel. The large cross section does not imply in being the most sensible channel for the Higgs observation. One has to take into account the experimental sensitivity to this final state (which rely on b-tagging techniques) and its enormous QCD background. Tagging on an specific production modes is usually explored in this kind of study~\cite{cms_higgs_to_bbar} to enhance the signal to background ratio. Similar to $b\bar{b}$, decays to other SM dileptons are also usually studied, such as dimuons~\cite{cms_higgs_mumu}, $\tau\tau$~\cite{cms_higgs_to_tautau} and  $c\bar{c}$~\cite{cms_higgs_to_ccbar}. 

Other decays include the $VV$ state, where $V$ is a electroweak vector boson ($Z$~\cite{Sirunyan:2018sgc}, $W^{\pm}$~\cite{Sirunyan:2020tzo} and $\gamma$~\cite{Sirunyan:2020xwk}). Even tough the branching fraction for these ones are relatively smaller, they offer a clear signature for event selection, with reduced QCD background. It is important to notice that $H \rightarrow Z\gamma$ also play a role in this decay mode. CMS (and ATLAS) has a very good sensitivity for leptonic final states of these bosons and for a direct measurement of photons, with resolutions to the order of 1\% for the Higgs. Other channels will have resolutions larger than 10\%~\cite{pdg_2020}.

Gluonic Higgs decays ($H \rightarrow gg$) are allowed in the Standard Model, but they would be overwhelmed by the QCD background. This is considered to be measurable only in the context of a $e^{+}e^{-}$ collider~\cite{Spira:1995rr}.

As already mentioned on Section~\ref{section_sm_higgs_results}, the Higgs was found at CMS and ATLAS in 2012, with Run1 data at $\sqrt{s} =$ 7 and 8 TeV, by investigating the $H \rightarrow ZZ \rightarrow 4l$ and $H \rightarrow \gamma\gamma$ decays. Figures~\ref{higgs_discovery_hgg} and~\ref{higgs_discovery_hzz4l} present the reconstructed final state invariant masses that lead to its discovery. Since then, a broad program have been carried out by both, ATLAS and CMS, to extend the understanding of the Higgs boson to all accessible decays, production modes and also its properties and differential cross section.

% Higgs discovery
\begin{figure}[htbp]
  \centering
  \begin{subfigure}[htbp]{0.48\textwidth}
    \centering
    \includegraphics[width=\textwidth]{figures_and_tables/theory/higgs_discovery_hgg.pdf}
    \caption{}
    \label{higgs_discovery_hgg}
  \end{subfigure}
  \hfill
  \begin{subfigure}[htbp]{0.48\textwidth}
    \centering
    \includegraphics[width=\textwidth]{figures_and_tables/theory/higgs_discovery_hzz4l.pdf}
    \caption{}
    \label{higgs_discovery_hzz4l}
  \end{subfigure}
  \label{higgs_discovery}
  \caption{(a) The diphoton invariant-mass distribution for the 7 and 8 \TeV datasets (points), with each event weighted by the predicted $S/(S+B)$ ratio of its event class. The solid and dotted lines give the results of the signal-plus-background and background-only fit, respectively. The light and dark bands represent the $\pm$1 and $\pm$2 standard deviation uncertainties respectively on the background estimate. The inset shows the corresponding unweighted invariant-mass distribution around \mbox{$m_{\gamma\gamma}$ = 125 \GeV}. Source:~\cite{higgs_discovery_cms}. (b) Distribution of the observed four-lepton invariant mass from the combined 7 and 8 \TeV data 
  for the $\PH \to \cPZ\cPZ\to 4\ell$ analysis (points).
  The prediction for the expected $\cPZ$+X and $\cPZ\cPZ(\cPZ\gamma^*)$ background are shown by the dark and light histogram, respectively. The open histogram gives the expected distribution for a Higgs boson of mass 125 \GeV. Source:~\cite{higgs_discovery_cms}.}
\end{figure}

A complete list of Higgs publications and public result from CMS can be found at~\cite{cms_higgs_publications,cms_higgs_public_results}. With the Higgs measurements being carried out per decay channel, a important effort of combination of these results in performed independently by each collaboration, as well as joint combinations. Some of the Higgs boson measurements by CMS are summarized.

The signal strength modifier is the ratio of the measured cross section or branching ratio over the expected one. 

\begin{equation}
  \mu^{i} = \frac{\sigma^{i}}{\sigma^{i}_{SM}} \qquad\qquad   \mu^{f} = \frac{\mathcal{B}^{i}}{\mathcal{B}^{i}_{SM}}, 
  \label{signal_strength_modifier}
\end{equation}
where $\sigma^{i}$ and $\mathcal{B}^{i}$ stand for the measured cross section and branching ratio of a certain production mode or decay channel, respectively. Figure~\ref{signal_strength_modifier} presents the most updated measurements of $\mu^{i}$ and  $\mu^{f}$ during Run2. The overall combined strength modifier is $\mu=1.02^{+0.07}_{-0.06}$~\cite{cms_higgs_comb_run2}, for $m_{H} = 125.09$ GeV, which shows very good agreement with the SM expectation.

% signal strength modifier - COMB RUn2
\begin{figure}[htbp]
  \centering
  \begin{subfigure}[htbp]{0.48\textwidth}
    \centering
    \includegraphics[width=\textwidth]{figures_and_tables/theory/signal_strength_modifier_prod.pdf}
    \caption{ }
    \label{signal_strength_modifier_prod}
  \end{subfigure}
  \hfill
  \begin{subfigure}[htbp]{0.48\textwidth}
    \centering
    \includegraphics[width=\textwidth]{figures_and_tables/theory/signal_strength_modifier_decay.pdf}
    \caption{ }
    \label{signal_strength_modifier_decay}
  \end{subfigure}
  \caption{Signal strength modifiers for the production modes, (a) $\mu^{i}$, and for the decay channels, (b) $\mu^{f}$. The thick (thin) black lines report the $1\sigma$ ($2\sigma$) confidence intervals. The thick blue and red lines report the statistical and systematic components of the $1\sigma$ confidence intervals. Source:~\cite{cms_higgs_comb_run2}.}
  \label{signal_strength_modifier}
\end{figure}

The Higgs mass was also subject of many study, here we quote the results on Figure~\ref{higgs_mass}~\cite{Sirunyan:2020xwk}, for Run1 and partial Run2 datasets, for both $H \rightarrow ZZ \rightarrow 4l$ and $H \rightarrow \gamma\gamma$ decays. The combined measurement is $m_H = 125.38 \pm 0.14$ GeV. This is the \textit{state-of-art} value for the Higgs mass.

% Higgs mass
\begin{figure}[htbp]
  \centering
  \includegraphics[width=0.7\textwidth]{figures_and_tables/theory/cms_higgs_mass.pdf}
  \caption{A summary of the measured Higgs boson mass in the $H \rightarrow ZZ \rightarrow 4l$ and $H \rightarrow \gamma\gamma$ decay channels, and for the combination of the two is presented here. The statistical (wider, yellow-shaded bands), and total (black error bars) uncertainties are indicated. The (red) vertical line and corresponding (grey) shaded column indicate the central value and the total uncertainty of the Run 1 + 2016 combined measurement, respectively. Source:~\cite{Sirunyan:2020xwk}.}
  \label{higgs_mass}
\end{figure}

Other properties studied comprehends its quantum numbers. The Landau-Yang theorem~\cite{Landau:1948kw,Yang:1950rg} rules out the spin-1 possibility, based on its observation on the $\gamma\gamma$ channel. All the tests conducted, so far, support the $J^P = 0^+$ hypothesis~\cite{cms_higgs_spin_tests}.

A recent very relevant Higgs result published by CMS is the evidence of the $H \rightarrow \mu\mu$ decay~\cite{cms_higgs_mumu}. In this paper it is reported an excess on data, with respect to the background only hypothesis, with 3$\sigma$ of significance. This is the first evidence of the Higgs coupling to second generation fermions. Figure~\ref{h_to_mumu_result} presents a weighted invariant mass distribution of the dimuon system ($m_{\mu\mu}$) for all the categories included in this analysis.

% H to mumu
\begin{figure}[htbp]
  \centering
  \begin{subfigure}[htbp]{0.48\textwidth}
    \centering
    \includegraphics[width=\textwidth]{figures_and_tables/theory/h_to_mumu_result.pdf}
    \caption{}
    \label{h_to_mumu_result}
  \end{subfigure}
  \hfill
  \begin{subfigure}[htbp]{0.48\textwidth}
    \centering
    \includegraphics[width=\textwidth]{figures_and_tables/theory/higgs_coups.pdf}
    \caption{}
    \label{higgs_coups}
  \end{subfigure}
  \label{h_to_mumu_result_higgs_coups}
  \caption{(a) T the $m_{\mu\mu}$ distribution for the weighted combination of all event categories used in the analysis. The upper panel is dominated by the gluon-gluon fusion categories with many data events but relatively small $S/(S + B)$. The lower panel shows the residuals after background subtraction, with the best-fit SM $H \rightarrow \mu\mu$ signal contribution with $m_H = 125.38$ GeV indicated by the red line. The measured signal strength is ${1.19^{+0.41}_{-0.39}(\mathrm{stat})^{+0.17}_{-0.16}(\mathrm{sys})}$. Source:~\cite{cms_higgs_mumu}. (b) The best-fit estimates for the reduced coupling modifiers extracted for fermions and weak bosons from the resolved $\kappa$-framework model compared to their corresponding prediction from the SM. The error bars represent 68\% CL intervals for the measured parameters. The lower panel shows the ratios of the measured coupling modifiers values to their SM predictions. Source:~\cite{cms_higgs_mumu}.}
\end{figure}

The same note also updates the coupling constant modifier by combining the new results for $H \rightarrow \mu\mu$ with previous Higgs results from Run2~\cite{cms_higgs_comb_run2}. The measured parameters are presented at Figure~\ref{higgs_coups} and they also present very good agreement with the SM prediction, where the coupling constants to fermions is proportional to the fermion mass($M_{f}$), while for electroweak boson, it is proportional to the square of the boson mass ($M_{V}$). The fit results are scaled to the reduced coupling strength modifiers, defined as $y_V=\sqrt{\kappa_V}\frac{m_V}{\nu}$ and $y_f=\kappa_f\frac{m_F}{\nu}$, where $\nu$ is the vacuum expectation value of the Higgs field of 246.22 GeV.

% \clearpage

\section{Rare Z and Higgs decays to quarkonia}
\label{section_rare_deays}

The rare decays of the Higgs boson~\cite{higgs_discovery_atlas,higgs_discovery_cms} to a quarkonium state and a photon provide a unique sensitivity to the magnitude of the Yukawa couplings of the Higgs boson to quarks~\cite{PhysRevD.88.053003, PhysRevD.90.113010,PhysRevLett.114.191803}. These couplings are difficult to access on hadronic collisions through direct decay of Higgs in quark-antiquark, due to the immense background from QCD~\cite{PhysRevD.89.033014}. 

Among the channels available to explore Yukawa's couplings of light quarks~\cite{PhysRevD.90.113010,PhysRevLett.114.191803} the prominent candidates are those with heavy-quarkonia. The rare modes of decay of the \Z boson have attracted attention focused on establishing its sensitivity to New Physics \cite{PEREZ}, being an alternative environment to investigate the Yukawa couplings of the Higgs boson.

% Several estimates of the branching ratio of the Standard Model for the decay of the vector bosons in simple vector meson + photon are available with the latest branching ratio value of the order of $10^{-8}$ and compatible with results published by the ATLAS collaboration~\cite{atlas_results_2016_data}. Thus, the objective of this physics analysis is to explore exclusive rare decays of vector bosons in the CMS experiment using the data taken during 2016.

Also, the exclusive rare decays of vector bosons (\Z, W) provide a favorable environment for testing the factorization of QCD, thus allowing an approach in a context where the power of corrections are definitely under control. The main focus of this kind of analysis are the hadronic radioactive decays, $Z\rightarrow M \gamma$, where M can be a pseudoscalar or a vector meson ($J/ \Psi, \phi, \Upsilon_{n}$). 

They offer the perfect way to explore some of the leading order properties of the light-cone distribution amplitudes (LCDAs)~\cite{Grossman2015} of several mesons, but they present a difficulty, considering that in the LHC energy scale the branching ratio of these processes is very small. There are theoretical predictions~\cite{PhysRevD.97.016009,PhysRevD.96.116014} that point out a branching ratio for several decay channels in the Standard Model, as shown in the Table \ref{fig:XsecTable}.

\begin{table}[htp]
  \begin{center}
    \caption{Summary of branching ratios for $H/Z \rightarrow \Upsilon(1S,2S,3S)+\gamma  \rightarrow \mu^{+} \mu^{-} +\gamma$ analysis. The effective cross-section will be discussed in section \ref{sec:datasets}.}
    %\resizebox{.5\width}{!}{\begin{tabular}{l|llll}
\multicolumn{4}{c}{95\% C.L. Upper Limit} \\
\hline
\hline
& \multicolumn{3}{c}{$\mathcal{B}(Z \rightarrow \Upsilon\gamma)$ $[\times10^{-6}]$}      \\
\cline{2-4}
&  $\Upsilon(1S)$ & $\Upsilon(2S)$ & $\Upsilon(3S)$  \\
\hline
Expected     & $6.4^{+3.1}_{-2.0}$ &  $8.3^{+4.0}_{-2.5}$  & $8.0^{+3.9}_{-2.4}$            \\
Observed     & 9.0 &  12.3  & 11.4      \\
\hline
SM Prediction $[\times10^{-8}]$ & 4.8  &  2.4  & 1.9      \\
\hline
\hline
& \multicolumn{3}{c}{$\mathcal{B}(H \rightarrow \Upsilon\gamma)$ $[\times10^{-4}]$}       \\
\cline{2-4}
&  $\Upsilon(1S)$ & $\Upsilon(2S)$ & $\Upsilon(3S)$ &   \\
\hline
Expected     & $12.5^{+6.1}_{-3.9}$ &  $14.6^{+7.1}_{-4.5}$  & $13.6^{+6.6}_{-4.2}$        \\
Observed     & 11.5 &  13.6  & 12.7     \\
\hline
SM Prediction $[\times10^{-9}]$ & 5.2  &  1.4  & 0.9      \\
\hline
\hline
\end{tabular}

}
    %%\begin{table}[htp]
%%\begin{center}
%\begin{tabular}{|c|l|l|}
%\hline
%Description of the Physics Processes  : & Cross Section ( $\sigma$ in pb) & Branching Ratio (BR$_{SM}$): \\ \hline
%H$\rightarrow  \Upsilon(1S) +\gamma$ & 7.1996$\times 10^{-9}$ &5.22$\times 10^{-9}$ \\ \hline
%H$\rightarrow  \Upsilon(2S) +\gamma$ & 1.5242$\times 10^{-9}$ &1.42$\times 10^{-9}$ \\ \hline
%H$\rightarrow  \Upsilon(3S) +\gamma$ & 1.1033$\times 10^{-9}$ &9.10$\times 10^{-10}$ \\ \hline \hline
%%Z$\rightarrow J/ \psi +\gamma$ & 9.9123$\times 10^{-6}$ & 2.99$\times 10^{-6}$ \\ \hline
%Z$\rightarrow  \Upsilon(1S) +\gamma$ & 6.7965$\times 10^{-5}$ &4.88$\times 10^{-8}$ \\ \hline
%Z$\rightarrow  \Upsilon(2S) +\gamma$ & 2.6887$\times 10^{-5}$ &2.44$\times 10^{-8}$ \\ \hline
%Z$\rightarrow  \Upsilon(3S) +\gamma$ & 2.3400$\times 10^{-5}$ &1.88$\times 10^{-8}$ \\ \hline \hline
%H$\rightarrow \gamma\gamma^{*}$ Dalitz Decay & 1.8614$\times 10^{-3}$ & 3.83$\times 10^{-5}$ \\ \hline
%Z$\rightarrow  \mu\mu\gamma_{FSR}$ & 7.9260$\times 10^{-2}$& --- \\ \hline
%
%\end{tabular}
%%\caption{Summary of data samples used for $H/Z \rightarrow \Upsilon(1S,2S,3S)+\gamma$ analysis }
%%\label{Tablebkg}
%%\end{center}
%%\end{table}
%% Ref latex: https://tex.stackexchange.com/questions/112343/beautiful-table-samples
%



%\begin{table}[htp]
%\begin{center}
\begin{tabular}{c|c}
\hline
Physics Processes & Branching Ratio (BR$_{SM}$): \\ \hline
H$\rightarrow  \Upsilon(1S) +\gamma$ & 5.22$\times 10^{-9}$ \\ \hline
H$\rightarrow  \Upsilon(2S) +\gamma$ & 1.42$\times 10^{-9}$ \\ \hline
H$\rightarrow  \Upsilon(3S) +\gamma$ & 9.10$\times 10^{-10}$ \\ \hline \hline
Z$\rightarrow  \Upsilon(1S) +\gamma$ & 4.88$\times 10^{-8}$ \\ \hline
Z$\rightarrow  \Upsilon(2S) +\gamma$ & 2.44$\times 10^{-8}$ \\ \hline
Z$\rightarrow  \Upsilon(3S) +\gamma$ & 1.88$\times 10^{-8}$ \\ \hline 
%H$\rightarrow \gamma\gamma^{*}$ Dalitz Decay & 1.8614$\times 10^{-3}$ & 3.83$\times 10^{-5}$ \\ \hline
%Z$\rightarrow  \mu\mu\gamma_{FSR}$ & 7.9260$\times 10^{-2}$& --- \\ \hline

\end{tabular}
%\caption{Summary of data samples used for $H/Z \rightarrow \Upsilon(1S,2S,3S)+\gamma$ analysis }
%\label{Tablebkg}
%\end{center}
%\end{table}
% Ref latex: https://tex.stackexchange.com/questions/112343/beautiful-table-samples


    %\caption{Summary of cross section and branching ratio for $H/Z \rightarrow \Upsilon(1S,2S,3S)+\gamma  \rightarrow \mu^{+} \mu^{-} +\gamma$ analysis. Assuming that of $\sigma (pp\rightarrow$ H) is 55.614 $pb^{-1}$ and  $\sigma (pp\rightarrow Z \rightarrow \mu\mu$ ) is 57094.5 $pb^{-1}$ with the phase space selection in invariant mass of the dimuon system of $m_{\mu\mu} > 50 GeV$ (Missing References). }
    \label{fig:XsecTable}
  \end{center}
\end{table}





%Table
% https://docs.google.com/spreadsheets/d/1zP8P9kp-yFrkMu9bGt4fpIirKAKYlH-w2em_kVkYYKw/edit?usp=sharing
%Pag2

Recent studies on exclusive Higgs boson decays \cite{ISIDORI2014131,PhysRevLett.114.101802,GAO2014366} in final states containing a vector meson and a photon have caused interest in these physics topics. It was proposed to use these decays as a possible way to explore non-standard Yukawa couplings of the Higgs boson. Such measures are quite challenging in the LHC environment. The observation of hadronic decays of vector bosons could provide a new frontier for the nature of heavy quarkonia production in hadronic collisions.

Along the same lines, the simple exploration of rare SM decays, even in scenarios where anomalous couplings are, in principle, ruled out by direct measurements~\cite{cms_h_to_bb_PhysRevLett.121.121801}, as in the case of this analysis ($H \rightarrow \Upsilon(nS) + \gamma$), are still important as a stress test of the SM and as reference for future measurements. Specially the later one, when you consider that the small predicted cross sections from Table~\ref{fig:XsecTable}, most probably, would imply that an observation of this decay would be unlikely even in the HL-LHC~\cite{hl_lhc}.

This measurement is sensitive to the direct and indirect production (Figure~\ref{direct_indirect}). The \textit{direct} process consists in the decay of boson (Higgs or Z) to a quark anti-quark pair, in which, one of the quarks radiates a photon and the pair hadronizes to a meson (a $\Upsilon(nS)$, for this study), while in \textit{indirect} process, the decay happens to a $\gamma \gamma^{*}(Z)$, with the subsequent decay of the $\gamma^{*}(Z)$ to a quark anti-quark that hadronizes. 


% direct and indirect decays
\begin{figure}[htbp]
  \centering
  \begin{subfigure}[htbp]{0.4\textwidth}
    \centering
    \includegraphics[width=\textwidth]{figures_and_tables/theory/diagrams/indirect}
    \caption{Indirect decay.}
    \label{indirect}
  \end{subfigure}
  \hfill
  \begin{subfigure}[htbp]{0.4\textwidth}
    \centering
    \includegraphics[width=\textwidth]{figures_and_tables/theory/diagrams/direct}
    \caption{Direct decay.}
    \label{direct}
  \end{subfigure}
  \caption{Example of leading order diagrams for the indirect and direct production mechanisms. The dashed blob should be understood as the $\Upsilon(nS)$, where $n=\text{1, 2, 3}$. In the indirect diagram, the majority of the contribution comes from a top loop.}
  \label{direct_indirect}
\end{figure}


Clearly, only the direct process is sensible to the Yukawa coupling of the boson with the quarks, but, since both processes are indistinguishable in their final state, the indirect process needs to be taken into account. In this study, a dimuon final state is used to tag the $\Upsilon(nS)$.

Even though there is different theoretical predictions for the cross section of this process and its twin brother ($H \rightarrow \text{J/}\Psi + \gamma$), each one taking into account different levels of complexity, the 2013 paper~\cite{PhysRevD.88.053003}, from G. Bodwin, F. Petriello, S. Stoynev and M. Velasco, summarizes very well and in a simpler manner, the most relevant phenomenological results on these decays. For the decay to $J/\Psi + \gamma$, the quantum interference with the indirect amplitude, enhances the directed production, leading to a larger, and potentially observable, cross section. This is not true for the $\Upsilon(nS) + \gamma$ decay, since the interference is destructive, diminishing the cross sections. 

Another interesting aspect of this study is that, for both $Hc\bar{c}$ and $Hb\bar{b}$ direct coupling measurements are not sensible to the sign of the Yukawa coupling, while the presence of the indirect process in the $H \rightarrow M + \gamma$ ($M$ standing for J/$Psi$ or $\Upsilon(nS)$) decays resolve this ambiguity.

Finally, since the $\Upsilon(nS) + \gamma$ decay has a much smaller cross section, because of the destructive quantum interference between direct and indirect production mechanisms, a small deviation in the $Hb\bar{b}$ Yukawa coupling, can lead a large increase in the expected branching ratio, making this channel sensible any non-Standard Model process that might interfere in this final state. This becomes clear when we look to Figure~\ref{hbb_coup}.

\begin{figure}[!htbp]
  \begin{center}
    \includegraphics[width=0.6\textwidth ]{figures_and_tables/theory/hbb_coup.png}
  \end{center}\vspace*{-.5cm}
  \caption{Expected relative variation of the branching ratio for the $H \rightarrow \Upsilon(nS) + \gamma$ to $k_b$, where $k_b = g(Hb\bar{b})/g(Hb\bar{b})_{SM}$ is the ratio for the observed and expected Yukawa coupling of $Hb\bar{b}$. Source:~\cite{PhysRevD.88.053003}}
  \label{hbb_coup}
\end{figure}


\section{Recent results}

The ATLAS experiment~\cite{atlas_collaboration_2008} already has two results on this decays~\cite{atlas_paper:PhysRevLett.114.121801, atlas_paper_2018:2018txb}. The first one corresponds to data taken from 2015, while the latter one, corresponds to data from 2016 (the same data taking period to which this study refers).

To what concerns the most updated result, the study corresponded to 36.1 $fb^{-1}$ at $\sqrt{s} = 13$ TeV and no significant excess was found by the experiment. Upper limits for the decay were obtained, assuming the Standard Model branching fractions predictions, at 95\% confidence level, according to Table~\ref{tab:atlas_results_2018}.


\begin{table}[htp]
  \begin{center}
    
    \caption{Observed upper limits, by the ATLAS experiment~\cite{atlas_paper:PhysRevLett.114.121801, atlas_paper_2018:2018txb}, on the branching fractions for the Higgs and Z decays (last result). Detailed comparisons with the results obtained in this study will be presented in Section~\ref{chaper_results}.}
    \begin{tabular}{cc}
      \hline
      Decay & $\mathcal{B}F$ at 95\% \CL \\ \hline
      H$\rightarrow  J/\Psi +\gamma$ & < 4.5$\times 10^{-4}$ \\ 
      H$\rightarrow  \Psi(2S) +\gamma$ & < 2.0$\times 10^{-3}$ \\ 
      H$\rightarrow  \Upsilon(1S) +\gamma$ & < 4.9$\times 10^{-4}$ \\ 
      H$\rightarrow  \Upsilon(2S) +\gamma$ & < 5.9$\times 10^{-4}$ \\ 
      H$\rightarrow  \Upsilon(3S) +\gamma$ & < 5.7$\times 10^{-4}$ \\          
      \hline \hline
      Z$\rightarrow  J/\Psi +\gamma$ & < 2.3$\times 10^{-6}$ \\ 
      Z$\rightarrow  \Psi(2S) +\gamma$ & < 4.5$\times 10^{-6}$ \\ 
      Z$\rightarrow  \Upsilon(1S) +\gamma$ & < 2.8$\times 10^{-6}$ \\ 
      Z$\rightarrow  \Upsilon(2S) +\gamma$ & < 1.7$\times 10^{-6}$ \\ 
      Z$\rightarrow  \Upsilon(3S) +\gamma$ & < 4.8$\times 10^{-6}$ \\        
    \end{tabular}
    \label{tab:atlas_results_2018}
  \end{center}
\end{table}

It is worth it to mention that the ATLAS papers present a broader analysis, including the decays to $J/\Psi +\gamma$ and $\Psi(2S) +\gamma$.


CMS~\cite{cms_paper} also has a result on $J/\Psi +\gamma$ and $\Psi(2S) +\gamma$ decay channel, of the Higgs and Z boson~\cite{papper_jpsi}. The observed upper limits on the branching fraction for these decays are presented in T\ref{tab:cms_jpsi_results}.

\begin{table}[htp]
  \begin{center}
    
    
    \caption{Observed upper limits, by CMS, on the branching fractions for the Higgs and Z decays. The number are compatible with the ones obtained by ATLAS. The results presented for different polarization scenarios of the $J/\Psi$.}
    \begin{tabular}{ ccccc }
      Channel & Polarization  & $\mathcal{B}F$ at 95\% \CL\\
      \hline
      & Unpolarized & $< 1.4\ten{-6}$  \\
      Z$\rightarrow  J/\Psi +\gamma$ & Transverse & $< 1.5\ten{-6}$  \\
      & Longitudinal & $< 1.2\ten{-6}$  \\
      \hline \hline
      H$\rightarrow  J/\Psi +\gamma$ & Transverse & $< 7.6\ten{-4}$  \\
    \end{tabular}
    
    \label{tab:cms_jpsi_results}
  \end{center}
\end{table}

No result on the Z and Higgs decays to $\Upsilon(nS) +\gamma$ have been published by CMS, yet.

The results presented here are a subset of a broader topic related to the rare decays of Standard Model (SM) boson, involving quarkonia. Sticking only to CMS results, we can cite:

\begin{itemize}
  \item Search for Higgs and Z boson decays to $\mathrm{J}/\Psi$ or $\Upsilon$ pairs in proton-proton collisions at $\sqrt{s} = $ 13 TeV~\cite{Sirunyan:2676242}.
  \item Observation of the $\mathrm{Z} \to \Psi \ell^{+}\ell^{-}$ decay in pp collisions at $\sqrt{s} = $ 13 TeV~\cite{Sirunyan:2623687}. This one specifically, is the first observation a such decay, involving a Z boson.
  \item Search for decays of the 125 GeV Higgs boson into a Z boson and a $\rho$ or $\phi$ meson~\cite{cms_higgs_rho_phi}.
\end{itemize}

In this study, we consider the nominal values of mass for the Higgs, Z and the $Y(\text{1S, 2S, 3S})$, as in~\cite{pdg_2020} are considered.



% detector

\chapter{Experimental Setup}

This chapter describes the experimental setup used in this study, for the sake of brevity, it is provided a brief descriptions of the Large Hadron Collider (LHC), the Compact Muons Solenoid (CMS), and its subdetectors, and the process of high-level physics objects processing and reconstruction.

\section{The Large Hadron Collider}

The Large Hadron Collider (LHC) is the world largest and powerful particle accelerator for protons and heavy-ions ever build. It is located in a complex of other accelerator operated by the European Organization for Nuclear Research (CERN), in the border of between Switzerland and France. The LHC is built in the same 26.7 km extension tunnel with depth varying from 45 m to 170 m below the surface (the LHC plane is tilted 1.4\% for construction reasons), once used by Large Electron–Positron Collider. The CERN complex is a composition of many accelerators, for proton and heavy-ions, used to provide beams of particles for smaller experiments and as a sequence of injectors for the LHC. Figure~\ref{lhc_complex} presents the many components of the LHC complex of accelerators. A detailed description of the LHC can be found at~\cite{Evans:2008zzb, Bruning:782076, Bruning:815187, Benedikt:823808}.

% lhc complex
\begin{figure}[htbp]
    \centering
    \includegraphics[width=\textwidth]{figures_and_tables/experimental_setup/lhc_complex.pdf}
    \caption{The LHC is the last ring (dark grey line) in a complex chain of particle accelerators. The smaller machines are used in a chain to help boost the particles to their final energies and provide beams to a whole set of smaller experiments. Source:~\cite{lhc_complex}.}
    \label{lhc_complex}
\end{figure}

A LHC section is composed of two vacuum pipes, in which the bunch of particles travels in opposite directions. This means that both beams are magnetically coupled by the same super-conducting magnetic system, saving space and allowing the use of the pre-built LEP tunnel. The particle acceleration is made by Resonant Cavities~\cite{Baird:1017689}. Those cavities apply to each beam a set of radio-frequencies (RF) used to transfer energy by means of a 2 MV electric potential per cavity, at a revolution frequency of 400.789 MHz. The acceleration is applied to bunches of particles. The bunch configuration depends of the injection mode (configurable), but a typical $pp$ injection would be composed by 2808 bunches of $1.1 \times 10^{11}$ protons each. Proper timing of the bunches injection and the RF is a key factor for an efficient energy transfer inside the RF cavities. The cavities also are operated in low temperatures of 4.5 K, to ensure superconducting properties and reduce energy losses.

The nominal time spacing between each bunch (bunch crossing - BX) is 25 ns. This defines the clock frequency of the LHC at $f_{LHC} = 400$ MHz. This frequency is propagated to all experiments and used as a reference for timing and synchronization. 

In certain positions, called the interaction points (IP), those two bunches are allowed to cross, possibiliting the particle collisions. The experiments on the LHC are located in those interaction points. ATLAS (A Toroidal LHC ApparatuS)~\cite{atlas_collaboration_2008} and CMS (Compact Muon Solenoid, better explained in the next section), at P1 and P5, respectively, are so called general proposes experiments, with focus on different aspects of a particle interactions in the LHC energy scale, including extensive test of known Standard Model process (in high and low transverse momentum regime), including the Higgs sector and Heavy Flavour Physics (phenomena involving the hadrons composed by $c$ and $b$ quarks), exploration of Beyond Standard Model (BSM) phenomena, as well as an competitive program in heavy-ions collisions. The LHCb (Large Hadron Collider beauty)~\cite{Alves:2008zz} is a experiment devoted, mostly, to precision measurements of CP violation and rare decays of $B$ hadrons. The ALICE (A Large Ion Collider Experiment)~\cite{Aamodt:2008zz} experiment is dedicated to the study of $p$-$Pb$ and $Pb$-$Pb$ collisions and processes such as QCD, strongly interacting matter and the quark-gluon plasma at extreme values of energy density and temperature.

The number of events of a certain kind $i$, per unit of time, is given by Equation~\ref{events_rates_cross_section}.

\begin{equation}
    \frac{dN^{i}}{dt} = \sigma^{i} \mathcal{L},
    \label{events_rates_cross_section}
\end{equation}
where $\sigma^{i}$ is the cross-section for a certain process $i$ and $\mathcal{L}$ is the instantaneous luminosity delivered by the LHC.

 In order to accumulate as much statistics as possible, in the shortest amount of time (for the most efficiently use of the resources available, including person-power), the luminosity is a key factor in the exploration of the collisions. This is dependent of the number of particles per bunch, number of bunches per beam, revolution frequency, form factors of the bunches, crossing angles at the interaction points and correction factors to address relativistic and electromagnetic associated phenomena. For $pp$ collisions, the LHC aims peak luminosities of, for ATLAS and CMS, around $2x10^{34} cm^{-2}s^{-1}$. For future upgrades of the LHC (called HL-LHC~\cite{ApollinariG.:2017ojx}), the peak luminosity might increase 10 times, allowing an accumulated luminosity~\footnote{Accumulated (or integrated) luminosity is defined as $L = \int \mathcal{L}\text{ }dt$.} of 3000 $fb^{-1}$.

 The LHC can collide protons with center-of-mass energy $\sqrt{s}$ up to 14 TeV. Different energy configurations have been used so far, historically increasing the energy. For the operation cycle used in this study (Run2, from 2015 to 2018), the machine was producing collisions at $\sqrt{s} = 13$ TeV. For the next operation cycle (Run3), to start in 2022, it is expected that the LHC might reach the 14 TeV energy values.

\chapter{Experimental Setup}

The central feature of the CMS apparatus is a superconducting solenoid of 6\unit{m} internal diameter, providing a magnetic field of 3.8\unit{T}. Within the solenoid volume are a silicon pixel and strip tracker, a lead tungstate crystal electromagnetic calorimeter (ECAL), and a brass and scintillator hadron calorimeter (HCAL), each composed of a barrel and two endcap sections. Forward calorimeters extend the pseudorapidity coverage provided by the barrel and endcap detectors. Muons are detected in gas-ionization chambers embedded in the steel flux-return yoke outside the solenoid. 

A detailed description of the CMS detector, together with a definition of the coordinate system used and the relevant kinematic variables, can be found in~\cite{Chatrchyan:2008zzk}.  

\todo[inline]{falar do sistema de coordenadas e definir $\eta$}



\section{Tracker}
\todo[inline]{FAZER!}

The silicon tracker measures charged particles within the pseudorapidity range $\abs{\eta} < 2.5$. It consists of 1440 silicon pixel and 15\,148 silicon strip detector modules. For non-isolated particles of $1 < \pt < 10\GeV$ and $\abs{\eta} < 1.4$, the track resolutions are typically 1.5\% in \pt and 25--90 (45--150)\mum in the transverse (longitudinal) impact parameter \cite{TRK-11-001} 


\section{Electromagnetic Calorimeter}

The Electromagnetic Calorimeter (ECAL) is responsible for absorb (and measure) the energy of photons and electrons produced as final state particles of the collisions. The ECAL consists of 75\,848 lead tungstate ($PbWO_4$) crystals, which provide coverage in pseudorapidity $\abs{\eta} < 1.48 $ in a barrel region (EB, $2.2 \times 2.2 $ $cm^2$ and a length of 23 cm) and $1.48 < \abs{\eta} < 3.0$ in two endcap regions (EE, $2.86 \times 2.86$ $cm^2$ front cross section and 22 cm long). Preshower detectors consisting of two planes of silicon sensors interleaved with a total of $3 X_0$ of lead are located in front of each EE detector \cite{Khachatryan:2015hwa}, as shown in Figure~\ref{cms_ecal}. 

% cms ecal
\begin{figure}[htbp]
    \centering
    \includegraphics[width=0.5\textwidth]{figures_and_tables/experimental_setup/cms_ecal.png}
    \caption{Longitudinal section view of the ECAL and its components. Source:~\cite{Chatrchyan:2008zzk}.}
    \label{cms_ecal}
\end{figure}

When a electron (or photon) enters the high density region of the lead tungstate crystals (8.3 $g/cm^3$), it initiates a cascade effect of pair production and photon emission via bremsstrahlung. The intensity of light produce is proportional to the energy of the particle adsorbed. With radiation length~\footnote{Distance an electron or a photon travels
until its energy is reduced by a factor of $1/e$.} of 0.89 cm and a small Molière radius (2.2 cm) the ECAL was built with compact size and its fine granularity. The preshower, located in front of the endcap ECAL (EE), is used to distinguish from high momentum photons and pair of photons coming from $\pi^0$ decays, highly boosted, in such a way that they would be indistinguishable one from the other. Its first layer is composed by the lead tungstate crystal, followed by silicon strip sensor, that allow to measure the shape of the initiated cascade on the first layer and correlate this with the source of the radiation. 

Each ECAL crystal is isolated by a carbon fiber layer and it is connected to two photodetectors with a gain of 50. Their signal is collected by a ADC (Analog to Digital Converter) which catches the charge from the photodetectors and convert it to a digital signal.

In the barrel section of the ECAL, an energy resolution of about 1\% is achieved for unconverted or late-converting photons that have energies in the range of tens of GeV. The remaining barrel photons have a resolution of about 1.3\% up to a pseudorapidity of $\abs{\eta} = 1$, rising to about 2.5\% at $\abs{\eta} = 1.4$. In the endcaps, the resolution of unconverted or late-converting photons is about 2.5\%, while the remaining endcap photons have a resolution between 3 and 4\%~\cite{CMS:EGM-14-001}. When combining information from the entire detector, the jet energy resolution amounts typically to 15\% at 10\GeV, 8\% at 100\GeV, and 4\% at 1\TeV, to be compared to about 40\%, 12\%, and 5\% obtained when the ECAL and Hadronic Calorimeter (HCAL) alone are used. 

Due to its responsability on photon and electrons identification, the ECAL had a very important role on the HIggs observation, specially concerning its relation with the $\gamma\gamma$ and 4-leptons finals states of the discovery.





\section{Hadronic Calorimeter}
\todo[inline]{FAZER!}

In the region $\abs{\eta} < 1.74$, the HCAL cells have widths of 0.087 in pseudorapidity and 0.087 in azimuth ($\phi$). In the $\eta$-$\phi$ plane, and for $\abs{\eta} < 1.48$, the HCAL cells map on to $5 \times 5$ arrays of ECAL crystals to form calorimeter towers projecting radially outwards from close to the nominal interaction point. For $\abs{\eta} > 1.74$, the coverage of the towers increases progressively to a maximum of 0.174 in $\Delta \eta$ and $\Delta \phi$. Within each tower, the energy deposits in ECAL and HCAL cells are summed to define the calorimeter tower energies, which are subsequently used to provide the energies and directions of hadronic jets.

Jets are reconstructed offline from the energy deposits in the calorimeter towers, clustered using the anti-\kt algorithm~\cite{Cacciari:2008gp, Cacciari:2011ma} with a distance parameter of 0.4. In this process, the contribution from each calorimeter tower is assigned a momentum, the absolute value and the direction of which are given by the energy measured in the tower, and the coordinates of the tower. The raw jet energy is obtained from the sum of the tower energies, and the raw jet momentum by the vectorial sum of the tower momenta, which results in a nonzero jet mass. The raw jet energies are then corrected to establish a relative uniform response of the calorimeter in $\eta$ and a calibrated absolute response in transverse momentum \pt.



\section{Muon System}

Muons at CMS~\cite{muon_tdr} are measured in the pseudorapidity range $\abs{\eta} < 2.4$, with detection planes made using three technologies: drift tubes, cathode strip chambers, and resistive plate chambers, as presented in Figure~\ref{cms_muon}. The single muon trigger efficiency exceeds 90\% over the full $\eta$ range, and the efficiency to reconstruct and identify muons is greater than 96\%. Matching muons to tracks measured in the silicon tracker results in a relative transverse momentum resolution, for muons with \pt up to 100\GeV, of 1\% in the barrel and 3\% in the endcaps. The \pt resolution in the barrel is better than 7\% for muons with \pt up to 1\TeV~\cite{Sirunyan:2018}. 

% cms muon
\begin{figure}[htbp]
    \centering
    \includegraphics[width=\textwidth]{figures_and_tables/experimental_setup/cms_muon.pdf}
    \caption{Longitudinal section view of the ECAL and its components. Source:~\cite{Chatrchyan:2013sba}.}
    \label{cms_muon}
\end{figure}

The muon detection system has around 1 million channels. For Run3, the muon system is being expanded and upgraded, by the inclusion of new chamber with the Gas Electron Multiplier (GEM)~\cite{Sauli:2262884} technology.

\subsection{Drift Tubes}

The Drift Tubes (DT)~\cite{Teyssier:2015xjj} is a gaseous detector (85\% Ar and 15\% CO2) installed in the central region of CMS (Barrel), covering the region of $|\eta < 1.2|$. The barrel is divided in 5 wheels, along $z$, W+2, W+1, W0, W-1 and W-2. Each will is composed by four concentric stations along $r$, MB 1 to MB4, and each station is divided in 12 sectors along $\phi$, S01 to S12. In total, there are 205 DT chambers. Each tube has 50 $\mu$m tick (diameter) gold-plated stainless steel wire, as well as, kept at positive voltage, and aluminum electrodes. The signal is read on the golden wire only.

The tubes are arranged in layers and occupy the whole length of the chamber. The tubes are arranged in coaxial layers. Each set of three layers, forms a Super-Layer (SL).The first and the last SL are aligned in the, so called, $r-\phi$ direction, while the middle one, in the $r-z$ direction, transversal to the previous one. This arrangement give the DTs, the possibility to measure the passage of a muon in $\eta$ and $\phi$ direction, with a resolution of 100 $\mu$m.



\subsection{CSC}

The Cathode Strip Chamber (CSC) is also a gaseous detector (50\% CO2, 40\% Ar, and 10\% CF4) of the Muon System which covers the endcap region, up to $|\eta| < 2.4$ composed by wires perpendicular to $\eta$ (radial measurement) and strips along $\eta$, the former operating at 3.9 to 3.6 kV. With 8.4 to 16 mm strip width and a wire-distance of 2.5 to 3.16 mm depending on their location, they provide a 75 to 150 $\mu$m resolution.

They are installed in four layers (or disk) on each side of CMS, with each disk divided in up to three rings.


\subsection{RPC}

The Resistive Plate Chambers (RPC) is the only muon detection technology present in both barrel and endcap. It has very good timing resolution and it is used mostly for triggering.

Due to the particularities of the study, especially the contributions given to the RPC project of CMS, Chapter~\ref{chapter_rpc} is devoted exclusively to this subdetector.



\section{Trigger and Data Acquisition}
\todo[inline]{FAZER!}

A two-tiered trigger system~\cite{Khachatryan:2016bia}. The first level (L1), composed of custom hardware processors, uses information from the calorimeters and muon detectors to select events at a rate of around 100\unit{kHz} within a time interval of less than 4\mus. The second level, known as the high-level trigger (HLT), consists of a farm of processors running a version of the full event reconstruction software optimized for fast processing, and reduces the event rate to around 1\unit{kHz} before data storage.

 

\section{Particle Flow Algorithim}
\todo[inline]{FAZER!}

% rpc
%%%%%%%%%%%%%%%%%%%%%%%%%%%%%%%%%%%%%
\chapter{CMS Resistive Plate Chambers - RPC}

% \todo{DESCREVER **MELHOR** AS ATIVIDADES PARA A RPC ATIVIDADES}

In the course of this PhD study, there were a lot of opportunities to work for the RPC project, in the context of the CMS Collaboration. The main activities consists of shifts for the RPC operation and data certification, upgrade and maintenance of the online software, R\&D activities for the RPC upgrade and detector maintenance during the LHC Long Shutdown 2 (2019 to 2020).

In this chapter, it is presented a summary of the Resistive Plate Chamber technology and the contributions to the RPC project at CMS.

\section{Resistive Plate Chambers}

The seminal paper on the Resistive Plate Chamber (RPC) technology was presented by R. Santonico and R. Cardarelli, in which they described a "dc operated particle detector (...) whose constituent elements are two parallel electrode Bakelite plates between"~\cite{rpc_seminal}. The key idea behind the RPC, with respect to other similar gaseous detectors, is the use of two resistive plates as anode an cathode, which makes possible to have a small localized region of dead time, achieving very good time resolution. 

The working principle for RPCs relies on the idea that a ionizing particle crossing the detector, tend to interact with the gas gap between the two plates and form a ionizing cascade process, in which the the produced charged particle are driven by the strong uniform electrical field produced by the two plates.

The gas mixture is a key component of a RPC. Even though the first RPCs were produced with a mixture of argon and butane, nowadays RPCs use a mixture of gases that would enhance an ionization caused by the incident particle and quench secondary (background) effects.

Another feature of the RPCs is its construction simplicity and low cost. This allow the use RPC to cover larger at a reasonable cost. 

A extensive review of the RPC technology and its application can be found in~\cite{livro_rpc}.

\todo{DESCREVER A TECNOLOGIA DAS RPCS}

\todo{DESCEREVER OS PRINCÍPOS DE OPERACAO - TDR}

\section{CMS Resistive Plate Chambers}

At CMS, the Resistive Plate Chamber are installed in both the barrel and endcap region, forming a redundant system with the DT (barrel) and CSC (endcap). As described in the CMS Muon Technical Design Report (Muon-TDR)~\cite{muon_tdr}, the RPC are composed of 423 endcap chambers and XXXXXX \todo{QUAL O NUMERO DE BARREL CHAMBERS????} barrel chambers.

Each chamber consists of two gas gaps (double gap), 2 mm tick each, made of Bakelite (phenolic resin) with bulk resistivity of $10^{10}$ - $10^{11}$ $\Omega m$. The choice of the bulk resistivity of the electrode has high impact on the rate capability of the detector.

Each gap has its external surface is coated with a thin layer of graphite paint, which acts as conductive material, distributing the applied high-voltage (HV). On top of the graphite a PET film is used for isolation. A sheet of copper strips is sandwiched between the gaps. Everything is wrapped in aluminum case.

The double gap configuration increases the efficiency of the chamber, since the signal is picked up from the OR combination of the two gaps. A chamber with only one gap working, looses around 15\% of efficiency, even though, this can be recovered by increasing the HV applied during operation mode (working point - WP).

A characteristic that differentiate the CMS RPC  from from previous RPC application in HEP is the operation mode. A RPC at CMS, operates in avalanche mode, while previous experiments used the streamer mode. Both modes are related to the applied HV, in commitment with the strength of the generated signal, and are capable of generate a well localized signal, which can be picked up by the readout electronics, but the avalanche mode offer a higher rate capability around 1 kHZ/$cm^2$, while the streamer mode goes up to 100 HZ/$cm^2$. The high rate capability is a key factor in order to cope with requirements of the LHC luminosity, specially in the high background regions.

Besides the rate capability, the key factors that driven the CMS RPC design were:  high efficiency ($> 95\%$), low cluster size ($> 2$) for better spatial resolution (this reflects in the momentum resolution) and good timing in order to do the readout of the signal within the 25 ns of a LHC bunch cross (BX) and provide it to the CMS trigger system. These requirements have implications in the choice of material, dimensions, electronics and gas mixture.

In the barrel, along the radial direction, there are 4 muon layers (called stations), MB1 to MB4. MB1 and MB2 is composed of a DT chamber sandwiched between two RPC chambers (RB1 and RB2) with rectangular shape. The stations MB3 and MB4 have only one RPC (RB3 and RB4) are composed by two RPC chambers (named - and + chambers with the increase of $\phi$) attached to one DT chamber, except in sector 9 and 11, where there is only one RPC. RE4, sector 10 is a special case, since it is composed of four chambers (--, -, + and ++). These stations are replicated along the z direction in five different wheels of the CMS (W-2, W-1, W0, W+1 and W+2) and in twelve azimuthally distributed sectors (S1 to S12). Figure~\ref{fig:barrel_rphi_rz} show the different barrel stations and wheel.

\todo{DAR AS DIMENSOES - BARREL}


\begin{figure}[h]
\begin{center}
\includegraphics[width=1.0\textwidth,keepaspectratio]{figures/rpc/barrel_rphi_rz.png}
\end{center}
\caption{R-$\phi$ (left) and R-Z (right) projections of the barrel Muon System.}\label{fig:barrel_rphi_rz}
\end{figure}


In the endcap, the RPC chambers have a trapezoidal shape and are distributed in four disks (or stations) each side (RE$\pm4$, RE$\pm3$, RE$\pm2$, RE$\pm1$), each one with 72 chambers. CMS split up its disks in 3 rings, along the radial direction, and 36 sector in the azimuthal angle. RPCs are present in the two outer rings (R2 and R3), in all 36 sectors. The RE$\pm4$ are special cases, since these chambers were installed only in 2014, a design choice was made the mechanically attached R2 and R3 chambers, each sector, in what is called, a super-module. Figure~\ref{fig:endcap_rz} show the different endcap disks.

\todo{DAR AS DIMENSOES - ENDCAP }


\begin{figure}[h]
\begin{center}
\includegraphics[width=1.0\textwidth,keepaspectratio]{figures/rpc/endcap_rz.png}
\end{center}
\caption{R-Z projections of the endcap Muon System (positive Z side). This is the same configuration for the 36 $\phi$ sectors.}\label{fig:endcap_rz}
\end{figure}

The length of the strips is chosen, for both barrel and endcap, in such a way to control the area of each strip, in order to reduce the fake muons, due to random coincidence. This has to do with the time-of-flight and signal propagation along the strip. In the barrel, each chamber readout is divided in two regions (rolls), called forward and backward (along increasing $|\eta|$)~\footnote{Some chamber are divided in three rolls, forward, middle and backward, for trigger propose.}. In the endcap, the strips are divided in 3 regions, called partitions A, B and C (from inside the detector to outside).


The gas mixture used in the CMS RPCs is composed by C2H2F4 (Freon R-134a, tetrafluoroethane), C4H10 (isobutane), SF6 (sulphur hexafluoride) (95.2 : 4.5 : 0.3 ratio) and with controlled humidity of 40\% at 20-22 $^{\circ}$C. The Freon is used to enhance the ionization and charge multiplication that characterizes the avalanche, while the isobutane is introduced for quenching proposes, in order to reduce the secondary ionizations that could lead to formation of streamers and the SF6 is used to reduce the electron background. The choice of Freon over other gases, i.e. argon-based and helium-based, was motivated by previous studies~\cite{gas_mixture_BERNARDINI1995428,gas_mixture_Gorini}.


Since its R\&D, the RPC have shown good performance over aging. This is even historical over previous RPC experiments \todo{ADICIONAR REFERÊNCIAS 5.14, 5.15 E 5.16 DO MUON TDR}. Even the most recent studies of aging, taking into account future LHC conditions (High-Luminosity LHC - HL-LHC) plus a safety margin of 3 times the expected background (600 HZ/$cm^2$) have shown good aging hardness~\cite{andrea_rpc_2018}.

\subsection{Performance}

\todo{PERFORMANCE NO RUN2}

\section{Contribution to the CMS RPC project}

During the curse of this study, a head collaboration of our research group and the CMS RPC project was established. Many contributions were given to the project as part of the graduation as a experimental particle physicist, with focus on getting acquaintance with a subsystem technology and give a meaningful collaboration to the detector operation.  Those are considered by the community important steps on the student graduation.

Bellow it is described the contributions given to the CMS RPC project.

\subsection{RPC Operation - Shifts and Data Certification}

The first activities done for the CMS RPC project were shifts for data certification of data taken. This certification is done by specialized people for different CMS subsystems and physics objects groups~\footnote{Groups of reconstruction and performance experts for different high-level reconstructed objects from CMS, i.e. muons, taus, jets/MET, electrons/photons}. 

This certification is done in order to ensure the quality of the date recorded based on the well functionality of each system during the data taking and the reconstruction of the physics objects in the expected matter. A certain collection of data (run) is said certificate when all subsystems and object experts agrees on this.

Figure~\ref{lumi} shows, as an example of the luminosity delivered by the LHC, recorded one by CMS and the certified (validated), from the 22nd of April, 2016 to the 27th of October, 2016. Only certified data is available for physics analysis.

Shifts are a continuous weekly activity (specially during the data taking period), performed in a weekly basis, in order to ensure the availability of certified data, as soon as possible.
 
\begin{figure}%
    \centering
    \subfloat[]{{\includegraphics[width=0.55\textwidth,keepaspectratio]{figures/rpc/lumi_plot.png} }}%
    \qquad
    \c[]{{\includegraphics[width=0.35\textwidth,keepaspectratio]{figures/rpc/lumi_table.png} }}%
    \caption{(a) Luminosity delivered by the LHC, recorded one by CMS and the certified (validated), from the 22nd of April, 2016 to the 27th of October, 2016. (b) Efficiency of the certification process for each subsystem, from the 22nd of April, 2016 to the 27th of October, 2016. Total CMS efficiency is above 90\%. \cite{certification}}%
    \label{lumi}%
\end{figure}



\section{RPC Online Software}

On what concerns the Online Software (OS) of the CMS RPC system, the main contribution given was the upgrade of the Trigger Supervisor libraries.

The Trigger Supervisor is a web-based software, which run over the xDAQ backend and provides, through a mudules organized in a tree system, called cells, a standard inferface for the operation and monitoring of different system at CMS. In principle only systems which contribute directly to the L1 trigger should have a Trigger Supervisor implementation. This was the case for the  RPC during the Run1. Since Run2, RPC contributes to the trigger indirectly, by providing data to the muon processors (CPPF, OMTF and TwinMux). The Trigger Supervisor implementation is a legacy from that period.

Each subsystem is responsible for its own implementation of the Trigger Supervisor, based on the functionalities that it wants to have (requirements). The xDAQ~\cite{xdaq} is a middleware, developed by CERN and widely used at CMS, as a tool for control and monitoring of data acquisition system in a distributed environment. It is capable of providing a software layer for direct access of hardware functionalities and monitoring.

The upgrade made (figure~\ref{ts_upgrade}), consists in upgrade the higher level of the RPC online software. In summary, up to 2017, the online software, was using Scientific Linux 6 as operational system, which executed xDAQ 12, in turn, servers as backend for Trigger Supervisor 2. A upgrade of the operational system to Centos 7, demanded the upgrade to xDAQ 14. On top of that, Trigger Supervisor 2 would not work and had to be updated to Trigger Supervisor 5 in order to be functional in 2018.

\begin{figure}[h]
\begin{center}
\includegraphics[width=0.8\textwidth,keepaspectratio]{figures/rpc/ts_upgrade.png}
\end{center}
\caption{Upgrade of the RPC online software.}\label{ts_upgrade}
\end{figure}

Between versions 2 and 5 of Trigger Supervisor, part of the source-code had to be reworked, keep the majority of the code structures. Most of the changes were made in the front-end of the system. The standard JavaScript library Dojo~\cite{dojo}, used in version2, was deprecated in favor of Google's Polymer\cite{polymer}. The main reason for this change was to isolate C++ code from HTML, which was impossible with Dojo. This implied to rewrite all the screen of the RPC Trigger Supervisor implementation, as in figure~\ref{ts_view}.

The upgrade  was done in time to ensure the control and operation of the RPC for 2018 data taking.

\begin{figure}[h]
\begin{center}
\includegraphics[width=0.95\textwidth,keepaspectratio]{figures/rpc/ts_view.png}
\end{center}
\caption{Example of the updated screens, using Trigger Supervisor 5.}\label{ts_view}
\end{figure}


\subsection{iRPC R\&D}

For the next 4 year of CMS activities it is foreseen  the upgrade of the the Muon Systems~\cite{muon_tdr}. These upgrades are planed in order to extend the pseudorapidity coverage ($\eta$) and to guarantee the operation conditions of the present system in the HL-LHC (High Luminosity LHC) era. The RPC (Resistive Plate Chambers)~\cite{muon_tdr} subsystem, it will have maintenance of the present chambers and installation of new chambers in the region of $|\eta| < 1,8$ para $|\eta| < 2,4$~\cite{pedrazamorales2018rpc}. These new chambers (iRPC) will be added in the most internal part of the muon spectrometer, RE3/1 e RE4/1, as in Figure~\ref{muons_eta}.

\begin{figure}[h]
\begin{center}
\includegraphics[width=1.0\textwidth,keepaspectratio]{figures/rpc/muon_eta.png}
\end{center}
\caption{$\eta$ projection of the Muon System subdetectors. In purple, is labeled the iRPCS to be installed during the CMS upgrade.}\label{muons_eta}
\end{figure}

Even thought this region is covered by the CSC detectors CSC (Cathode Strip Chambers), there are some loss of efficiency due the the system geometry. The installation of additional chambers will mitigate this problem and potentially increase the global efficiency of the muon system. The new chamber, called iRPC (\textit{improved RPC}), will be different them the present one. For a luminosity of $5 \times 10^{34} cm^{-2} s^{-1}$  the neutrons, photons, electrons and positrons background in the high $|\eta|$ region is expected to by be around 700 Hz/$cm^{2}$ (for the chambers in RE3-4/1). Applying a safety factor of 3, the new chambers should support up to 2 Hz/$cm^{2}$ of gamma radiation and still keep more than 95\% of efficiency. Studies indicate, so far, that the use of High Pressure Laminates (HPL) for the double gap chambers is the most suitable choice. In order to reduce the aging and increase the rate capability, the electrodes and the gap size should be reduced in comparison with the present system.

One of the challenges for the R\&D of the iRPC chambers is measuring the their performance in a high radiation environment, as the one for HL-LHC. For this, the CMS RPC project uses the Gamma Irradiation Facility (GIF++)~\cite{gifpp}, at CERN. The GIF++ is located at the H4 beam line in EHN1 providing high energy charged particle beams (mainly muon beam with momentum up to 100 GeV/c), combined with a 14 TBq 137-Cesium source. In the GIFF++ it is possible to achieve the HL-LHC total dose in a much reasonable amount of time. With the shutdown of LHC, the muon beam source is also off and will stay like this for 3 years. This means that the only muon sources for studies in GIF++ are cosmic muons. 

In order to create a trigger system for iRPC R\&D, the usual procedure is to use scintillators, on the top and on the bottom of the chamber. This is effective, but in a high gamma radiation environment, scintillators can be very sensitive which could lead to an undesirable amount of fake triggers which can degrade the measurements. Also, if one wants to covers a large area with scintillators, this can be expensive and they will not provide any means of tracking to measuring not only the global, but also the local chamber performance.

To provide a solution, the CMS RPC got in agreement with the LHCb~\cite{lhcb} Muon Project to use their Multiwire Proportional Chambers (MWPC)~\cite{mwpc}, which were removed from LHCb, to be replaced by new chambers, and use them as trigger and/or tracking system at GIF++. This chambers, by design, have relatively low gamma sensitivity, already have some 1-dimensional resolution ($O$(cm)) and, the biggest advantage, with respect to any other gaseous particle detector option: LHCb has hundreds of vacant chambers. Any other detector would have to be build.

Not going in details of the MWPC technology nor the LHCb chamber construction~\cite{lhcb_mwpc}, these chambers have a total active area of 968 $\times$ 200 $mm^2$ divided 2 layers (top and bottom) of 24 wire pads (40 $\times$ 200 $mm^2$) composed of around 25 wires/channel, grouped by construction. Each chamber is equipped with 3 FEBs (Front-End Boards) with 16 pads each.

A channel is a logical combination of a top layer (pads) and a bottom layer readout. These readouts can be combined in AND or OR logics. One can have 8 channels per FEB, each channel being a logical combination of top and bottom pads. In this mode they are called AND2 and OR2. It is also possible to have the FEB configured in a 2 channels mode, each one corresponding to one sixth of the total readout pads. In this mode, all the pads can be combined in OR (called OR8) or they can be AND'ed in top and bottom pads and then group in OR (called OR4AND2). Figures~\ref{8channels} and~\ref{2channels} presents a logical diagram for each readout mode.

\begin{figure}[!htbp]
\begin{center}
\includegraphics[width=0.27\textwidth,keepaspectratio]{figures/rpc/mwpc/or2.png}\hspace*{1.cm}
\includegraphics[width=0.27\textwidth,keepaspectratio]{figures/rpc/mwpc/and2.png}\hspace*{1.cm}
\end{center}\vspace*{-.5cm}
\caption{FEB configured 8 channels modes. Group should be understood as wire pad. Left: Logical diagram for OR2. Right: Logical diagram for AND2.}
\label{8channels}
\end{figure}


\begin{figure}[!htbp]
\begin{center}
\includegraphics[width=0.40\textwidth,keepaspectratio]{figures/rpc/mwpc/or8.png}\hspace*{1.cm}
\includegraphics[width=0.40\textwidth,keepaspectratio]{figures/rpc/mwpc/or4and2.png}\hspace*{1.cm}
\end{center}\vspace*{-.5cm}
\caption{FEB configured 2 channels modes. Left: Logical diagram for OR8. Group should be understood as wire pad. Right: Logical diagram for OR4AND2.}
\label{2channels}
\end{figure}

The nominal gas mixture for these chambers is Ar/CO2/CF4 (40:55:5). For a matter of simplicity, it was used an already available similar gas line in the same building, used by CMS CSC (Cathode Strip Chamber)~\cite{muon_tdr}, which has a similar composition (40:50:10). Optimal conditions are obtained with 2 to 4 liters/hour of gas flux and 2.65 kV of applied voltage.

Figure~\ref{setup} shows the setup that was prepared for commissioning of this chambers. It was mounted two chambers on top of another (chambers A and B) above an RPC R\&D chamber and two other chambers on the bottom (chambers C and D). These four MWPC will be used as telescope for the RPC chamber. All the services were mounted in rack, as in Figure~\ref{setup}. This includes power supply (low voltage and high voltage), distribution panel, VME crate and boards for FEB control, computer for control (high voltage, and FEB control) and NIM crate and boards for LVDS to NIM signal conversion, logics and counting.

\begin{figure}[!htbp]
\begin{center}
\includegraphics[width=0.57\textwidth,keepaspectratio]{figures/rpc/mwpc/setup.png}\hspace*{1.cm}
\includegraphics[width=0.23\textwidth,keepaspectratio]{figures/rpc/mwpc/rack.png}\hspace*{1.cm}
\end{center}\vspace*{-.5cm}
\caption{Left: Setup mounted for commissioning. Two MWPC chamber on the top (chambers A and B) and two (chambers C and D) on the bottom with a RPC R\&D in the middle. Right: Rack with all the services for the operation of these chambers.}
\label{setup}
\end{figure}

Due to the short amount of time available for the commissioning, only two measurements measurements were made with these chambers. They were meant to be a proof of concept for future activities.

The first measurement was to measure the coincidence rate of two chambers as a function of the distance between the two top planes (Figure~\ref{coincidence}). This measurements were done with nominal working point, with one FEB configured in  2 channels mode with 7 pC threshold, in (160 mm x 160 mm) area per chamber. One can observe that, if we go for a telescope trigger in the order of 1 meter of separation between the chamber, the logical combination chosen has negligible effect in the coincidence rate. Also, the fits can be used to estimate the rate in a configuration with chamber on the roof and under the floor. This could be the case of a universal trigger, to be mounted in GIF++ with these chamber.

\begin{figure}[h]
\begin{center}
\includegraphics[width=0.7\textwidth,keepaspectratio]{figures/rpc/mwpc/coincidence.png}
\end{center}
\caption{Coincidence rate of two chambers with respect to an arbitrary distance between the two top planes. Measured in 10 minutes, for 2 logical combinations (OR8 and OR4AND2). Applied high voltage: 2.65 kV. Threshold: 7 pC. Active area: readout of 160 mm x 160 mm per chamber.}
\label{coincidence}
\end{figure}

The second measurement consist on evaluate the impact of $\gamma$ background by placing a small Cs-137 source on top of the chamber A (Figure~\ref{gamma}). For this measurement, the distance between top planes of each pair of chamber (A to B and C to D) is 65 mm, while the distance between the top planes of A and C is 570 mm. It is clear the the $\gamma$ source has an impact on chamber A rate, but this is negligible when we take into account the coincidence between two chambers.

\begin{figure}[h]
\begin{center}
\includegraphics[width=0.7\textwidth,keepaspectratio]{figures/rpc/mwpc/gamma.png}
\end{center}
\caption{Individual rates (chambers A, B, C and D) and coincidence rates for two chambers (A AND B, C AND D), for without $\gamma$ source (blue), a shielded $\gamma$ source (orange) and an unshielded $\gamma$ source (green). Source sitting on top of chamber A. Applied high voltage: 2.65 kV. Threshold: 7 pC. Active area: readout of 160 mm x 320 mm per chamber. Logical combination: AND2}
\label{gamma}
\end{figure}

This two measurements were enough to validate this chambers as possible trigger pro RPC R\&D with cosmic muons in the laboratory and at GIF++. The next steps would be use this MWPC chamber to implement a tracking system from triggering. This would demand some developments, since, due to bandwidth restrictions, the signal from each FEB would have to go to a programmable fast electronics, i.e. a FPGA, which would reconstruct muon tracks and provide the trigger to the DAQ system. This can be done by placing the two pair of chambers (AB and CD) in orthogonal configuration and read the signal in a CAEN V2495 board~\cite{caen_fpga}. 


\clearpage

\subsection{LS2 and the RPC Standard Maintenance}

In December 2018, the LS2 (Long Shutdown 2) was started. This a period in which the LHC and its detectors (CMS included) stop their operation for maintenance and upgrade. The LS2 will go up to 2021, when LHC and CMS restart the data taking with the Run3. 

During the LS2 it is being installed services for the new chambers (gas pipes, low voltage (LV), cables, signal and control optical fibers, high voltage (HV) cable and support equipment, and HV/LV power supplies), as well as continuity to the to the RPC R\&D studies, besides the reparation of broken elements of the present system, i.e. chamber in the barrel region which present gas leak problems, maintenance of the LV and HV connectivity and power system, maintenance of the control system of problematic chambers (Front-Ends boards, cabling and Distribution Boards) and the dismount and reinstallation of four stations in the endcap (RE4) on both sides of CMS~\cite{re4_dismount}.

What concerns the standard maintenance of the present RPC system, the main LS2 activities in which the student was involved, can be divided in three main tasks: (a) HV maintenance, (b) LV and control maintenance and (c) detector commissioning. 

\subsubsection{HV maintenance}

A key factor of and RPC performance is the applied high voltage (HV). The CMS RPC achieve their optimal performance with, around, 9.5 kV applied in each gap. This voltage is in the range of the dielectric breakdown of many gases, which could lead to potential current leakages, if some part of the system is damaged, poorly operated or badly installed. If the currents are high enough this can make impossible the operation of the chamber. In cases like this, during the operation period (data taking), the problematic HV channel is identified and turned off (each chamber has two channels, one for each lawyer of gaps). Chambers in this situation are said to be operating in single gap mode (SG).

The goal for the HV maintenance is to, now that the CMS is off and the chambers are accessible, identify which part of the HV supply system is causing the current leak and fix it the best way possible. Usually the problem is beyond the power supply, very often connectors or the gap itself are damaged.

The CMS RPCs uses two kind of HV connectors, monopolar and tripolar connector. The monopolar are used to connect the chamber to the power supply. If mounted properly, rarely they present problems. The connection to the chamber is made by tripolar connectors, in which the ground and the HV for both gaps arrives to the chamber in a single connector, for simplicity and to save space in the patch panel. Unfortunately these connectors are relatively fragile, and they could be a potential source of leak, specially if they were poorly mounted, badly operated or with aging itself. Also, since this was a connector made exclusively for the CMS RPC system, some design choices had to be improved after the installation of other chamber. Those installed with old batches of tripolar connectors are sensitive ones. The reparation of this connectors consists in isolate the connector from the chamber and power it up to 15 kV (maximum voltage allowed by the system). If the tested connector is broken one will observe a very fast increase in the current of the HV channel. The only solution to this kind of problem is to replace the connector.

On the other hand, if the connector is powered isolated and pass the test, the problem beyond the connector (assuming that the power system have already been tested), i.e. inside the chamber. When a chamber is in SG mode it means that a full layer is off, but not necessarily all the gaps in that layer are bad (a RPC can have up to three per layer). In this situation, the procedure consists in cutting the cables that comes from the gaps to the chamber side connector one by one and identify which gap of the problematic layer is the broken by powering it. Once identified, this gap should isolated and the other ones reconnected. The broken gap is unrecoverable, since it is inside the chamber, but 5\% to 10 \% of efficiency can be retaken, without changing the applied HV and increasing the longevity of the chamber.

Another contribution to the HV maintenance was the proposal of a procedure to replace the problematic tripolar connector by a monopolar (also called jupiter) connector, which are known for being much more stable and reliable. The figure~\ref{jupiterized} (left) show the designed adapter for the chamber patch panel which would made this change possible. Figure~\ref{jupiterized} (right) shows a tryout of a chamber in which this procedure was tested. The proposal was presented to the RPC community and approved to be used from now on. Technical drawings and instructions were provided.

\begin{figure}[!htbp]
\begin{center}
\includegraphics[width=0.45\textwidth,keepaspectratio]{figures/rpc/endcap_hv_pp.pdf}\hspace*{1.cm}
\includegraphics[width=0.45\textwidth,keepaspectratio]{figures/rpc/endcap_hv_pp_tryout.jpg}\hspace*{1.cm}
\end{center}\vspace*{-.5cm}
\caption{Left: Proposed adapter the chamber patch panel which make it possible to replace a tripolar by a jupiter HV connector. Right: Try out of the proposed HV connector replacement.}
\label{jupiterized}
\end{figure}

\subsubsection{LV and control maintenance}

The low voltage (LV) and control maintenance consists in make sure that the Front-End Boards (FEBs) are powered and configurable, which means that the LV power system is working from supply board to the cable, that the signal cables are in good state and properly connected to the chamber and to the link boards and that the on-detector electronics (FEBs and Distribution Boards - DBs) are working fine.

Usually, this system is very reliable. The weak point, in most of the cases, is the detector electronics. When a FEB~\cite{rpc_feb} (as in Figure~\ref{rpc_feb}) is problematic it can present regions of very high noise or no signal at all (silent), which can not be recovered by the threshold control. In cases like this, when the FEB is accessible, it can be replaced in order to recover efficiency in the problematic chamber. This procedure is done by extracting the chamber from inside the detector (only for barrel chamber) and opening its cover to have access to the problematic component. Removed boards are send back to production labs for refurbishment.

\begin{figure}[h]
\begin{center}
\includegraphics[width=0.7\textwidth,keepaspectratio]{figures/rpc/rpc_feb.jpg}
\end{center}
\caption{RPC Front-end board (FEB) used in the barrel chambers.}\label{rpc_feb}
\end{figure}

The most usual problem is a chamber in which the threshold control was lost. For those chamber, most probably, the problem is in the distribution board of the chamber, which is a piece of hardware responsible for distribute the LV power to the FEBs (3 to 6 per chamber) and send the threshold control signal to the FEBs via I2C line. If a chamber has no threshold control, it means that the RPC operation has no control over the signal selection, which can potentially induce performance issues. 

For the barrel this maintenance happens concomitantly with the gas leak reparations on the barrel chamber, since both demands the chamber extraction, which is a complex procedure in terms of operation and demands specialized equipment and manpower. For technical reasons, the gas leak extractions have precedence over LV ones.



\subsubsection{Detector commissioning}
 
All the LS2 activities demands uncabling of the chamber to be repaired and possibly some neighbor chambers. Also, it can involve the replacement of components of the chamber. To avoid damage to the system a compromising procedure is needed after all this activities. Given the responsibilities of the commissioning it was necessary to: (a) make sure that the the RPC system keep tracks of all the interventions, (b) maintain all the algorithms used in the commissioning procedure, (c) together with the RPC Coordination, define a pool of people and a schedule to the commissioning of the system and (d) follow-up, with other CMS RPC experts, the availability of materials and resources for the commissioning operations.

Besides the organizational tasks, the commissioning demanded to establish procedures to ensure the connectivity and functionality of HV and LV connections. For the HV, it is needed to make sure that the chambers are properly connected, without miscabling~\footnote{Mixed cable connections.} and that the currents at stand-by HV  and working point HV are compatible with the ones in the end of last data-taking (end of 2018). This activity will start in November/2019, when the CMS RPC Standard Gas Mixture will be available again.
 
For the LV point of view, the LV power cable and signal cables should also be properly connected, and presenting a noise profile compatible with last data-taking. One key point for this task is to make sure that that there are no miscabling of signal cable. One RPC chamber can have from 6 to 18 signal cable, which are connected very close one to another. There is a good chance that a chamber, after reparation, have its signal cables mixed. In order to diagnose situations like this, it was validated a algorithm present in the RPC Online Software, but never used since LS1, which, by changing the threshold of each component of the RPC system, from very high to very low values (component by component), can spot miscabled chambers. Since the control line is independent of the signal line, a misclabed will present a different noise from what is expected.

Besides the validation of this algorithm, it was also implemented a web system (Figure~\ref{comm}), developed in Flask~\cite{flask} wich automatize the execution of the algorithm, making transparent to the shifter (or the one performing the commissioning) the procedure to get miscabling report.

\begin{figure}[h]
\begin{center}
\includegraphics[width=0.7\textwidth,keepaspectratio]{figures/rpc/comm.png}
\end{center}
\caption{RPC FEB Commissioning Analyzer.}\label{comm}
\end{figure}

The LV commissioning is ongoing, since it happens, as much as possible, right after the chamber reparation.

\clearpage

% analysis
\chapter{Physics Analysis}

The analysis here presented corresponds to the search for rare decays of $H \rightarrow \Upsilon + \gamma$, where the $\Upsilon$ might appear in the states $1S$, $2S$ or $3S$, and shall decay to a pair of muons (from here on, called dimuon system) and the $\gamma$ will be identified as a offline reconstructed photon. The decay to the dimuon channel offers a very efficient triggering for this process, characteristic of CMS. The analogous process of the $Z$ boson decays to the same channel is also studied, as a benchmark for the Higgs decay.

The main process contributing to the accessible phase space of these decays are described in Figure~\ref{analysis_process_diagram}, in which the different process are represented in a diagram for the reconstructed invariant masses of the muon-muon-photon system ($\mu\mu\gamma$ - horizontal axis) and the muon-muon system ($\mu\mu$ - vertical axis). The vicinity of the $H/Z$ mass and $\Upsilon$ mass regions are represented in the midpoint for each axis. The backgrounds can be divided in \textbf{Resonant} and \textbf{Non-Resonant} backgrounds. The Non-Resonant might come from two sources, a Full Combinatorial background is composed by the combination of two non-correlated muons with a photon in the final state of the event. This is expected to be spread all over the phase space and in the diagram, it is represented by the color blue. The $\Upsilon+ \gamma$ Combinatorial background is a combination of two correlated muons (e.g.: the decay of a $\Upsilon$ to a dimuon muon system) combined with a photon from a secondary process (e.g.: Multiple Particle Interaction - MPI, pile-up, a jet mis-identified as a photon). This should be concentrated in the region around the $\Upsilon(1S, 2S, 3S)$ and it is represented by the gray region.

% analysis process 
\begin{figure}[htbp]
    \centering
    \includegraphics[width=0.6\textwidth]{figures_and_tables/analysis_process.pdf}
    \caption{A diagram for the reconstructed invariant mass of the $\mu\mu\gamma$ final state. The blue and gray regions represent the Full Combinatorial and $\Upsilon + \gamma$ Combinatorial contributions, respectively, while the yellow and red regions represent the Resonant background and the signal region.}
    \label{analysis_process_diagram}
\end{figure}

The Resonant background is composed by the processes where the boson (Higgs or Z) decays to a $\mu\mu\gamma$ final state without going trough the the intermediate meson state. For the Z decays, this background is modeled based on a Drell-Yan to dimuon decays, with a final state radiated (FSR) photon ($Z \rightarrow \mu\mu\gamma_{FSR}$), while for the Higgs decay, a  Higgs Dalitz decay ($H \rightarrow \mu\mu\gamma$) is used. The Resonant background (also called Peaking Background) is represented in the diagram by the region in yellow.

The Signal is represented by the red region on the diagram.

Around these representations, the a 2-dimensional model of the reconstructed invariant masses ($m_{\mu\mu\gamma}$ and $m_{\mu\mu}$) is constructed for each contributing process and tested against the collected data by the experiment, by means of a unbinned maximum likelihood fit. No significant excess above the background-only model is observed and a upper limit of the signal branch fraction is extracted. The following sections describes the data and simulated samples used in this analysis, the event selection applied in order to enhance the signal to background ratio and the process to construct the statistical models used in the upper limits extraction.
 
\section{Datasets and simulated events} \label{datasets_and_MC_samples}

\subsection{Data samples}

The data sample used in this analysis consists of the 2016 13 \TeV run
with 25 ns bunch separation recorded by CMS. This data sample is composed only by events that were certified from all CMS subsystems and and reconstruction specialist as good for physics analysis.

This data sample corresponds to 35.86 $fb^{-1}$ of integrated luminosity~\cite{CMS-PAS-LUM-17-001}.

\subsection{Simulated datasets}
\label{sec:datasets}

Data simulation at CMS is done by the use of Monte Carlo (from here on, simply called MC) simulations the generates pseudo-random events, constrained by the physics of the related process to which we are interested, including the effect of the produced particles interacting with the detector. The simulation starts with the \textbf{hard-scattering} process, at parton (constituents of the proton) level, done usually, by matrix element generators, which impose to the incoming and outgoing partons, the dynamics of the simulated process, according to some pre-defined theoretical model. Along the hard interaction simulation, the \textbf{fragmentation} process takes place. Since the the matrix element generator provide information on the parton level, it is necessary to extract the momentum distribution of the parton as a function of the $Q^2$ (transferred momentum) of the process. TO do so, MC generators use the PDFs (parton distribution functions) to sample those values, accordingly. The matrix element formalism also allows the simulation of the process, taking into account, different orders of perturbations, like NLO (next-to-leading order), NNLO (next-to-next-to-leading order), and so on.

After the hard-scattering, the \textbf{showering} process simulates the radiation emission by gluons and quarks in the initial and final states. Along the hard interaction, the other proton constituents may also interact through soft interaction. This part of the simulation is called \textbf{multiple parton interaction} (MPI). The last components of the simulation is the \textbf{hadronization} and the \textbf{decay of heavy hadrons and leptons}. The former one, imposes the QCD confinement to low energy quarks and gluons~\footnote{QCD forbids coloured objects to exist in non-asymptotic states. They must decays to more complex systems, until they form stable colorless states.}, while the latter one, implements specific models to decays heavy hadrons and leptons, like $B$ hadrons and taus.

Usually, different generators are used to simulate a process. Each specialized in one or more steps.

A summary of the signal and background MC samples used is presented in Table~\ref{tab:MC}. These simulated data are comparable with the proton-proton collision using 2016 data conditions and the \textbf{pileup}~\footnote{Each LHC collision recorded by CMS, is composed not by a single $pp$ interaction, but by a bunch of protons crossing. In this case, a hard interaction is actually surrounded by many soft interaction between adjacent protons. The extra activity produced in these interactions, and caught by the detector, is called \textbf{pileup} and has its signal mixed with the hard one. Since, during the data taking, the number of pileup interactions varies according to the instantaneous luminosity, the MC are mixed some soft interaction only sample, called Minimum Bias. The number of mixed soft interaction (pileup events) is governed by a Poisson distribution with a pre-defined mean. The MC samples are then reweighted to match the same distribution of primary vertexes for the corresponding Data. Systematics related to this procedure will be discussed later.} events are added to the simulated event in this step. The pileup events distribution used is modeled as a Poisson pdf (probability distribution function) with mean of 23 events, as recommended by CMS. Detector response in the MC samples is simulated using a detailed description of the CMS detector, based on \GEANTfour~\cite{Agostinelli:2002hh}.

The signal MC samples are simulated for the Higgs bosons decaying to $\Upsilon(nS) (\rightarrow \mu\mu) + \gamma$  channels with POWHEG v2.0~\cite{powheg2_1,powheg2_2,powheg2_3}, at next-to-leading order (NLO) of Feynman graphs computation, for the following production modes: gluon-gluon fusion (ggF), vector boson fusion (VBF), associated production (VH) and associated top production (ttH), with cross-section summarized at table \ref{tab:MC}. A extensive review of these production modes can be found at~\cite{DJOUADI20081}. The \PYTHIA 8 generator~\cite{SJOSTRAND2008852,Sjostrand:2014zea} is used for hadronization and fragmentation with underlying event tune CUETP8M1~\cite{Khachatryan:2015pea}. The parton distribution functions (pdf) NNPDF3.0~\cite{NNPDF3} are used. 

 For Z decaying to $\Upsilon(nS) (\rightarrow \mu\mu) + \gamma$  channels, the signal samples are simulated with \MADGRAPH5 $\_$\MCATNLO 2.6.0 matrix element generator~\cite{Alwall2014} at next leading order and the \PYTHIA 8 generator~\cite{SJOSTRAND2008852,Sjostrand:2014zea} for hadronization and fragmentation with underlying event tune CUETP8M1~\cite{Khachatryan:2015pea}.

The Drell-Yan process, $pp \rightarrow \Z \rightarrow \mu\mu \gamma_{FSR}$, results in the same final state as the signal. This process exhibits a peak in the three-body invariant mass, $m_{\mu\mu\gamma}$, at the \Z boson mass, $m_{Z}$, and it is a resonant background for this channel, therefore referred to as a Peaking Background. 

It is taken into account when deriving the upper limit on the branching fraction for Z$\rightarrow \Upsilon(nS) + \gamma \rightarrow \mu\mu + \gamma$. The \MADGRAPH5 $\_$\MCATNLO 2.6.0 matrix element generator~\cite{Alwall2014} at leading order, interfaced with \PYTHIA 8.226 for parton showering and hadronization with tune CUETP8M1~\cite{Khachatryan:2015pea}, is used to generate a sample of these resonant background events. The photons in these events are all produced as final-state radiation from the $ Z \rightarrow \mu\mu$ decay and therefore the $m_{\mu\mu\gamma}$ distribution peaks at the \Z boson mass and there is no continuum contribution.  

Similarly, the Higgs boson Dalitz decay~\cite{PhysRevD.55.5647}, $H \rightarrow \gamma^{*} \gamma\rightarrow \mu\mu + \gamma$, is a Peaking Background (resonant) to $H \rightarrow \Upsilon(nS) \rightarrow \mu\mu + \gamma$. It is simulated at NLO with \MADGRAPH5 $\_$\MCATNLO 2.6.0 matrix element generator~\cite{Alwall2014} at next-to-leading order and the \PYTHIA 8 generator~\cite{SJOSTRAND2008852,Sjostrand:2014zea} for hadronization and fragmentation with underlying event tune CUETP8M1~\cite{Khachatryan:2015pea}. This Higgs Dalitz Decay sample, was generated only for the ggF production mode, but its cross-section was rescaled to the full Higgs cross-section. This process will present a small contribuition of selected events, so this approximation should be sufficient for the Higgs Peaking Background modeling.

There are also background processes that do not give resonance peaks in the three-body invariant mass spectrum. They are modeled from data, as it will be explained latter in more details.
 
\begin{table}[htp]
\begin{center}
\caption{Datasets simulated (MC) for 2016 conditions. Assuming that $\sigma (pp\rightarrow$ H), taking into consideration all the simulated Higgs production modes, is 55.13 $pb$~\cite{CERNYellowReportPageAt13TeV} and  $\sigma (pp\rightarrow Z \rightarrow \mu\mu$ ) is 57094.5 $pb$, including the next-to-next-to-leading order
(NNLO) QCD contributions, and the next-to-leading order (NLO) electroweak corrections from fewz 3.1~\cite{FEWZ} calculated using the NLO PDF set NNPDF3.0, with the phase space selection in invariant mass of the dimuon system of $m_{\mu\mu} > 50$ GeV. For the Higgs Dalitz $\sigma$, we consider only the gluon fusion contribution ($\sigma_{\text{ggF}}  = $ 48.6 $pb$)~\cite{CERNYellowReportPageAt13TeV}. The Higgs Dalitz Decay $BR_{SM}$ and the $Z \rightarrow \mu\mu\gamma_{FSR}$ were obtained with MCFM 6.6~\cite{CAMPBELL201010} (as in the CMS search for Higgs Dalitz Decay in at $\sqrt{s} =$ 8 TeV~\cite{dalitz_decay_8_Tev}) and with \MADGRAPH5 $\_$\MCATNLO, respectively. The $BR^{PDG}_{\Upsilon(1S,2S,3S) \rightarrow \mu\mu} = \text{(2.48, 1.93, 2.18)} \times 10^{-2}$ is quoted from Particle Data Group report (PDG)~\cite{pdg_2020}. The "Effective $\sigma$" for the signal samples is $\sigma(pp \rightarrow Z(H)) \times BR_{SM} \times BR^{PDG}_{\Upsilon(nS) \rightarrow \mu\mu}$.}
%
%\begin{tabular}{lcccc} \hline
%Physics Processes &  $\sigma$ in pb   & Branching Ratio (BR$_{SM}$) & Generator & Sample Type \\ \hline
%H$\rightarrow  \Upsilon(1S) +\gamma$ & 7.1996$\times 10^{-9}$ &5.22$\times 10^{-9}$ & PYTHIA8 & Signal \\ 
%H$\rightarrow  \Upsilon(2S) +\gamma$ & 1.5242$\times 10^{-9}$ &1.42$\times 10^{-9}$  & PYTHIA8 & Signal \\ 
%H$\rightarrow  \Upsilon(3S) +\gamma$ & 1.1033$\times 10^{-9}$ &9.10$\times 10^{-10}$ & PYTHIA8 & Signal \\ \hline
%Z$\rightarrow  \Upsilon(1S) +\gamma$ & 6.7965$\times 10^{-5}$ &4.88$\times 10^{-8}$ & PYTHIA8 & Signal \\ 
%Z$\rightarrow  \Upsilon(2S) +\gamma$ & 2.6887$\times 10^{-5}$ &2.44$\times 10^{-8}$ &  PYTHIA8 & Signal \\
%Z$\rightarrow  \Upsilon(3S) +\gamma$ & 2.3400$\times 10^{-5}$ &1.88$\times 10^{-8}$ &  PYTHIA8 & Signal \\  \hline \hline
%H$\rightarrow \gamma\gamma^{*}$ Dalitz Decay & 1.8614$\times 10^{-3}$ & 3.83$\times 10^{-5}$ & --- & Peaking Background \\ 
%Z$\rightarrow  \mu\mu\gamma_{FSR}$ & 7.9260$\times 10^{-2}$& --- & --- & Peaking Background \\ \hline
%\end{tabular}





\begin{tabular}{lcccc} \hline
Physics Processes & Branching Ratio (BR$_{SM}$)  & Effective $\sigma$ (in pb) & Generator & Sample Type \\ \hline
H$\rightarrow  \Upsilon(1S) +\gamma$ &5.22$\times 10^{-9}$ & 7.20$\times 10^{-9}$ & \PYTHIA 8.226 & Signal \\ 
H$\rightarrow  \Upsilon(2S) +\gamma$ &1.42$\times 10^{-9}$ &  1.52$\times 10^{-9}$ & \PYTHIA 8.226 & Signal \\ 
H$\rightarrow  \Upsilon(3S) +\gamma$ &9.10$\times 10^{-10}$ & 1.10$\times 10^{-9}$ & \PYTHIA 8.226 & Signal \\ \hline
Z$\rightarrow  \Upsilon(1S) +\gamma$ &4.88$\times 10^{-8}$ & 6.80$\times 10^{-5}$ & \PYTHIA 8.226 & Signal \\ 
Z$\rightarrow  \Upsilon(2S) +\gamma$ &2.44$\times 10^{-8}$ & 2.69$\times 10^{-5}$ &  \PYTHIA 8.226 & Signal \\
Z$\rightarrow  \Upsilon(3S) +\gamma$ &1.88$\times 10^{-8}$ & 2.34$\times 10^{-5}$ &  \PYTHIA 8.226 & Signal \\  \hline \hline
H Dalitz Decay & 3.83$\times 10^{-5}$ & 1.86$\times 10^{-3}$ &\MADGRAPH5 & Resonant Background \\ 
Z$\rightarrow  \mu\mu\gamma_{FSR}$ & --- & 7.93 $\times 10^{-2}$ & \MADGRAPH5  & Resonant Background \\ \hline
\end{tabular}
\label{tab:MC}
\end{center}
\end{table}

The number of simulated events is is rescaled by the Effective $\sigma$, from table~\ref{tab:MC}, in order to match 35.86 $fb^{-1}$ of integrated luminosity, from the recorded data. Being $N = \sigma \mathcal{L}$, $N$ in the number of events for a process, $\sigma$ is the cross-section and $\mathcal{L}$ is the integrated luminosity, the reweighting factor, for a simulated sample is:

\begin{equation}
\label{eqn:mc_weight}
w_{MC} = \frac{\sigma \mathcal{L}}{N_{sim}},
\end{equation}
where $N_{sim}$ is the number of simulated events for a specific process.

The simulated sample are also corrected by the data pile-up distribution, since the pileup distribution of MC is different from the pileup distribution of data. The way to correct the MC is to assign a weight to each bin of the MC pileup distribution, with repect to the data. The rescaling is defined as the ration between normalized Pile-up (PU) distribution for Data and MC.

\begin{equation}
\label{eqn:mc_weight}
w_{PU}(n) = \frac{P^{Data}_{PU}(n)}{P^{Sim}_{PU}(n)},
\end{equation}
where $n$ is the number of interaction per bunch crossing (pile-up).

\section{Contribution of the \texorpdfstring{$\Upsilon(nS)$} p polarisation}
\label{sec:polarization}

% Tem que ter esse erro de digitacao no titulo senao nao compila.
Measurements of quarkonium polarization observables may yield information about quarkonium production mechanisms that are not available from the study of unpolarized cross sections alone. The three polarization states of a $J = 1$ quarkonium can be specified in terms of a particular coordinate system in the rest frame of the quarkonium. This coordinate system is often called the "spin-quantization frame". 

In a hadron collider, $\Upsilon(1S,2S,3S)$ are reconstructed through their electromagnetic decays into a lepton pair. The information about the polarization of the quarkonium state is encoded in the angular distribution of the leptons. This angular distribution is usually described in the quarkonium rest frame with respect to a particular spin-quantization frame~\cite{Brambilla:2011bph}. The polarization of the  $\Upsilon(1S,2S,3S)$ is not simulated for signal MC sample and we only apply a reweighting scale factor to each event and so we can emulate the polarization effects~\cite{PhysRevD.83.031503}. Figure \ref{fig:ZUpsilonPolarization} present the distributions of $\cos \Theta$ of $\Upsilon \rightarrow \mu\mu$, 
% and $\gamma^{*} \rightarrow \mu\mu$, 
where $\Theta$ is the angle between the positive muon and the $\Upsilon$ in the Z (Higgs) rest-frame. At Table \ref{fig:polTable} we show the the analytical functions used to describe the extremes scenarios (Unpolarized, Transverse Polarization and Longitudinal Polarization) reweighting, presented in this analysis. 

It is worth stating that, for the Higgs decay, on the Transverse Polarization is considered. For the Z decay, because of its spin nature, the Unpolarized scenario is used as the nominal one, and the effects of the two other extremes (Transverse Polarization and Longitudinal Polarization) are quoted as systematics.


% \begin{figure}[!htbp]
% \begin{center}
% \includegraphics[width=0.50\textwidth]{figures_and_tables/outputPlots/HtoUpsilon_Cat0_ZZZZZ/mc/polarizatioReweight/h_Gen_COS_theta}
% \end{center}\vspace*{-.5cm}
% \caption{Distributions of $\cos \theta$ of $\Upsilon \rightarrow \mu\mu$ and $\gamma^{*} \rightarrow \mu\mu$ The orange distribution is the $H \rightarrow  \Upsilon(1S,2S,3S) + \gamma$ sample before reweighting; the blue distribution is $H \rightarrow  \Upsilon(1S,2S,3S)$ sample after reweighting.}
% \label{fig:HUpsilonPolarization}
% \end{figure}


\begin{figure}[!htbp]
\begin{center}
\includegraphics[width=0.80\textwidth]{figures_and_tables/outputPlots/ZtoUpsilon_Cat0_ZZZZZ/mc/polarizatioReweight/h_Gen_COS_theta_extremes}
\end{center}\vspace*{-.5cm}
\caption{Distributions of $\cos \theta$ of $\Upsilon \rightarrow \mu\mu$ and $\gamma^{*} \rightarrow \mu\mu$ The orange distribution is the $Z \rightarrow  \Upsilon(1S,2S,3S) + \gamma$ sample before reweighting (Unpolarized); the blue and gray distributions are $Z \rightarrow  \Upsilon(1S,2S,3S)$ sample after reweighting, for the Transverse and Longitudinal Polarization.}
\label{fig:ZUpsilonPolarization}
\end{figure}


\begin{table}[htp]
\begin{center}
\caption{Summary of the impact of reweighted of polarization contribution using several scenarios.}
%\begin{table}[htp]
%\begin{center}
%\begin{tabular}{|cl }
\begin{tabular}{l|l|l}

% \hline

\textbf{$J_{Z}$} & \textbf{Polarisation Scenario} & \textbf{Analytic Description} \\ \hline
$\pm$ 1  & Transverse &  3/4 $\times (1 + (\cos \Theta)^{2})$ \\ \hline
0  & Longitudinal & 3/2 $\times (1- (\cos \Theta)^{2})$ \\ 

\end{tabular}
%\caption{Summary of data samples used for $H/Z \rightarrow \Upsilon(1S,2S,3S)+\gamma$ analysis }
%\label{Tablebkg}
%\end{center}
%\end{table}
% Ref latex: https://tex.stackexchange.com/questions/112343/beautiful-table-samples


\label{fig:polTable}
\end{center}
\end{table} 



\section{Kinematical studies using MC generator}


Using the \PYTHIA 8.226 generator, the Monte Carlo signals are produced for Higgs (Z) boson events decaying in ($\Upsilon$(1S,2S,3S)) + $\gamma$, which are highly boosted. Observing the kinematic generator level distributions in Figure \ref{fig:MC_ZtoUpsilon_Cat0} for Z boson and Figure \ref{fig:MC_HtoUpsilon_Cat0} for Higgs boson, we could conclude that the high-\ET (transverse energy, with respect to the beam line) photon will be back-to-back to the $\Upsilon$ particles being possible to apply an isolation selection to identify a photon in this kinematic topology. Also, we can observe those transverse momenta of the leading/trailing \PT (transverse momemtum, with respect to the beam line) muon~\footnote{In this study we define leading muon and the muon, decaying from the $\Upsilon$, with highest \PT. Trailing muon is the one with the second hight \PT.} and the photon and distances $\Delta R=\sqrt{\Delta\eta^2 + \Delta\phi^2}$ between the two muons and between the muons and the photon are a good variable that can be used to discriminate the contribution between signal and background events. The leading muon transverse momentum can be greater than 45(30)\GeV and trailing muon is greater than 10(20)\GeV in Higgs(Z) decay.
$\Delta R$ distributions of the two muons and between the muons and the photon in the both cases show that the two muons are very close and the photon is back-to-back in relation of dimuon system. Another feature of this kinematic topology is that the production vertex between muons produced in $\Upsilon$ decaying events and the high-\ET photon is very well defined.  

% Concerning the main reducible background (combinatorial background), it is due to Drell-Yan process where the photon comes from the initial-state radiation (ISR) or final-state radiation (FSR) and background reducible events are produced from Drell-Yan with jet associate and inclusive quarkonium production, where the jet in both processes is misidentified as a photon in reconstruction level. For the analysis, besides the peaking background, most of the background contributions will be modeled from data. 

\begin{figure}[!htbp]
\begin{center}
% Muon
%\hspace*{1.cm}
\includegraphics[width=0.45\textwidth]{figures_and_tables/outputPlots/ZtoUpsilon_Cat0_ZZZZZ/mc/unpolarized/h_Gen_Mu_pt}
%\hspace*{1.cm}
\includegraphics[width=0.45\textwidth]{figures_and_tables/outputPlots/ZtoUpsilon_Cat0_ZZZZZ/mc/unpolarized/h_Gen_Mu_eta}
%\hspace*{1.cm}
\includegraphics[width=0.45\textwidth]{figures_and_tables/outputPlots/ZtoUpsilon_Cat0_ZZZZZ/mc/unpolarized/h_Gen_Mu_phi}
%\hspace*{1.cm}
%Delta R mu_trading and mu_leading
\includegraphics[width=0.45\textwidth]{figures_and_tables/outputPlots/ZtoUpsilon_Cat0_ZZZZZ/mc/unpolarized/h_Gen_deltaR_Leading_Trailing}
%\hspace*{1.cm}
%Photon
\includegraphics[width=0.45\textwidth]{figures_and_tables/outputPlots/ZtoUpsilon_Cat0_ZZZZZ/mc/unpolarized/h_Gen_Photon_pt}
%\hspace*{1.cm}
\includegraphics[width=0.45\textwidth]{figures_and_tables/outputPlots/ZtoUpsilon_Cat0_ZZZZZ/mc/unpolarized/h_Gen_Photon_eta}
%\hspace*{1.cm}
\includegraphics[width=0.45\textwidth]{figures_and_tables/outputPlots/ZtoUpsilon_Cat0_ZZZZZ/mc/unpolarized/h_Gen_Photon_phi}
%\hspace*{1.cm}
%Delta R mu+ photon
\includegraphics[width=0.45\textwidth]{figures_and_tables/outputPlots/ZtoUpsilon_Cat0_ZZZZZ/mc/unpolarized/h_Gen_deltaR_Mu_Photon}
%\hspace*{1.cm}
%Upsilon
%\includegraphics[width=0.25\textwidth]{figures_and_tables/outputPlots/ZtoUpsilon_Cat0_ZZZZZ/mc/unpolarized/h_Gen_Upsilon_Pt}
%\hspace*{1.cm}
%\includegraphics[width=0.25\textwidth]{figures_and_tables/outputPlots/ZtoUpsilon_Cat0_ZZZZZ/mc/unpolarized/h_Gen_Upsilon_eta}
%\hspace*{1.cm}
%\includegraphics[width=0.25\textwidth]{figures_and_tables/outputPlots/ZtoUpsilon_Cat0_ZZZZZ/mc/unpolarized/h_Gen_Upsilon_phi}
%\hspace*{1.cm}
%Z
%\includegraphics[width=0.25\textwidth]{figures_and_tables/outputPlots/ZtoUpsilon_Cat0_ZZZZZ/mc/unpolarized/h_Gen_Z_Pt}
%\hspace*{1.cm}
%\includegraphics[width=0.25\textwidth]{figures_and_tables/outputPlots/ZtoUpsilon_Cat0_ZZZZZ/mc/unpolarized/h_Gen_Z_eta}
%\hspace*{1.cm}
%\includegraphics[width=0.25\textwidth]{figures_and_tables/outputPlots/ZtoUpsilon_Cat0_ZZZZZ/mc/unpolarized/h_Gen_Z_phi}
%\hspace*{1.cm}
%h_Gen_deltaR_Mu_Photon.png
\end{center}%\vspace*{-.5cm}
\caption{Generator level distributions of main variables for $Z\rightarrow  \Upsilon(1S,2S,3S) + \gamma$ : Transverse momenta of the leading/trailing \PT muon and the photon, pseudorapidity ($\eta$) and $\phi$ of the muons and the photon, distances $\Delta R$ between the two muons and between the muons and the photon. All the distributions shown in the figure are normalized to the unity of area.}
\label{fig:MC_ZtoUpsilon_Cat0}
\end{figure}


%%%%
\begin{figure}[!htbp]
\begin{center}
% Muon
%\hspace*{1.cm}
\includegraphics[width=0.45\textwidth]{figures_and_tables/outputPlots/HtoUpsilon_Cat0_ZZZZZ/mc/unpolarized/h_Gen_Mu_pt}%\hspace*{1.cm}
\includegraphics[width=0.45\textwidth]{figures_and_tables/outputPlots/HtoUpsilon_Cat0_ZZZZZ/mc/unpolarized/h_Gen_Mu_eta}
%\hspace*{1.cm}
\includegraphics[width=0.45\textwidth]{figures_and_tables/outputPlots/HtoUpsilon_Cat0_ZZZZZ/mc/unpolarized/h_Gen_Mu_phi}
%\hspace*{1.cm}
%Delta R mu_trading and mu_leading
\includegraphics[width=0.45\textwidth]{figures_and_tables/outputPlots/HtoUpsilon_Cat0_ZZZZZ/mc/unpolarized/h_Gen_deltaR_Leading_Trailing}
%Photon
\includegraphics[width=0.45\textwidth]{figures_and_tables/outputPlots/HtoUpsilon_Cat0_ZZZZZ/mc/unpolarized/h_Gen_Photon_pt}
%\hspace*{1.cm}
\includegraphics[width=0.45\textwidth]{figures_and_tables/outputPlots/HtoUpsilon_Cat0_ZZZZZ/mc/unpolarized/h_Gen_Photon_eta}
%\hspace*{1.cm}
\includegraphics[width=0.45\textwidth]{figures_and_tables/outputPlots/HtoUpsilon_Cat0_ZZZZZ/mc/unpolarized/h_Gen_Photon_phi}
%\hspace*{1.cm}
%Delta R mu+ photon
\includegraphics[width=0.45\textwidth]{figures_and_tables/outputPlots/HtoUpsilon_Cat0_ZZZZZ/mc/unpolarized/h_Gen_deltaR_Mu_Photon}
%\hspace*{1.cm}
%Upsilon
%\includegraphics[width=0.25\textwidth]{figures_and_tables/outputPlots/HtoUpsilon_Cat0_ZZZZZ/mc/unpolarized/h_Gen_Upsilon_Pt}
%\hspace*{1.cm}
%\includegraphics[width=0.25\textwidth]{figures_and_tables/outputPlots/HtoUpsilon_Cat0_ZZZZZ/mc/unpolarized/h_Gen_Upsilon_eta}
%\hspace*{1.cm}
%\includegraphics[width=0.25\textwidth]{figures_and_tables/outputPlots/HtoUpsilon_Cat0_ZZZZZ/mc/unpolarized/h_Gen_Upsilon_phi}
%\hspace*{1.cm}
%Z
%\includegraphics[width=0.25\textwidth]{figures_and_tables/outputPlots/HtoUpsilon_Cat0_ZZZZZ/mc/unpolarized/h_Gen_Z_Pt}
%\hspace*{1.cm}
%\includegraphics[width=0.25\textwidth]{figures_and_tables/outputPlots/HtoUpsilon_Cat0_ZZZZZ/mc/unpolarized/h_Gen_Z_eta}
%\hspace*{1.cm}
%\includegraphics[width=0.25\textwidth]{figures_and_tables/outputPlots/HtoUpsilon_Cat0_ZZZZZ/mc/unpolarized/h_Gen_Z_phi}
%\hspace*{1.cm}
%\includegraphics[width=0.25\textwidth]{figures_and_tables/outputPlots/ZtoUpsilon_Cat0_ZZZZZ/mc/unpolarized/h_Gen_Z_Mass}
%\hspace*{1.cm}
\end{center}%\vspace*{-.5cm}
\caption{Generator level distributions of main variables for $H\rightarrow  \Upsilon(1S,2S,3S) + \gamma$ : Transverse momenta of the leading/trailing $\PT$ muon and the photon, pseudorapidity ($\eta$) and $\phi$ of the muons and the photon, distances $\Delta R$ between the two muons and between the muons and the photon.All the distributions shown in the figure are normalized to the unity of area.}
\label{fig:MC_HtoUpsilon_Cat0}
\end{figure}

 
\clearpage
%%%%%%%%%%%%%%%%%%%%%%%%%%%%%%%%%%%%%

\section{Event selection}
\label{sec:selection}

The event selection is divided in two steps. At first, a set of cuts related to trigger and physics object (muons and photons) selection is applied. High level physics objects at CMS are reconstructed based of the Particle Flow (PF) algorithm \cite{PF_paper_2016}. This selection is called, within this analysis, Group I. 

For the events that pass the Group I selection, another set of cuts is applied, this time focusing on kinematical (phase space) event selection, in order to enhance the signal to background ratio. This later set is called, within this analysis, Group II. After full selection, three exclusive categories are defined, based on the photon's $\eta$ region and its energy spread shape within the ECAL cells (R9).

% It was a deliberated choice to apply on this analysis the same event selection of the $H/Z \rightarrow J/\psi + \gamma$  with 2016 data. Full documentation on this analysis can be found at \cite{CMS_jpsi_analysis}. Since the two analyses are similar, not only in final state, but also on the kind of intermediate meson ($\Upsilon$ or $J/\psi$) and energy scale, this choice was taken in order to keep the compatibility of the two studies. The only difference in the event selection is the dimuon mass window, which was changed to a suitable cut for a $\Upsilon$ selection.  

After the full selection, a background and signal modeling process is applied, based on the invariant mass distributions, which will be explained in the next section.


%%%%%%%%%%%%%%%%%%%%%%%%%%%%%%%%%%%%%%%%%%%%%%%%%%%%%%%%%%%%%%%%%%%%%%%%%%%%%%%%%%%%%%%%%%%%%%%%%%%%%%%%%%%%%%%%%%%%%%%%
%%%%%%%%%%%%%%%%%%%%%%%%%%%%%%%%%%%%%%%%%%%%%%%%%%%%%%%%%%%%%%%%%%%%%%%%%%%%%%%%%%%%%%%%%%%%%%%%%%%%%%%%%%%%%%%%%%%%%%%%
%%%%%%%%%%%%%%%%%%%%%%%%%%%%%%%%%%%%%%%%%%%%%%%%%%%%%%%%%%%%%%%%%%%%%%%%%%%%%%%%%%%%%%%%%%%%%%%%%%%%%%%%%%%%%%%%%%%%%%%%
%%%%%%%%%%%%%%%%%%%%%%%%%%%%%%%%%%%%%%%%%%%%%%%%%%%%%%%%%%%%%%%%%%%%%%%%%%%%%%%%%%%%%%%%%%%%%%%%%%%%%%%%%%%%%%%%%%%%%%%%
%%%%%%%%%%%%%%%%%%%%%%%%%%%%%%%%%%%%%%%%%%%%%%%%%%%%%%%%%%%%%%%%%%%%%%%%%%%%%%%%%%%%%%%%%%%%%%%%%%%%%%%%%%%%%%%%%%%%%%%%
%%%%%%%%%%%%%%%%%%%%%%%%%%%%%%%%%%%%%%%%%%%%%%%%%%%%%%%%%%%%%%%%%%%%%%%%%%%%%%%%%%%%%%%%%%%%%%%%%%%%%%%%%%%%%%%%%%%%%%%%
%%%%%%%%%%%%%%%%%%%%%%%%%%%%%%%%%%%%%%%%%%%%%%%%%%%%%%%%%%%%%%%%%%%%%%%%%%%%%%%%%%%%%%%%%%%%%%%%%%%%%%%%%%%%%%%%%%%%%%%%
%%%%%%%%%%%%%%%%%%%%%%%%%%%%%%%%%%%%%%%%%%%%%%%%%%%%%%%%%%%%%%%%%%%%%%%%%%%%%%%%%%%%%%%%%%%%%%%%%%%%%%%%%%%%%%%%%%%%%%%%
%%%%%%%%%%%%%%%%%%%%%%%%%%%%%%%%%%%%%%%%%%%%%%%%%%%%%%%%%%%%%%%%%%%%%%%%%%%%%%%%%%%%%%%%%%%%%%%%%%%%%%%%%%%%%%%%%%%%%%%%
%%%%%%%%%%%%%%%%%%%%%%%%%%%%%%%%%%%%%%%%%%%%%%%%%%%%%%%%%%%%%%%%%%%%%%%%%%%%%%%%%%%%%%%%%%%%%%%%%%%%%%%%%%%%%%%%%%%%%%%%
%%%%%%%%%%%%%%%%%%%%%%%%%%%%%%%%%%%%%%%%%%%%%%%%%%%%%%%%%%%%%%%%%%%%%%%%%%%%%%%%%%%%%%%%%%%%%%%%%%%%%%%%%%%%%%%%%%%%%%%%
%%%%%%%%%%%%%%%%%%%%%%%%%%%%%%%%%%%%%%%%%%%%%%%%%%%%%%%%%%%%%%%%%%%%%%%%%%%%%%%%%%%%%%%%%%%%%%%%%%%%%%%%%%%%%%%%%%%%%%%%
%%%%%%%%%%%%%%%%%%%%%%%%%%%%%%%%%%%%%%%%%%%%%%%%%%%%%%%%%%%%%%%%%%%%%%%%%%%%%%%%%%%%%%%%%%%%%%%%%%%%%%%%%%%%%%%%%%%%%%%%
%
% _______  ______    _______  __   __  _______    ___  
%|       ||    _ |  |       ||  | |  ||       |  |   | 
%|    ___||   | ||  |   _   ||  | |  ||    _  |  |   | 
%|   | __ |   |_||_ |  | |  ||  |_|  ||   |_| |  |   | 
%|   ||  ||    __  ||  |_|  ||       ||    ___|  |   | 
%|   |_| ||   |  | ||       ||       ||   |      |   | 
%|_______||___|  |_||_______||_______||___|      |___| 
%
%%%%%%%%%%%%%%%%%%%%%%%%%%%%%%%%%%%%%%%%%%%%%%%%%%%%%%%%%%%%%%%%%%%%%%%%%%%%%%%%%%%%%%%%%%%%%%%%%%%%%%%%%%%%%%%%%%%%%%%%
%%%%%%%%%%%%%%%%%%%%%%%%%%%%%%%%%%%%%%%%%%%%%%%%%%%%%%%%%%%%%%%%%%%%%%%%%%%%%%%%%%%%%%%%%%%%%%%%%%%%%%%%%%%%%%%%%%%%%%%%
%%%%%%%%%%%%%%%%%%%%%%%%%%%%%%%%%%%%%%%%%%%%%%%%%%%%%%%%%%%%%%%%%%%%%%%%%%%%%%%%%%%%%%%%%%%%%%%%%%%%%%%%%%%%%%%%%%%%%%%%
%%%%%%%%%%%%%%%%%%%%%%%%%%%%%%%%%%%%%%%%%%%%%%%%%%%%%%%%%%%%%%%%%%%%%%%%%%%%%%%%%%%%%%%%%%%%%%%%%%%%%%%%%%%%%%%%%%%%%%%%
%%%%%%%%%%%%%%%%%%%%%%%%%%%%%%%%%%%%%%%%%%%%%%%%%%%%%%%%%%%%%%%%%%%%%%%%%%%%%%%%%%%%%%%%%%%%%%%%%%%%%%%%%%%%%%%%%%%%%%%%
%%%%%%%%%%%%%%%%%%%%%%%%%%%%%%%%%%%%%%%%%%%%%%%%%%%%%%%%%%%%%%%%%%%%%%%%%%%%%%%%%%%%%%%%%%%%%%%%%%%%%%%%%%%%%%%%%%%%%%%%
%%%%%%%%%%%%%%%%%%%%%%%%%%%%%%%%%%%%%%%%%%%%%%%%%%%%%%%%%%%%%%%%%%%%%%%%%%%%%%%%%%%%%%%%%%%%%%%%%%%%%%%%%%%%%%%%%%%%%%%%
%%%%%%%%%%%%%%%%%%%%%%%%%%%%%%%%%%%%%%%%%%%%%%%%%%%%%%%%%%%%%%%%%%%%%%%%%%%%%%%%%%%%%%%%%%%%%%%%%%%%%%%%%%%%%%%%%%%%%%%%
%%%%%%%%%%%%%%%%%%%%%%%%%%%%%%%%%%%%%%%%%%%%%%%%%%%%%%%%%%%%%%%%%%%%%%%%%%%%%%%%%%%%%%%%%%%%%%%%%%%%%%%%%%%%%%%%%%%%%%%%
%%%%%%%%%%%%%%%%%%%%%%%%%%%%%%%%%%%%%%%%%%%%%%%%%%%%%%%%%%%%%%%%%%%%%%%%%%%%%%%%%%%%%%%%%%%%%%%%%%%%%%%%%%%%%%%%%%%%%%%%
%%%%%%%%%%%%%%%%%%%%%%%%%%%%%%%%%%%%%%%%%%%%%%%%%%%%%%%%%%%%%%%%%%%%%%%%%%%%%%%%%%%%%%%%%%%%%%%%%%%%%%%%%%%%%%%%%%%%%%%%
%%%%%%%%%%%%%%%%%%%%%%%%%%%%%%%%%%%%%%%%%%%%%%%%%%%%%%%%%%%%%%%%%%%%%%%%%%%%%%%%%%%%%%%%%%%%%%%%%%%%%%%%%%%%%%%%%%%%%%%%
%%%%%%%%%%%%%%%%%%%%%%%%%%%%%%%%%%%%%%%%%%%%%%%%%%%%%%%%%%%%%%%%%%%%%%%%%%%%%%%%%%%%%%%%%%%%%%%%%%%%%%%%%%%%%%%%%%%%%%%%
%%%%%%%%%%%%%%%%%%%%%%%%%%%%%%%%%%%%%%%%%%%%%%%%%%%%%%%%%%%%%%%%%%%%%%%%%%%%%%%%%%%%%%%%%%%%%%%%%%%%%%%%%%%%%%%%%%%%%%%%
%%%%%%%%%%%%%%%%%%%%%%%%%%%%%%%%%%%%%%%%%%%%%%%%%%%%%%%%%%%%%%%%%%%%%%%%%%%%%%%%%%%%%%%%%%%%%%%%%%%%%%%%%%%%%%%%%%%%%%%%

\section{Trigger and physics object selection (Group I)}

\subsection{Trigger}
\label{sec:trigger}
% The trigger used to select events, At hardware level (called L1), it is seeded by at least one muon with $p_{T} > 5$ GeV (transverse momentum with respect to the proton beam line)  and a isolated E/$\gamma$ with $p_{T} > 18$ GeV. At software level (HLT), it is activated by at least one muon with $p_{T} > 17$ GeV and a Photon of $p_{T} > 30$ GeV.

In this study, the same trigger requirements are applied to both data and simulated samples. For the first trigger level (L1), events are selected if they present at least one muon with transverse momentum greater than 5 GeV and an isolated~\footnote{The concept of isolation will be detailed later, but in summary, it means a object which has very small activity (above a certain threshold of energy/momentum) around it. In hadron-hadron collider, isolation is a characteristics of hard interactions.} photon or electron with transverse momentum greater than 18 GeV (at L1, there is no differentiation between photons and electrons). At the software level of the trigger system (HLT), the events are required to have at least one muon with transverse momentum greater than 17 GeV and a photon with transverse momentum greater than 30 GeV.

In order to compensate any difference in the trigger performance between simulated and data samples, for every selected MC a proper scale factor is applied, based on the the \PT of the reconstructed muon and photon. These scale factor computed by the ratio between efficiency of the trigger for the Data sample over the efficiency for a MC sample. These efficiencies are calculated with the tag-and-probe method, exploring the the resonance of a final state composted by two muon and one photon in the vicinity of the $Z$ boson invariant mass. To this final state, a selections was applied to ensure that the photon comes from a Final State Radiation process, allowing us to use the tag-and-probe method.

Considering the similarity of this analysis with the $H/Z \rightarrow J/\psi + \gamma$ analysis~\cite{papper_jpsi}, not only in therm of data samples, but also for triggering and physics object selection, the same scale factors were applied. More details are given in the same paper.

\subsection{Muon Identification}
\label{sec:muon_id}
% After trigger selection, the same muon selection of the $H \rightarrow ZZ^{*} \rightarrow 4l$ is applied \cite{CMS_higgs_zz_4l}. This is the same approach used by the $H/Z \rightarrow J/\psi + \gamma$ analysis. 

Ahead of any selection, a standard CMS "Ghost Cleaning" procedure is applied to all reconstructed muons in order to avoid that a single physical muon is reconstructed as two or more. For this procedure, reconstructed muons sharing 50\% or more segments in the system are arbitrated.

After the cleaning, a muon is chosen when it passes a a two step identification: the \textbf{Loose ID} and the \textbf{Tight ID}. Below the muon identification procedure is summarized .

% One should keep in mind that the naming conventions for these definitions are exclusive for the $H \rightarrow ZZ^{*} \rightarrow 4l$ and, therefore, to this analysis, not corresponding to the Muon-POG definitions. 

For the Loose ID, each muon is required to:
\begin{itemize}
  \item have transverse momentum greater than 5 GeV, in order to cope with Particle Flow requirements;
  \item be within the muon system acceptance: $|\eta| < 2.4$;
  \item to have a three dimensional impact parameter uncertainty smaller than 4;
  \item to have transverse distance smaller than 0.5 cm ($d_{xy} < 0.5$), with respect to the primary vertex (PV);
  \item to have longitudinal distance greater than 1.0 cm ($d_{z} < 1$), with respect to the primary vertex (PV).

  % \item reconstructed as \textbf{Global Muon} or \textbf{Tracker Muon};
  % \item muon tracks only in the muon system are rejected;
  % \item muons with \texttt{muonBestTrackType==2} (standalone) are discarded.
\end{itemize}

Muons reconstructed only in the muon system, without a correspondence with the tracker, are rejected. The last three requirements of the Loose ID are imposed in order to suppress muons from in-flight decays.

The primary vertex itself, is determined as the reconstructed vertex with the biggest sum of $p_T^2$ in the event. This sum is performed, considering all the charged PF candidates clustered by the jet finding algorithms~\cite{Cacciari:2008gp,Cacciari:2011ma} and the MET, which is defined as the $p_T$ vector sum of all the charged and neutral PF candidates associated to that vertex. 

For the Tight ID, muons with transverse momentum $p_{T} < 200$ GeV, are required to have been reconstructed with the Particle Flow (PF) algorithm. If they have $p_{T} > 200$ GeV, they should reconstructed with the Particle Flow (PF) algorithm or satisfy the strict tracker requirements (defined in table~\ref{tab:Tracker_High_pT}).

\begin{table}[h]
    \begin{center}
    \begin{tabular}{l|l}
     % \hline
     \textbf{Requirement}         & \textbf{Technical definition}                 \\
      \hline
      Muon station matching          & Muon is matched to segments           \\
                                     & in at least two stations in the muon system        \\
      \hline                                                          
      Good $\pt$ measurement         & $\frac{\pt}{\sigma_{\pt}} < 0.3$      \\
      \hline
      Vertex compatibility ($x-y$)   & $d_{xy} < 2$~mm                       \\
      \hline
      Vertex compatibility ($z$)     & $d_{z} < 5$~mm                        \\
      \hline
      Pixel hits                     & At least one pixel hit                \\
      \hline
      Tracker hits                   & Hits in at least six tracker layers   \\
      \hline
    \end{tabular}
    \caption{
      Conditions for a muon to pass the strict tracker requirements.
      }
    \label{tab:Tracker_High_pT}
    \end{center}
\end{table}


To mitigate spurious signal in the detector that would mimic a muon, the leading muon (the one with highest $p_T$) is required to be isolated within a cone of radious $\Delta R = \sqrt{\smash[b]{\Delta \eta}^2 + {\Delta \phi}^2}< 0.3$ in the $\eta - \phi$ plane. The isolation is evaluated in terms of ${\cal I}^{\mu} < 0.35$, defined as:


\begin{equation}
\label{eqn:pfiso}
{\cal I}^{\mu} \equiv \Big( \sum \PT^\text{charged} +
                                 \max\left[ 0, \sum \PT^\text{neutral}
                                 +
                                  \sum \PT^{\cPgg}
                                 - \PT^\mathrm{PU}(\mu) \right] \Big)
                                 / \PT^{\mu}.
\end{equation}


The $\sum \PT^\text{charged}$ is the scalar sum of the transverse momenta of charged hadrons originating from the chosen primary vertex of the event. The $\sum \PT^\text{neutral}$ and $\sum \PT^{\cPgg}$ are the scalar sums of the transverse momenta for neutral hadrons and photons, respectively.  Since the isolation variable is particularly sensitive to energy deposits from pileup interactions, a $\PT^\text{PU}(\mu)$ contribution is subtracted, where $\PT^\mathrm{PU}(\PGm) \equiv 0.5 \times \sum_i \PT^{\mathrm{PU}, i}$, where $i$ runs over the momenta of the charged hadron PF candidates not originating from the primary vertex, and the factor of 0.5 corrects for the different fraction of charged and neutral particles in the cone. 


% The last requirement over the selected muons is that the leading muon should be relatively isolated, over the Particle Flow ($\text{RelPFiso}(\Delta R = 0.3) < 0.35$).


% \begin{equation}
% \text{RelPFiso} = \frac{\sum^\text{charged had.} \pt + \max(\sum^\text{neutral had.} \ET 
% + \sum^\text{photon} \ET - \Delta \beta, 0)}{\pt^\text{lepton}}
% \label{eqn:mupfiso}
% \end{equation}

% The $\Delta\beta$, defined as $\Delta\beta = \frac{1}{2} \sum^\text{charged had.}_\text{PU} \pt$ gives an estimate of the pile-up contribution.  



One should keep in mind that this muon identification is the same as the one used by the $H \rightarrow ZZ^{*} \rightarrow 4l$~\cite{higgs_zz_4l_papper}. This was done in order to keep in phase with other Higgs analysis inside the collaboration. After the muon identification, an appropriate scale factor is applied to the MC events based on the leading muon \PT and $\eta$, in order to correct any possible discrepancy between data and simulated samples. The scale factors were taken from the $H \rightarrow ZZ^{*} \rightarrow 4l$ analysis.

In order to cope with trigger requirements, the leading muon should have $\PT > 20$ GeV and the trailing muon $\PT > 4$ GeV.


\subsection{Photon Identification}
\label{sec:photon_id}

For the photon identification and selection, standard CMS . The Multivariate (MVA) Photon identification is used with a working point of 90\%, together with a electron veto procedure, to avoid misidentification of electrons as photons. Kinematically, the photons are requested to have transverse energy, with respect to the beam line, $E_{T} > 33$ GeV and reconstructed within the CMS acceptance for photons $|\eta_{SC}| < 2.5$\footnote{SC stands for Super Cluster of the reconstructed photon. It is the set of cell in the Electromagnetic Calorimeter used for the photon observation.}, excluding the Electromagnetic Calorimeter (ECAL) Barrel-Endcap intersections.

The threshold of 33 GeV for the photon transverse energy is driven by the trigger requirements. The selecte photon, per event, is the one with highest $E_{T}$.


\subsection{Kinematical distributions}
\label{sec:kin_plots_group_1}

The selection described so far, is called Group I. The plots shown below are related to selected events after this set. 

Figures \ref{fig:pTMuons_ZtoUpsilon_Cat0} to \ref{fig:phiPhoton_ZtoUpsilon_Cat0} presents the \PT, $\eta$ and $\phi$ distributions for the leading muon, trailing muon and the photon, for the \Z decaying in $\Upsilon(1S,2S,3S)$ + $\gamma$. 

Figures \ref{fig:pTUpsilon_and_Z_ZtoUpsilon_Cat0} to \ref{fig:phiUpsilon_and_Z_ZtoUpsilon_Cat0} presents the \PT, $\eta$ and $\phi$ distributions for reconstructed $\Upsilon(nS)$ ($\mu\mu$ system) and the reconstructed boson ($\mu\mu\gamma$ system).

Figures \ref{fig:deltaR_ZtoUpsilon_Cat0} to \ref{fig:dimuon_mass_ZtoUpsilon_Cat0} presents the $\Delta R = \sqrt{\Delta\eta^2 + \Delta\phi^2}$ between the photon and the muons, the $\Delta R$ distributions between reconstructed dimuon ($\mu\mu$) system and the photon, the absolute value of the $\Delta \phi$ between the leading muon and the photon, the ratio for the transverse momentum of the reconstructed Upsilon and the reconstructed Z mass ($p_{T}^{\mu\mu}/M_{\mu\mu\gamma}$), the ratio for the transverse energy of the reconstructed Photon and the reconstructed Z mass ($E_{T}^{\mu\mu}/M_{\mu\mu\gamma}$) and dimuon mass distribution of the reconstructed $\Upsilon(nS)$.

Figures \ref{fig:pTMuons_HtoUpsilon_Cat0} to \ref{fig:dimuon_mass_HtoUpsilon_Cat0} present the same variables, but for the Higgs decaying in $\Upsilon(1S,2S,3S)$ + $\gamma$ channel.

%CONTROL PLOTS
%%$\pT$ muon distributions for ZtoUpsilon_Cat0
\begin{figure}[!htbp]
\begin{center}
\includegraphics[width=0.45\textwidth]{figures_and_tables/outputPlots/ZtoUpsilon_Cat0_ZZZZZ/au/data_x_mc/noKinCuts/h_noKin_TrailingMu_pt}\hspace*{1.cm}
\includegraphics[width=0.45\textwidth]{figures_and_tables/outputPlots/ZtoUpsilon_Cat0_ZZZZZ/au/data_x_mc/noKinCuts/h_noKin_LeadingMu_pt}
\end{center}\vspace*{-.5cm}
\caption{The \PT muon distributions from data and signal events for \Z decaying in $\Upsilon(1S,2S,3S)$ + $\gamma$ after Group I of selection cuts, where on left are presenting the trailing muons and on right are the leading muons. The plots are normalized to the unit of area.}
\label{fig:pTMuons_ZtoUpsilon_Cat0}
\end{figure}


%%%%%%%$\eta$ muon distributions for ZtoUpsilon_Cat0
\begin{figure}[!htbp]
\begin{center}
\includegraphics[width=0.45\textwidth]{figures_and_tables/outputPlots/ZtoUpsilon_Cat0_ZZZZZ/au/data_x_mc/noKinCuts/h_noKin_TrailingMu_eta}\hspace*{1.cm}
\includegraphics[width=0.45\textwidth]{figures_and_tables/outputPlots/ZtoUpsilon_Cat0_ZZZZZ/au/data_x_mc/noKinCuts/h_noKin_LeadingMu_eta}
\end{center}\vspace*{-.5cm}
\caption{The $\eta$ muon distributions from data and signal events of \Z decaying in $\Upsilon(1S,2S,3S)$ + $\gamma$ after Group I of selection cuts, where on left are presenting the trailing muons and on right are the leading muons. The plots are normalized to the unit of area.}
\label{fig:etaMuons_ZtoUpsilon_Cat0}
\end{figure}

%%%%%%%%% $\phi$ muon distributions for ZtoUpsilon_Cat0
\begin{figure}[!htbp]
\begin{center}
\includegraphics[width=0.45\textwidth]{figures_and_tables/outputPlots/ZtoUpsilon_Cat0_ZZZZZ/au/data_x_mc/noKinCuts/h_noKin_TrailingMu_phi}\hspace*{1.cm}
\includegraphics[width=0.45\textwidth]{figures_and_tables/outputPlots/ZtoUpsilon_Cat0_ZZZZZ/au/data_x_mc/noKinCuts/h_noKin_LeadingMu_phi}
\end{center}\vspace*{-.5cm}
\caption{The $\phi$ muon distributions from data and signal events of \Z decaying in $\Upsilon(1S,2S,3S)$ + $\gamma$ after Group I of selection cuts, where on left are presenting the trailing muons and on right are the leading muons. The plots are normalized to the unit of area.}
\label{fig:phiMuons_ZtoUpsilon_Cat0}
\end{figure}

%%%%%%%%%%%%%

%photon
%%$\pT$ Photon distributions for ZtoUpsilon_Cat0
\begin{figure}[!htbp]
\begin{center}
\includegraphics[width=0.45\textwidth]{figures_and_tables/outputPlots/ZtoUpsilon_Cat0_ZZZZZ/au/data_x_mc/noKinCuts/h_noKin_Photon_pt}\hspace*{1.cm}
\end{center}\vspace*{-.5cm}
\caption{The \PT photon distributions from data and signal events for Z decaying in $\Upsilon(1S,2S,3S)$ + $\gamma$ Group I of selection cuts. The plots normalized to the unit of area.}
\label{fig:pTPhoton_ZtoUpsilon_Cat0}
\end{figure}


%%%%%%%$\eta$ Photon distributions for ZtoUpsilon_Cat0
\begin{figure}[!htbp]
\begin{center}
\includegraphics[width=0.45\textwidth]{figures_and_tables/outputPlots/ZtoUpsilon_Cat0_ZZZZZ/au/data_x_mc/noKinCuts/h_noKin_Photon_eta}\hspace*{1.cm}
\end{center}\vspace*{-.5cm}
\caption{The $\eta$ photon distributions from data and signal events of Z decaying in $\Upsilon(1S,2S,3S)$ + $\gamma$ after Group I of selection cuts. The plot is normalized to the unit of area.}
\label{fig:etaPhoton_ZtoUpsilon_Cat0}
\end{figure}

%%%%%%%%% $\phi$ Photon distributions for ZtoUpsilon_Cat0
\begin{figure}[!htbp]
\begin{center}
\includegraphics[width=0.45\textwidth]{figures_and_tables/outputPlots/ZtoUpsilon_Cat0_ZZZZZ/au/data_x_mc/noKinCuts/h_noKin_Photon_phi}\hspace*{1.cm}
\end{center}\vspace*{-.5cm}
\caption{The $\phi$ photon distributions from data and signal events of Z decaying in $\Upsilon(1S,2S,3S)$ + $\gamma$ after Group I of selection cuts. The plot is normalized to the unit of area.}
\label{fig:phiPhoton_ZtoUpsilon_Cat0}
\end{figure}

%%%%%%%%%%%%%
% Upsilon and Z boson
%%$\pT$ Upsilon_and_Higgs distributions for ZtoUpsilon_Cat0
\begin{figure}[!htbp]
\begin{center}
\includegraphics[width=0.45\textwidth]{figures_and_tables/outputPlots/ZtoUpsilon_Cat0_ZZZZZ/au/data_x_mc/noKinCuts/h_noKin_Upsilon_Pt}\hspace*{1.cm}
\includegraphics[width=0.45\textwidth]{figures_and_tables/outputPlots/ZtoUpsilon_Cat0_ZZZZZ/au/data_x_mc/noKinCuts/h_noKin_Z_Pt}
\end{center}\vspace*{-.5cm}
\caption{The \PT distributions for the reconstructed $\Upsilon(1S,2S,3S)$ in the left and for Z in the right from data and signal events for Z decaying in $\Upsilon(1S,2S,3S)$ + $\gamma$ after Group I of selection cuts. The plots are normalized to the unit of area.}
\label{fig:pTUpsilon_and_Z_ZtoUpsilon_Cat0}
\end{figure}


%%%%%%%$\eta$ Upsilon_and_Z distributions for ZtoUpsilon_Cat0
\begin{figure}[!htbp]
\begin{center}
\includegraphics[width=0.45\textwidth]{figures_and_tables/outputPlots/ZtoUpsilon_Cat0_ZZZZZ/au/data_x_mc/noKinCuts/h_noKin_Upsilon_eta}\hspace*{1.cm}
\includegraphics[width=0.45\textwidth]{figures_and_tables/outputPlots/ZtoUpsilon_Cat0_ZZZZZ/au/data_x_mc/noKinCuts/h_noKin_Z_eta}
\end{center}\vspace*{-.5cm}
\caption{The $\eta$ distributions for $\Upsilon(1S,2S,3S)$ in the left and for Z in the right from data and signal events for Z decaying in $\Upsilon(1S,2S,3S)$ + $\gamma$ after Group I of selection cuts. The plots are normalized to the unit of area.}
\label{fig:etaUpsilon_and_Z_ZtoUpsilon_Cat0}
\end{figure}

%%%%%%%%% $\phi$ Upsilon_and_Zdistributions for ZtoUpsilon_Cat0
\begin{figure}[!htbp]
\begin{center}
\includegraphics[width=0.45\textwidth]{figures_and_tables/outputPlots/ZtoUpsilon_Cat0_ZZZZZ/au/data_x_mc/noKinCuts/h_noKin_Upsilon_phi}\hspace*{1.cm}
\includegraphics[width=0.45\textwidth]{figures_and_tables/outputPlots/ZtoUpsilon_Cat0_ZZZZZ/au/data_x_mc/noKinCuts/h_noKin_Z_phi}
\end{center}\vspace*{-.5cm}
\caption{The $\phi$ distributions for $\Upsilon(1S,2S,3S)$ in the left and for Z in the right from data and signal events for Z decaying in $\Upsilon(1S,2S,3S)$ + $\gamma$ after Group I of selection cuts. The plots are normalized to the unit of area.}
\label{fig:phiUpsilon_and_Z_ZtoUpsilon_Cat0}
\end{figure}

%%%%%%%%%%%%

%%%%%%%%%%%%%
% kin cuts
%% delta R -  mu x photon distributions for ZtoUpsilon_Cat0
\begin{figure}[!htbp]
\begin{center}
\includegraphics[width=0.45\textwidth]{figures_and_tables/outputPlots/ZtoUpsilon_Cat0_ZZZZZ/au/data_x_mc/noKinCuts/h_noKin_deltaR_Leading_Photon}\hspace*{1.cm}
\includegraphics[width=0.45\textwidth]{figures_and_tables/outputPlots/ZtoUpsilon_Cat0_ZZZZZ/au/data_x_mc/noKinCuts/h_noKin_deltaR_Trailing_Photon}\end{center}\vspace*{-.5cm}
\caption{The $\Delta R$ distributions between the photon and the leading muon (left) and the trailing muon (right) for for Z decaying in $\Upsilon(1S,2S,3S)$ + $\gamma$ from data and signal events after Group I of selection cuts. The plots are normalized to the unit of area.}
\label{fig:deltaR_ZtoUpsilon_Cat0}
\end{figure}

%% delta R and Delta Phi -  MuMu x photon distributions for ZtoUpsilon_Cat0
\begin{figure}[!htbp]
\begin{center}
\includegraphics[width=0.45\textwidth]{figures_and_tables/outputPlots/ZtoUpsilon_Cat0_ZZZZZ/au/data_x_mc/noKinCuts/h_noKin_deltaR_Upsilon_Photon}\hspace*{1.cm}
\includegraphics[width=0.45\textwidth]{figures_and_tables/outputPlots/ZtoUpsilon_Cat0_ZZZZZ/au/data_x_mc/noKinCuts/h_noKin_deltaPhi_Upsilon_Photon}\end{center}\vspace*{-.5cm}
\caption{Left: The $\Delta R$ distributions between reconstructed dimuon ($\mu\mu$) system and the photon. Right: absolute value of the $\Delta \phi$ between the leading muon and the photon for for Z decaying in $\Upsilon(1S,2S,3S)$ + $\gamma$ from data and signal events after Group I of selection cuts. The plots are normalized to the unit of area.}
\label{fig:deltaRdeltaPhi_ZtoUpsilon_Cat0}
\end{figure}


%%%%%%% energy/mass ratio distributions for ZtoUpsilon_Cat0
\begin{figure}[!htbp]
\begin{center}
\includegraphics[width=0.45\textwidth]{figures_and_tables/outputPlots/ZtoUpsilon_Cat0_ZZZZZ/au/data_x_mc/noKinCuts/h_noKin_upsilonPt_over_zMass}\hspace*{1.cm}
\includegraphics[width=0.45\textwidth]{figures_and_tables/outputPlots/ZtoUpsilon_Cat0_ZZZZZ/au/data_x_mc/noKinCuts/h_noKin_photonPt_over_zMass}
\end{center}\vspace*{-.5cm}
\caption{The ratio for the transverse momentum of the reconstructed Upsilon and the reconstructed Z mass ($p_{T}^{\mu\mu}/M_{\mu\mu\gamma}$ - left) and the ratio for the transverse energy of the reconstructed Photon and the reconstructed Z mass ($E_{T}^{\mu\mu}/M_{\mu\mu\gamma}$ - right) distribution for Z decaying in $\Upsilon(1S,2S,3S)$ + $\gamma$ from data and signal events after Group I of selection cuts. The plots are normalized to the unit of area.}
\label{fig:energy_ration_ZtoUpsilon_Cat0}
\end{figure}

%%%%%%%%% dimuon mass distributions for ZtoUpsilon_Cat0
\begin{figure}[!htbp]
\begin{center}
\includegraphics[width=0.45\textwidth]{figures_and_tables/outputPlots/ZtoUpsilon_Cat0_ZZZZZ/nEvts/data_x_mc/noKinCuts/h_noKin_Upsilon_Mass_Signal_and_Background_LargeRange}\hspace*{1.cm}
\end{center}\vspace*{-.5cm}
\caption{The dimuon mass distribution of the reconstructed $\Upsilon (1S,2S,3S)$ from data and signal events for Z decaying after Group I of selection cuts. The plot is normalized to the number of events. "Signal" stands for the $Z \rightarrow \Upsilon (1S,2S,3S) + \gamma$ sample (scaled by a factor of $\times 100$) and "Background" corresponds to the peaking background ($Z \rightarrow \mu\mu\gamma_{FSR}$) sample (scaled by a factor of x3).}
\label{fig:dimuon_mass_ZtoUpsilon_Cat0}
\end{figure}

%%%%%%%%%%%%


%CONTROL PLOTS
%%$\pT$ muon distributions for HtoUpsilon_Cat0
\begin{figure}[!htbp]
\begin{center}
\includegraphics[width=0.45\textwidth]{figures_and_tables/outputPlots/HtoUpsilon_Cat0_ZZZZZ/au/data_x_mc/noKinCuts/h_noKin_TrailingMu_pt}\hspace*{1.cm}
\includegraphics[width=0.45\textwidth]{figures_and_tables/outputPlots/HtoUpsilon_Cat0_ZZZZZ/au/data_x_mc/noKinCuts/h_noKin_LeadingMu_pt}
\end{center}\vspace*{-.5cm}
\caption{The \PT muon distributions from data and signal events for Higgs decaying in $\Upsilon(1S,2S,3S)$ + $\gamma$ after Group I of selection cuts, where on left are presenting the trailing muons and on right are the leading muons. The plots are normalized to the unit of area.}
\label{fig:pTMuons_HtoUpsilon_Cat0}
\end{figure}


%%%%%%%$\eta$ muon distributions for HtoUpsilon_Cat0
\begin{figure}[!htbp]
\begin{center}
\includegraphics[width=0.45\textwidth]{figures_and_tables/outputPlots/HtoUpsilon_Cat0_ZZZZZ/au/data_x_mc/noKinCuts/h_noKin_TrailingMu_eta}\hspace*{1.cm}
\includegraphics[width=0.45\textwidth]{figures_and_tables/outputPlots/HtoUpsilon_Cat0_ZZZZZ/au/data_x_mc/noKinCuts/h_noKin_LeadingMu_eta}
\end{center}\vspace*{-.5cm}
\caption{The $\eta$ muon distributions from data and signal events of Higgs decaying in $\Upsilon(1S,2S,3S)$ + $\gamma$ after Group I of selection cuts, where on left are presenting the trailing muons and on right are the leading muons. The plots are normalized to the unit of area.}
\label{fig:etaMuons_HtoUpsilon_Cat0}
\end{figure}

%%%%%%%%% $\phi$ muon distributions for HtoUpsilon_Cat0
\begin{figure}[!htbp]
\begin{center}
\includegraphics[width=0.45\textwidth]{figures_and_tables/outputPlots/HtoUpsilon_Cat0_ZZZZZ/au/data_x_mc/noKinCuts/h_noKin_TrailingMu_phi}\hspace*{1.cm}
\includegraphics[width=0.45\textwidth]{figures_and_tables/outputPlots/HtoUpsilon_Cat0_ZZZZZ/au/data_x_mc/noKinCuts/h_noKin_LeadingMu_phi}
\end{center}\vspace*{-.5cm}
\caption{The $\phi$ muon distributions from data and signal events of Higgs decaying in $\Upsilon(1S,2S,3S)$ + $\gamma$ after Group I of selection cuts, where on left are presenting the trailing muons and on right are the leading muons. The plots are normalized to the unit of area.}
\label{fig:phiMuons_HtoUpsilon_Cat0}
\end{figure}

%%%%%%%%%%%%%

%photon
%%$\pT$ Photon distributions for HtoUpsilon_Cat0
\begin{figure}[!htbp]
\begin{center}
\includegraphics[width=0.45\textwidth]{figures_and_tables/outputPlots/HtoUpsilon_Cat0_ZZZZZ/au/data_x_mc/noKinCuts/h_noKin_Photon_pt}\hspace*{1.cm}
\end{center}\vspace*{-.5cm}
\caption{The \PT photon distributions from data and signal events for Higgs decaying in $\Upsilon(1S,2S,3S)$ + $\gamma$ Group I of selection cuts. The plot is normalized to the unit of area.}
\label{fig:pTPhoton_HtoUpsilon_Cat0}
\end{figure}


%%%%%%%$\eta$ Photon distributions for HtoUpsilon_Cat0
\begin{figure}[!htbp]
\begin{center}
\includegraphics[width=0.45\textwidth]{figures_and_tables/outputPlots/HtoUpsilon_Cat0_ZZZZZ/au/data_x_mc/noKinCuts/h_noKin_Photon_eta}\hspace*{1.cm}
\end{center}\vspace*{-.5cm}
\caption{The $\eta$ photon distributions from data and signal events of Higgs decaying in $\Upsilon(1S,2S,3S)$ + $\gamma$ after Group I of selection cuts. The plot is normalized to the unit of area.}
\label{fig:etaPhoton_HtoUpsilon_Cat0}
\end{figure}

%%%%%%%%% $\phi$ Photon distributions for HtoUpsilon_Cat0
\begin{figure}[!htbp]
\begin{center}
\includegraphics[width=0.45\textwidth]{figures_and_tables/outputPlots/HtoUpsilon_Cat0_ZZZZZ/au/data_x_mc/noKinCuts/h_noKin_Photon_phi}\hspace*{1.cm}
\end{center}\vspace*{-.5cm}
\caption{The $\phi$ photon distributions from data and signal events of Higgs decaying in $\Upsilon(1S,2S,3S)$ + $\gamma$ after Group I of selection cuts. The plot is normalized to the unit of area.}
\label{fig:phiPhoton_HtoUpsilon_Cat0}
\end{figure}

%%%%%%%%%%%%%
% Upsilon and Higgs boson
%%$\pT$ Upsilon_and_Higgs distributions for HtoUpsilon_Cat0
\begin{figure}[!htbp]
\begin{center}
\includegraphics[width=0.45\textwidth]{figures_and_tables/outputPlots/HtoUpsilon_Cat0_ZZZZZ/au/data_x_mc/noKinCuts/h_noKin_Upsilon_Pt}\hspace*{1.cm}
\includegraphics[width=0.45\textwidth]{figures_and_tables/outputPlots/HtoUpsilon_Cat0_ZZZZZ/au/data_x_mc/noKinCuts/h_noKin_Z_Pt}
\end{center}\vspace*{-.5cm}
\caption{The \PT distributions for $\Upsilon(1S,2S,3S)$ in the left and for Higgs in the right from data and signal events for Higgs decaying in $\Upsilon(1S,2S,3S)$ + $\gamma$ after Group I of selection cuts. The plots are normalized to the unit of area.}
\label{fig:pTUpsilon_and_Higgs_HtoUpsilon_Cat0}
\end{figure}


%%%%%%%$\eta$ Upsilon_and_Higgs distributions for HtoUpsilon_Cat0
\begin{figure}[!htbp]
\begin{center}
\includegraphics[width=0.45\textwidth]{figures_and_tables/outputPlots/HtoUpsilon_Cat0_ZZZZZ/au/data_x_mc/noKinCuts/h_noKin_Upsilon_eta}\hspace*{1.cm}
\includegraphics[width=0.45\textwidth]{figures_and_tables/outputPlots/HtoUpsilon_Cat0_ZZZZZ/au/data_x_mc/noKinCuts/h_noKin_Z_eta}
\end{center}\vspace*{-.5cm}
\caption{The $\eta$ distributions for $\Upsilon(1S,2S,3S)$ in the left and for Higgs in the right from data and signal events for Higgs decaying in $\Upsilon(1S,2S,3S)$ + $\gamma$ after Group I of selection cuts. The plots are normalized to the unit of area.}
\label{fig:etaUpsilon_and_Higgs_HtoUpsilon_Cat0}
\end{figure}

%%%%%%%%% $\phi$ Upsilon_and_Higgsdistributions for HtoUpsilon_Cat0
\begin{figure}[!htbp]
\begin{center}
\includegraphics[width=0.45\textwidth]{figures_and_tables/outputPlots/HtoUpsilon_Cat0_ZZZZZ/au/data_x_mc/noKinCuts/h_noKin_Upsilon_phi}\hspace*{1.cm}
\includegraphics[width=0.45\textwidth]{figures_and_tables/outputPlots/HtoUpsilon_Cat0_ZZZZZ/au/data_x_mc/noKinCuts/h_noKin_Z_phi}
\end{center}\vspace*{-.5cm}
\caption{The $\phi$ distributions for $\Upsilon(1S,2S,3S)$ in the left and for Higgs in the right from data and signal events for Higgs decaying in $\Upsilon(1S,2S,3S)$ + $\gamma$ after Group I of selection cuts. The plots are normalized to the unit of area.}
\label{fig:phiUpsilon_and_Higgs_HtoUpsilon_Cat0}
\end{figure}

%%%%%%%%%%%%

%%%%%%%%%%%%%
% kin cuts
%% delta R -  mu x photon distributions for HtoUpsilon_Cat0
\begin{figure}[!htbp]
\begin{center}
\includegraphics[width=0.45\textwidth]{figures_and_tables/outputPlots/HtoUpsilon_Cat0_ZZZZZ/au/data_x_mc/noKinCuts/h_noKin_deltaR_Leading_Photon}\hspace*{1.cm}
\includegraphics[width=0.45\textwidth]{figures_and_tables/outputPlots/HtoUpsilon_Cat0_ZZZZZ/au/data_x_mc/noKinCuts/h_noKin_deltaR_Trailing_Photon}\end{center}\vspace*{-.5cm}
\caption{The $\Delta R$ distributions between the photon and the leading muon (left) and the trailing muon (right) for Higgs decaying in $\Upsilon(1S,2S,3S)$ + $\gamma$ from data and signal events after Group I of selection cuts. The plots are normalized to the unit of area.}
\label{fig:deltaR_HtoUpsilon_Cat0}
\end{figure}

%% delta R and Delta Phi -  MuMu x photon distributions for HtoUpsilon_Cat0
\begin{figure}[!htbp]
\begin{center}
\includegraphics[width=0.45\textwidth]{figures_and_tables/outputPlots/HtoUpsilon_Cat0_ZZZZZ/au/data_x_mc/noKinCuts/h_noKin_deltaR_Upsilon_Photon}\hspace*{1.cm}
\includegraphics[width=0.45\textwidth]{figures_and_tables/outputPlots/HtoUpsilon_Cat0_ZZZZZ/au/data_x_mc/noKinCuts/h_noKin_deltaPhi_Upsilon_Photon}\end{center}\vspace*{-.5cm}
\caption{Left: The $\Delta R$ distributions between reconstructed dimuon ($\mu\mu$) system and the photon. Right: absolute value of the $\Delta \phi$ between the leading muon and the photon for for Higgs decaying in $\Upsilon(1S,2S,3S)$ + $\gamma$ from data and signal events after Group I of selection cuts. The plots are normalized to the unit of area.}
\label{fig:deltaRdeltaPhi_ZtoUpsilon_Cat0}
\end{figure}

%%%%%%% energy/mass ratio distributions for HtoUpsilon_Cat0
\begin{figure}[!htbp]
\begin{center}
\includegraphics[width=0.45\textwidth]{figures_and_tables/outputPlots/HtoUpsilon_Cat0_ZZZZZ/au/data_x_mc/noKinCuts/h_noKin_upsilonPt_over_zMass}\hspace*{1.cm}
\includegraphics[width=0.45\textwidth]{figures_and_tables/outputPlots/HtoUpsilon_Cat0_ZZZZZ/au/data_x_mc/noKinCuts/h_noKin_photonPt_over_zMass}
\end{center}\vspace*{-.5cm}
\caption{The ratio for the transverse momentum of the reconstructed Upsilon and the reconstructed Higgs mass ($p_{T}^{\mu\mu}/M_{\mu\mu\gamma}$ - left) and the ratio for the transverse energy of the reconstructed Photon and the reconstructed Higgs mass ($E_{T}^{\mu\mu}/M_{\mu\mu\gamma}$ - right) distribution for Higgs decaying in $\Upsilon(1S,2S,3S)$ + $\gamma$ from data and signal events after Group I of selection cuts. The plots are normalized to the unit of area.}
\label{fig:energy_ration_HtoUpsilon_Cat0}
\end{figure}

%%%%%%%%% dimuon mass distributions for HtoUpsilon_Cat0
\begin{figure}[!htbp]
\begin{center}
\includegraphics[width=0.45\textwidth]{figures_and_tables/outputPlots/HtoUpsilon_Cat0_ZZZZZ/nEvts/data_x_mc/noKinCuts/h_noKin_Upsilon_Mass_Signal_and_Background_LargeRange}\hspace*{1.cm}
\end{center}\vspace*{-.5cm}
\caption{The dimuon mass distribution of the reconstructed $\Upsilon (1S,2S,3S)$ from data and signal events for Higgs decaying after Group I of selection cuts. This plot is normalized the expected number of events.  The plot is normalized to the number of events. "Signal" stands for the $H \rightarrow \Upsilon (1S,2S,3S) + \gamma$ sample (scaled by a factor of $\times 60000$) and "Background" corresponds to the peaking background (Higgs Dalitz Decay) sample (scaled by a factor of $\times 400$).}
\label{fig:dimuon_mass_HtoUpsilon_Cat0}
\end{figure}

%%%%%%%%%%%%%%%%%%%%%%%%%%%%%%%%%%%%%%%%%%%%%%%%%%%%%%%%%%%%%%%%%%%%%%%%%%%%%%%%%%%%%%%%%%%%%%%%%%%%%%%%%%%%%%%%%%%%%%%%
%%%%%%%%%%%%%%%%%%%%%%%%%%%%%%%%%%%%%%%%%%%%%%%%%%%%%%%%%%%%%%%%%%%%%%%%%%%%%%%%%%%%%%%%%%%%%%%%%%%%%%%%%%%%%%%%%%%%%%%%
%%%%%%%%%%%%%%%%%%%%%%%%%%%%%%%%%%%%%%%%%%%%%%%%%%%%%%%%%%%%%%%%%%%%%%%%%%%%%%%%%%%%%%%%%%%%%%%%%%%%%%%%%%%%%%%%%%%%%%%%
%%%%%%%%%%%%%%%%%%%%%%%%%%%%%%%%%%%%%%%%%%%%%%%%%%%%%%%%%%%%%%%%%%%%%%%%%%%%%%%%%%%%%%%%%%%%%%%%%%%%%%%%%%%%%%%%%%%%%%%%
%%%%%%%%%%%%%%%%%%%%%%%%%%%%%%%%%%%%%%%%%%%%%%%%%%%%%%%%%%%%%%%%%%%%%%%%%%%%%%%%%%%%%%%%%%%%%%%%%%%%%%%%%%%%%%%%%%%%%%%%
%%%%%%%%%%%%%%%%%%%%%%%%%%%%%%%%%%%%%%%%%%%%%%%%%%%%%%%%%%%%%%%%%%%%%%%%%%%%%%%%%%%%%%%%%%%%%%%%%%%%%%%%%%%%%%%%%%%%%%%%
%%%%%%%%%%%%%%%%%%%%%%%%%%%%%%%%%%%%%%%%%%%%%%%%%%%%%%%%%%%%%%%%%%%%%%%%%%%%%%%%%%%%%%%%%%%%%%%%%%%%%%%%%%%%%%%%%%%%%%%%
%%%%%%%%%%%%%%%%%%%%%%%%%%%%%%%%%%%%%%%%%%%%%%%%%%%%%%%%%%%%%%%%%%%%%%%%%%%%%%%%%%%%%%%%%%%%%%%%%%%%%%%%%%%%%%%%%%%%%%%%
%%%%%%%%%%%%%%%%%%%%%%%%%%%%%%%%%%%%%%%%%%%%%%%%%%%%%%%%%%%%%%%%%%%%%%%%%%%%%%%%%%%%%%%%%%%%%%%%%%%%%%%%%%%%%%%%%%%%%%%%
%%%%%%%%%%%%%%%%%%%%%%%%%%%%%%%%%%%%%%%%%%%%%%%%%%%%%%%%%%%%%%%%%%%%%%%%%%%%%%%%%%%%%%%%%%%%%%%%%%%%%%%%%%%%%%%%%%%%%%%%
%%%%%%%%%%%%%%%%%%%%%%%%%%%%%%%%%%%%%%%%%%%%%%%%%%%%%%%%%%%%%%%%%%%%%%%%%%%%%%%%%%%%%%%%%%%%%%%%%%%%%%%%%%%%%%%%%%%%%%%%
%%%%%%%%%%%%%%%%%%%%%%%%%%%%%%%%%%%%%%%%%%%%%%%%%%%%%%%%%%%%%%%%%%%%%%%%%%%%%%%%%%%%%%%%%%%%%%%%%%%%%%%%%%%%%%%%%%%%%%%%
%%%%%%%%%%%%%%%%%%%%%%%%%%%%%%%%%%%%%%%%%%%%%%%%%%%%%%%%%%%%%%%%%%%%%%%%%%%%%%%%%%%%%%%%%%%%%%%%%%%%%%%%%%%%%%%%%%%%%%%%
%
%
% _______  ______    _______  __   __  _______    ___   ___  
%|       ||    _ |  |       ||  | |  ||       |  |   | |   | 
%|    ___||   | ||  |   _   ||  | |  ||    _  |  |   | |   | 
%|   | __ |   |_||_ |  | |  ||  |_|  ||   |_| |  |   | |   | 
%|   ||  ||    __  ||  |_|  ||       ||    ___|  |   | |   | 
%|   |_| ||   |  | ||       ||       ||   |      |   | |   | 
%|_______||___|  |_||_______||_______||___|      |___| |___| 
%
%%%%%%%%%%%%%%%%%%%%%%%%%%%%%%%%%%%%%%%%%%%%%%%%%%%%%%%%%%%%%%%%%%%%%%%%%%%%%%%%%%%%%%%%%%%%%%%%%%%%%%%%%%%%%%%%%%%%%%%%
%%%%%%%%%%%%%%%%%%%%%%%%%%%%%%%%%%%%%%%%%%%%%%%%%%%%%%%%%%%%%%%%%%%%%%%%%%%%%%%%%%%%%%%%%%%%%%%%%%%%%%%%%%%%%%%%%%%%%%%%
%%%%%%%%%%%%%%%%%%%%%%%%%%%%%%%%%%%%%%%%%%%%%%%%%%%%%%%%%%%%%%%%%%%%%%%%%%%%%%%%%%%%%%%%%%%%%%%%%%%%%%%%%%%%%%%%%%%%%%%%
%%%%%%%%%%%%%%%%%%%%%%%%%%%%%%%%%%%%%%%%%%%%%%%%%%%%%%%%%%%%%%%%%%%%%%%%%%%%%%%%%%%%%%%%%%%%%%%%%%%%%%%%%%%%%%%%%%%%%%%%
%%%%%%%%%%%%%%%%%%%%%%%%%%%%%%%%%%%%%%%%%%%%%%%%%%%%%%%%%%%%%%%%%%%%%%%%%%%%%%%%%%%%%%%%%%%%%%%%%%%%%%%%%%%%%%%%%%%%%%%%
%%%%%%%%%%%%%%%%%%%%%%%%%%%%%%%%%%%%%%%%%%%%%%%%%%%%%%%%%%%%%%%%%%%%%%%%%%%%%%%%%%%%%%%%%%%%%%%%%%%%%%%%%%%%%%%%%%%%%%%%
%%%%%%%%%%%%%%%%%%%%%%%%%%%%%%%%%%%%%%%%%%%%%%%%%%%%%%%%%%%%%%%%%%%%%%%%%%%%%%%%%%%%%%%%%%%%%%%%%%%%%%%%%%%%%%%%%%%%%%%%
%%%%%%%%%%%%%%%%%%%%%%%%%%%%%%%%%%%%%%%%%%%%%%%%%%%%%%%%%%%%%%%%%%%%%%%%%%%%%%%%%%%%%%%%%%%%%%%%%%%%%%%%%%%%%%%%%%%%%%%%
%%%%%%%%%%%%%%%%%%%%%%%%%%%%%%%%%%%%%%%%%%%%%%%%%%%%%%%%%%%%%%%%%%%%%%%%%%%%%%%%%%%%%%%%%%%%%%%%%%%%%%%%%%%%%%%%%%%%%%%%
%%%%%%%%%%%%%%%%%%%%%%%%%%%%%%%%%%%%%%%%%%%%%%%%%%%%%%%%%%%%%%%%%%%%%%%%%%%%%%%%%%%%%%%%%%%%%%%%%%%%%%%%%%%%%%%%%%%%%%%%
%%%%%%%%%%%%%%%%%%%%%%%%%%%%%%%%%%%%%%%%%%%%%%%%%%%%%%%%%%%%%%%%%%%%%%%%%%%%%%%%%%%%%%%%%%%%%%%%%%%%%%%%%%%%%%%%%%%%%%%%
%%%%%%%%%%%%%%%%%%%%%%%%%%%%%%%%%%%%%%%%%%%%%%%%%%%%%%%%%%%%%%%%%%%%%%%%%%%%%%%%%%%%%%%%%%%%%%%%%%%%%%%%%%%%%%%%%%%%%%%%
%%%%%%%%%%%%%%%%%%%%%%%%%%%%%%%%%%%%%%%%%%%%%%%%%%%%%%%%%%%%%%%%%%%%%%%%%%%%%%%%%%%%%%%%%%%%%%%%%%%%%%%%%%%%%%%%%%%%%%%%
%%%%%%%%%%%%%%%%%%%%%%%%%%%%%%%%%%%%%%%%%%%%%%%%%%%%%%%%%%%%%%%%%%%%%%%%%%%%%%%%%%%%%%%%%%%%%%%%%%%%%%%%%%%%%%%%%%%%%%%%
%%%%%%%%%%%%%%%%%%%%%%%%%%%%%%%%%%%%%%%%%%%%%%%%%%%%%%%%%%%%%%%%%%%%%%%%%%%%%%%%%%%%%%%%%%%%%%%%%%%%%%%%%%%%%%%%%%%%%%%%

\clearpage

\section{Kinematical selection (Group II)}


After all Trigger and Object Identification cuts, described in before (\textbf{Group I}), a set of kinematical cuts are applied in order to improve the signal to background relation. They are

\begin{itemize}
  \item $\Delta R(\text{leading }\mu, \gamma) > 1$;
  \item $\Delta R(\text{trailing }\mu, \gamma) > 1$;
  \item $\Delta R(\mu\mu, \gamma) > 2$;
  \item $|\Delta \phi (\text{leading }\mu, \gamma)| > 1.5$;
  \item 8.4 GeV $<$ $M_{\mu\mu}$ $<$ 11.1 GeV;
  \item $E_{T}^{\gamma}/M_{\mu\mu\gamma} > 35/91.2 \text{ for the Z decay or } 35/125 \text{ for the Higgs decay} $;
  \item $p_{T}^{\mu\mu}/M_{\mu\mu\gamma} > 35/91.2 \text{ for the Z decay or } 35/125 \text{ for the Higgs decay} $.
\end{itemize}

The choice of these thresholds were based on the visual inspection of the distributions (besides the invariant mass distribution of the dimuon system $M_{\mu\mu}$, which needs to be defined around the $\upsilon(1S, 2S, 3S)$ mass) and to keep this analysis in phase with other similar analysis within CMS.

% A detailed discussion about the choices on the thresholds can be found in the $H/Z \rightarrow J/\psi + \gamma$ analysis \cite{CMS_jpsi_analysis}. In any case, besides the dimuon mass window, which obviously should match with the $\Upsilon(1S,2S,3S)$ mass, the values were not changed, with respect to the reference analysis. Even though some optimization was tried, it brought no reasonable gain that would justify change the thresholds and lose compatibility with the $J/\psi$ study.

Below it is shown the same set of plot shown before, but this time, taking into account the full selection (\textbf{Group I+II}).


%CONTROL PLOTS
%%$\pT$ muon distributions for ZtoUpsilon_Cat0
\begin{figure}[!htbp]
\begin{center}
\includegraphics[width=0.45\textwidth]{figures_and_tables/outputPlots/ZtoUpsilon_Cat0_ZZZZZ/nEvts/data_x_mc/withKinCuts/h_withKin_TrailingMu_pt}\hspace*{1.cm}
\includegraphics[width=0.45\textwidth]{figures_and_tables/outputPlots/ZtoUpsilon_Cat0_ZZZZZ/nEvts/data_x_mc/withKinCuts/h_withKin_LeadingMu_pt}
\end{center}\vspace*{-.5cm}
\caption{The \PT muon distributions from data and signal events for Z decaying in $\Upsilon(1S,2S,3S)$ + $\gamma$ after Group I of selection cuts, where on left are presenting the trailing muons and on right are the leading muons. The plots are normalized to the number of events. Signal sample is scaled by a factor of $\times 100$).}
\label{fig:pTMuons_ZtoUpsilon_Cat0_groupI_plus_II}
\end{figure}


%%%%%%%$\eta$ muon distributions for ZtoUpsilon_Cat0
\begin{figure}[!htbp]
\begin{center}
\includegraphics[width=0.45\textwidth]{figures_and_tables/outputPlots/ZtoUpsilon_Cat0_ZZZZZ/nEvts/data_x_mc/withKinCuts/h_withKin_TrailingMu_eta}\hspace*{1.cm}
\includegraphics[width=0.45\textwidth]{figures_and_tables/outputPlots/ZtoUpsilon_Cat0_ZZZZZ/nEvts/data_x_mc/withKinCuts/h_withKin_LeadingMu_eta}
\end{center}\vspace*{-.5cm}
\caption{The $\eta$ muon distributions from data and signal events of Z decaying in $\Upsilon(1S,2S,3S)$ + $\gamma$ after Group I of selection cuts, where on left are presenting the trailing muons and on right are the leading muons. The plots are normalized to the number of events. Signal sample is scaled by a factor of $\times 100$).}
\label{fig:etaMuons_ZtoUpsilon_Cat0_groupI_plus_II}
\end{figure}

%%%%%%%%% $\phi$ muon distributions for ZtoUpsilon_Cat0
\begin{figure}[!htbp]
\begin{center}
\includegraphics[width=0.45\textwidth]{figures_and_tables/outputPlots/ZtoUpsilon_Cat0_ZZZZZ/nEvts/data_x_mc/withKinCuts/h_withKin_TrailingMu_phi}\hspace*{1.cm}
\includegraphics[width=0.45\textwidth]{figures_and_tables/outputPlots/ZtoUpsilon_Cat0_ZZZZZ/nEvts/data_x_mc/withKinCuts/h_withKin_LeadingMu_phi}
\end{center}\vspace*{-.5cm}
\caption{The $\phi$ muon distributions from data and signal events of Z decaying in $\Upsilon(1S,2S,3S)$ + $\gamma$ after Group I of selection cuts, where on left are presenting the trailing muons and on right are the leading muons. The plots are normalized to the number of events. Signal sample is scaled by a factor of $\times 100$).}
\label{fig:phiMuons_ZtoUpsilon_Cat0_groupI_plus_II}
\end{figure}

%%%%%%%%%%%%%

%photon
%%$\pT$ Photon distributions for ZtoUpsilon_Cat0
\begin{figure}[!htbp]
\begin{center}
\includegraphics[width=0.45\textwidth]{figures_and_tables/outputPlots/ZtoUpsilon_Cat0_ZZZZZ/nEvts/data_x_mc/withKinCuts/h_withKin_Photon_pt}\hspace*{1.cm}
\end{center}\vspace*{-.5cm}
\caption{The \PT photon distributions from data and signal events for Z decaying in $\Upsilon(1S,2S,3S)$ + $\gamma$ all (Group I+II) selection cuts. The plot is normalized to the number of events. Signal sample is scaled by a factor of $\times 100$).}
\label{fig:pTPhoton_ZtoUpsilon_Cat0_groupI_plus_II}
\end{figure}


%%%%%%%$\eta$ Photon distributions for ZtoUpsilon_Cat0
\begin{figure}[!htbp]
\begin{center}
\includegraphics[width=0.45\textwidth]{figures_and_tables/outputPlots/ZtoUpsilon_Cat0_ZZZZZ/nEvts/data_x_mc/withKinCuts/h_withKin_Photon_eta}\hspace*{1.cm}
\end{center}\vspace*{-.5cm}
\caption{The $\eta$ photon distributions from data and signal events of Z decaying in $\Upsilon(1S,2S,3S)$ + $\gamma$ after all (Group I+II) selection cuts. The plot is normalized to the number of events. Signal sample is scaled by a factor of $\times 100$).}
\label{fig:etaPhoton_ZtoUpsilon_Cat0_groupI_plus_II}
\end{figure}

%%%%%%%%% $\phi$ Photon distributions for ZtoUpsilon_Cat0
\begin{figure}[!htbp]
\begin{center}
\includegraphics[width=0.45\textwidth]{figures_and_tables/outputPlots/ZtoUpsilon_Cat0_ZZZZZ/nEvts/data_x_mc/withKinCuts/h_withKin_Photon_phi}\hspace*{1.cm}
\end{center}\vspace*{-.5cm}
\caption{The $\phi$ photon distributions from data and signal events of Z decaying in $\Upsilon(1S,2S,3S)$ + $\gamma$ after all (Group I+II) selection cuts. The plot is normalized to the number of events. Signal sample is scaled by a factor of $\times 100$).}
\label{fig:phiPhoton_ZtoUpsilon_Cat0_groupI_plus_II}
\end{figure}

%%%%%%%%%%%%%
% Upsilon and Z boson
%%$\pT$ Upsilon_and_Higgs distributions for ZtoUpsilon_Cat0
\begin{figure}[!htbp]
\begin{center}
\includegraphics[width=0.45\textwidth]{figures_and_tables/outputPlots/ZtoUpsilon_Cat0_ZZZZZ/nEvts/data_x_mc/withKinCuts/h_withKin_Upsilon_Pt}\hspace*{1.cm}
\includegraphics[width=0.45\textwidth]{figures_and_tables/outputPlots/ZtoUpsilon_Cat0_ZZZZZ/nEvts/data_x_mc/withKinCuts/h_withKin_Z_Pt}
\end{center}\vspace*{-.5cm}
\caption{The \PT distributions for $\Upsilon(1S,2S,3S)$ in the left and for Z in the right from data and signal events for Z decaying in $\Upsilon(1S,2S,3S)$ + $\gamma$ after all (Group I+II) selection cuts. The plots are normalized to the number of events. Signal sample is scaled by a factor of $\times 100$).}
\label{fig:pTUpsilon_and_Z_ZtoUpsilon_Cat0_groupI_plus_II}
\end{figure}


%%%%%%%$\eta$ Upsilon_and_Z distributions for ZtoUpsilon_Cat0
\begin{figure}[!htbp]
\begin{center}
\includegraphics[width=0.45\textwidth]{figures_and_tables/outputPlots/ZtoUpsilon_Cat0_ZZZZZ/nEvts/data_x_mc/withKinCuts/h_withKin_Upsilon_eta}\hspace*{1.cm}
\includegraphics[width=0.45\textwidth]{figures_and_tables/outputPlots/ZtoUpsilon_Cat0_ZZZZZ/nEvts/data_x_mc/withKinCuts/h_withKin_Z_eta}
\end{center}\vspace*{-.5cm}
\caption{The $\eta$ distributions for $\Upsilon(1S,2S,3S)$ in the left and for Z in the right from data and signal events for Z decaying in $\Upsilon(1S,2S,3S)$ + $\gamma$ after all (Group I+II) selection cuts. The plots are normalized to the number of events. Signal sample is scaled by a factor of $\times 100$).}
\label{fig:etaUpsilon_and_Z_ZtoUpsilon_Cat0_groupI_plus_II}
\end{figure}

%%%%%%%%% $\phi$ Upsilon_and_Zdistributions for ZtoUpsilon_Cat0
\begin{figure}[!htbp]
\begin{center}
\includegraphics[width=0.45\textwidth]{figures_and_tables/outputPlots/ZtoUpsilon_Cat0_ZZZZZ/nEvts/data_x_mc/withKinCuts/h_withKin_Upsilon_phi}\hspace*{1.cm}
\includegraphics[width=0.45\textwidth]{figures_and_tables/outputPlots/ZtoUpsilon_Cat0_ZZZZZ/nEvts/data_x_mc/withKinCuts/h_withKin_Z_phi}
\end{center}\vspace*{-.5cm}
\caption{The $\phi$ distributions for $\Upsilon(1S,2S,3S)$ in the left and for Z in the right from data and signal events for Z decaying in $\Upsilon(1S,2S,3S)$ + $\gamma$ after all (Group I+II) selection cuts. The plots are normalized to the number of events. Signal sample is scaled by a factor of $\times 100$).}
\label{fig:phiUpsilon_and_Z_ZtoUpsilon_Cat0_groupI_plus_II}
\end{figure}

%%%%%%%%%%%%

%%%%%%%%%%%%%
% kin cuts
%% delta R -  mu x photon distributions for ZtoUpsilon_Cat0
\begin{figure}[!htbp]
\begin{center}
\includegraphics[width=0.45\textwidth]{figures_and_tables/outputPlots/ZtoUpsilon_Cat0_ZZZZZ/nEvts/data_x_mc/withKinCuts/h_withKin_deltaR_Leading_Photon}\hspace*{1.cm}
\includegraphics[width=0.45\textwidth]{figures_and_tables/outputPlots/ZtoUpsilon_Cat0_ZZZZZ/nEvts/data_x_mc/withKinCuts/h_withKin_deltaR_Trailing_Photon}\end{center}\vspace*{-.5cm}
\caption{The $\Delta R$ distributions between the photon and the leading muon (left) and the trailing muon (right) for $\Upsilon(1S,2S,3S)$ from data and signal events for Z decaying after all (Group I+II) selection cuts. The plots are normalized to the number of events. Signal sample is scaled by a factor of $\times 100$).}
\label{fig:deltaR_ZtoUpsilon_Cat0_groupI_plus_II}
\end{figure}


%%%%%%% energy/mass ratio distributions for ZtoUpsilon_Cat0
\begin{figure}[!htbp]
\begin{center}
\includegraphics[width=0.45\textwidth]{figures_and_tables/outputPlots/ZtoUpsilon_Cat0_ZZZZZ/nEvts/data_x_mc/withKinCuts/h_withKin_upsilonPt_over_zMass}\hspace*{1.cm}
\includegraphics[width=0.45\textwidth]{figures_and_tables/outputPlots/ZtoUpsilon_Cat0_ZZZZZ/nEvts/data_x_mc/withKinCuts/h_withKin_photonPt_over_zMass}
\end{center}\vspace*{-.5cm}
\caption{The ratio for the transverse momentum of the reconstructed Upsilon and the reconstructed Z mass ($p_{T}^{\mu\mu}/M_{\mu\mu\gamma}$ - left) and the ratio for the transverse energy of the reconstructed Photon and the reconstructed Z mass ($E_{T}^{\mu\mu}/M_{\mu\mu\gamma}$ - right) distribution for $\Upsilon(1S,2S,3S)$ from data and signal events for Z decaying after all (Group I+II) selection cuts. The plots are normalized to the number of events. Signal sample is scaled by a factor of $\times 100$).}
\label{fig:energy_ration_ZtoUpsilon_Cat0_groupI_plus_II}
\end{figure}

%%%%%%%%%% dimuon mass distributions for ZtoUpsilon_Cat0
%\begin{figure}[!htbp]
%\begin{center}
%\includegraphics[width=0.45\textwidth]{figures_and_tables/outputPlots/ZtoUpsilon_Cat0_ZZZZZ/nEvts/data_x_mc/withKinCuts/h_withKin_Upsilon_Mass_Signal_and_Background_LargeRange}\hspace*{1.cm}
%\end{center}\vspace*{-.5cm}
%\caption{The dimuon mass distribution of the reconstructed $\Upsilon (1S,2S,3S)$ from data and signal events for Z decaying after all (Group I+II) selection cuts. The plot is normalized to the number of events. Signal sample is scaled by a factor of $\times 100$).}
%\label{fig:dimuon_mass_ZtoUpsilon_Cat0_groupI_plus_II}
%\end{figure}

%%%%%%%%%%%%



%CONTROL PLOTS
%%$\pT$ muon distributions for HtoUpsilon_Cat0
\begin{figure}[!htbp]
\begin{center}
\includegraphics[width=0.45\textwidth]{figures_and_tables/outputPlots/HtoUpsilon_Cat0_ZZZZZ/nEvts/data_x_mc/withKinCuts/h_withKin_TrailingMu_pt}\hspace*{1.cm}
\includegraphics[width=0.45\textwidth]{figures_and_tables/outputPlots/HtoUpsilon_Cat0_ZZZZZ/nEvts/data_x_mc/withKinCuts/h_withKin_LeadingMu_pt}
\end{center}\vspace*{-.5cm}
\caption{The \PT muon distributions from data and signal events for Higgs decaying in $\Upsilon(1S,2S,3S)$ + $\gamma$ after Group I of selection cuts, where on left are presenting the trailing muons and on right are the leading muons. The plots are normalized to the number of events. Signal sample is scaled by a factor of $\times 600000$).}
\label{fig:pTMuons_HtoUpsilon_Cat0_groupI_plus_II}
\end{figure}


%%%%%%%$\eta$ muon distributions for HtoUpsilon_Cat0
\begin{figure}[!htbp]
\begin{center}
\includegraphics[width=0.45\textwidth]{figures_and_tables/outputPlots/HtoUpsilon_Cat0_ZZZZZ/nEvts/data_x_mc/withKinCuts/h_withKin_TrailingMu_eta}\hspace*{1.cm}
\includegraphics[width=0.45\textwidth]{figures_and_tables/outputPlots/HtoUpsilon_Cat0_ZZZZZ/nEvts/data_x_mc/withKinCuts/h_withKin_LeadingMu_eta}
\end{center}\vspace*{-.5cm}
\caption{The $\eta$ muon distributions from data and signal events of Higgs decaying in $\Upsilon(1S,2S,3S)$ + $\gamma$ after Group I of selection cuts, where on left are presenting the trailing muons and on right are the leading muons. The plots are normalized to the number of events. Signal sample is scaled by a factor of $\times 600000$).}
\label{fig:etaMuons_HtoUpsilon_Cat0_groupI_plus_II}
\end{figure}

%%%%%%%%% $\phi$ muon distributions for HtoUpsilon_Cat0
\begin{figure}[!htbp]
\begin{center}
\includegraphics[width=0.45\textwidth]{figures_and_tables/outputPlots/HtoUpsilon_Cat0_ZZZZZ/nEvts/data_x_mc/withKinCuts/h_withKin_TrailingMu_phi}\hspace*{1.cm}
\includegraphics[width=0.45\textwidth]{figures_and_tables/outputPlots/HtoUpsilon_Cat0_ZZZZZ/nEvts/data_x_mc/withKinCuts/h_withKin_LeadingMu_phi}
\end{center}\vspace*{-.5cm}
\caption{The $\phi$ muon distributions from data and signal events of Higgs decaying in $\Upsilon(1S,2S,3S)$ + $\gamma$ after Group I of selection cuts, where on left are presenting the trailing muons and on right are the leading muons. The plots are normalized to the number of events. Signal sample is scaled by a factor of $\times 600000$).}
\label{fig:phiMuons_HtoUpsilon_Cat0_groupI_plus_II}
\end{figure}

%%%%%%%%%%%%%

%photon
%%$\pT$ Photon distributions for HtoUpsilon_Cat0
\begin{figure}[!htbp]
\begin{center}
\includegraphics[width=0.45\textwidth]{figures_and_tables/outputPlots/HtoUpsilon_Cat0_ZZZZZ/nEvts/data_x_mc/withKinCuts/h_withKin_Photon_pt}\hspace*{1.cm}
\end{center}\vspace*{-.5cm}
\caption{The \PT photon distributions from data and signal events for Higgs decaying in $\Upsilon(1S,2S,3S)$ + $\gamma$ all (Group I+II) selection cuts. The plot is normalized to the number of events. Signal sample is scaled by a factor of $\times 600000$).}
\label{fig:pTPhoton_HtoUpsilon_Cat0_groupI_plus_II}
\end{figure}


%%%%%%%$\eta$ Photon distributions for HtoUpsilon_Cat0
\begin{figure}[!htbp]
\begin{center}
\includegraphics[width=0.45\textwidth]{figures_and_tables/outputPlots/HtoUpsilon_Cat0_ZZZZZ/nEvts/data_x_mc/withKinCuts/h_withKin_Photon_eta}\hspace*{1.cm}
\end{center}\vspace*{-.5cm}
\caption{The $\eta$ photon distributions from data and signal events of Higgs decaying in $\Upsilon(1S,2S,3S)$ + $\gamma$ after all (Group I+II) selection cuts. The plot is normalized to the number of events. Signal sample is scaled by a factor of $\times 600000$).}
\label{fig:etaPhoton_HtoUpsilon_Cat0_groupI_plus_II}
\end{figure}

%%%%%%%%% $\phi$ Photon distributions for HtoUpsilon_Cat0
\begin{figure}[!htbp]
\begin{center}
\includegraphics[width=0.45\textwidth]{figures_and_tables/outputPlots/HtoUpsilon_Cat0_ZZZZZ/nEvts/data_x_mc/withKinCuts/h_withKin_Photon_phi}\hspace*{1.cm}
\end{center}\vspace*{-.5cm}
\caption{The $\phi$ photon distributions from data and signal events of Higgs decaying in $\Upsilon(1S,2S,3S)$ + $\gamma$ after all (Group I+II) selection cuts. The plot is normalized to the number of events. Signal sample is scaled by a factor of c).}
\label{fig:phiPhoton_HtoUpsilon_Cat0_groupI_plus_II}
\end{figure}

%%%%%%%%%%%%%
% Upsilon and Higgs boson
%%$\pT$ Upsilon_and_Higgs distributions for HtoUpsilon_Cat0
\begin{figure}[!htbp]
\begin{center}
\includegraphics[width=0.45\textwidth]{figures_and_tables/outputPlots/HtoUpsilon_Cat0_ZZZZZ/nEvts/data_x_mc/withKinCuts/h_withKin_Upsilon_Pt}\hspace*{1.cm}
\includegraphics[width=0.45\textwidth]{figures_and_tables/outputPlots/HtoUpsilon_Cat0_ZZZZZ/nEvts/data_x_mc/withKinCuts/h_withKin_Z_Pt}
\end{center}\vspace*{-.5cm}
\caption{The \PT distributions for $\Upsilon(1S,2S,3S)$ in the left and for Higgs in the right from data and signal events for Higgs decaying in $\Upsilon(1S,2S,3S)$ + $\gamma$ after all (Group I+II) selection cuts. The plots are normalized to the number of events. Signal sample is scaled by a factor of $\times 600000$).}
\label{fig:pTUpsilon_and_Higgs_HtoUpsilon_Cat0_groupI_plus_II}
\end{figure}


%%%%%%%$\eta$ Upsilon_and_Higgs distributions for HtoUpsilon_Cat0
\begin{figure}[!htbp]
\begin{center}
\includegraphics[width=0.45\textwidth]{figures_and_tables/outputPlots/HtoUpsilon_Cat0_ZZZZZ/nEvts/data_x_mc/withKinCuts/h_withKin_Upsilon_eta}\hspace*{1.cm}
\includegraphics[width=0.45\textwidth]{figures_and_tables/outputPlots/HtoUpsilon_Cat0_ZZZZZ/nEvts/data_x_mc/withKinCuts/h_withKin_Z_eta}
\end{center}\vspace*{-.5cm}
\caption{The $\eta$ distributions for $\Upsilon(1S,2S,3S)$ in the left and for Higgs in the right from data and signal events for Higgs decaying in $\Upsilon(1S,2S,3S)$ + $\gamma$ after all (Group I+II) selection cuts. The plots are normalized to the number of events. Signal sample is scaled by a factor of $\times 600000$).}
\label{fig:etaUpsilon_and_Higgs_HtoUpsilon_Cat0_groupI_plus_II}
\end{figure}

%%%%%%%%% $\phi$ Upsilon_and_Higgsdistributions for HtoUpsilon_Cat0
\begin{figure}[!htbp]
\begin{center}
\includegraphics[width=0.45\textwidth]{figures_and_tables/outputPlots/HtoUpsilon_Cat0_ZZZZZ/nEvts/data_x_mc/withKinCuts/h_withKin_Upsilon_phi}\hspace*{1.cm}
\includegraphics[width=0.45\textwidth]{figures_and_tables/outputPlots/HtoUpsilon_Cat0_ZZZZZ/nEvts/data_x_mc/withKinCuts/h_withKin_Z_phi}
\end{center}\vspace*{-.5cm}
\caption{The $\phi$ distributions for $\Upsilon(1S,2S,3S)$ in the left and for Higgs in the right from data and signal events for Higgs decaying in $\Upsilon(1S,2S,3S)$ + $\gamma$ after all (Group I+II) selection cuts. The plots are normalized to the number of events. Signal sample is scaled by a factor of $\times 600000$).}
\label{fig:phiUpsilon_and_Higgs_HtoUpsilon_Cat0_groupI_plus_II}
\end{figure}

%%%%%%%%%%%%

%%%%%%%%%%%%%
% kin cuts
%% delta R -  mu x photon distributions for HtoUpsilon_Cat0
\begin{figure}[!htbp]
\begin{center}
\includegraphics[width=0.45\textwidth]{figures_and_tables/outputPlots/HtoUpsilon_Cat0_ZZZZZ/nEvts/data_x_mc/withKinCuts/h_withKin_deltaR_Leading_Photon}\hspace*{1.cm}
\includegraphics[width=0.45\textwidth]{figures_and_tables/outputPlots/HtoUpsilon_Cat0_ZZZZZ/nEvts/data_x_mc/withKinCuts/h_withKin_deltaR_Trailing_Photon}\end{center}\vspace*{-.5cm}
\caption{The $\Delta R$ distributions between the photon and the leading muon (left) and the trailing muon (right) for $\Upsilon(1S,2S,3S)$ from data and signal events for Higgs decaying after all (Group I+II) selection cuts. The plots are normalized to the number of events. Signal sample is scaled by a factor of $\times 600000$).}
\label{fig:deltaR_HtoUpsilon_Cat0_groupI_plus_II}
\end{figure}


%%%%%%% energy/mass ratio distributions for HtoUpsilon_Cat0
\begin{figure}[!htbp]
\begin{center}
\includegraphics[width=0.45\textwidth]{figures_and_tables/outputPlots/HtoUpsilon_Cat0_ZZZZZ/nEvts/data_x_mc/withKinCuts/h_withKin_upsilonPt_over_zMass}\hspace*{1.cm}
\includegraphics[width=0.45\textwidth]{figures_and_tables/outputPlots/HtoUpsilon_Cat0_ZZZZZ/nEvts/data_x_mc/withKinCuts/h_withKin_photonPt_over_zMass}
\end{center}\vspace*{-.5cm}
\caption{The ratio for the transverse momentum of the reconstructed Upsilon and the reconstructed Higgs mass ($p_{T}^{\mu\mu}/M_{\mu\mu\gamma}$ - left) and the ratio for the transverse energy of the reconstructed Photon and the reconstructed Higgs mass ($E_{T}^{\mu\mu}/M_{\mu\mu\gamma}$ - right) distribution for $\Upsilon(1S,2S,3S)$ from data and signal events for Higgs decaying after all (Group I+II) selection cuts. The plots are normalized to the number of events. Signal sample is scaled by a factor of $\times 600000$).}
\label{fig:energy_ration_HtoUpsilon_Cat0_groupI_plus_II}
\end{figure}

%%%%%%%%%% dimuon mass distributions for HtoUpsilon_Cat0
%\begin{figure}[!htbp]
%\begin{center}
%\includegraphics[width=0.45\textwidth]{figures_and_tables/outputPlots/HtoUpsilon_Cat0_ZZZZZ/nEvts/data_x_mc/withKinCuts/h_withKin_Upsilon_Mass_Signal_and_Background_LargeRange}\hspace*{1.cm}
%\end{center}\vspace*{-.5cm}
%\caption{The dimuon mass distribution of the reconstructed $\Upsilon (1S,2S,3S)$ from data and signal events for Higgs decaying after all (Group I+II) selection cuts.}
%\label{fig:dimuon_mass_HtoUpsilon_Cat0_groupI_plus_II}
%\end{figure}

%%%%%%%%%%%%%%%%%%%%%%%%%%%%%%%%%%%%%%%%%%%%%%%%%%%%%%%%%%%%%%%%%%%%%%%%%%%%%%%%%%%%%%%%%%%%%%%%%%%%%%%%%
%%%%%%%%%%%%%%%%%%%%%%%%%%%%%%%%%%%%%%%%%%%%%%%%%%%%%%%%%%%%%%%%%%%%%%%%%%%%%%%%%%%%%%%%%%%%%%%%%%%%%%%%%
%%%%%%%%%%%%%%%%%%%%%%%%%%%%%%%%%%%%%%%%%%%%%%%%%%%%%%%%%%%%%%%%%%%%%%%%%%%%%%%%%%%%%%%%%%%%%%%%%%%%%%%%%
%%%%%%%%%%%%%%%%%%%%%%%%%%%%%%%%%%%%%%%%%%%%%%%%%%%%%%%%%%%%%%%%%%%%%%%%%%%%%%%%%%%%%%%%%%%%%%%%%%%%%%%%%
%%%%%%%%%%%%%%%%%%%%%%%%%%%%%%%%%%%%%%%%%%%%%%%%%%%%%%%%%%%%%%%%%%%%%%%%%%%%%%%%%%%%%%%%%%%%%%%%%%%%%%%%%
%%%%%%%%%%%%%%%%%%%%%%%%%%%%%%%%%%%%%%%%%%%%%%%%%%%%%%%%%%%%%%%%%%%%%%%%%%%%%%%%%%%%%%%%%%%%%%%%%%%%%%%%%
%%%%%%%%%%%%%%%%%%%%%%%%%%%%%%%%%%%%%%%%%%%%%%%%%%%%%%%%%%%%%%%%%%%%%%%%%%%%%%%%%%%%%%%%%%%%%%%%%%%%%%%%%
%%%%%%%%%%%%%%%%%%%%%%%%%%%%%%%%%%%%%%%%%%%%%%%%%%%%%%%%%%%%%%%%%%%%%%%%%%%%%%%%%%%%%%%%%%%%%%%%%%%%%%%%%
%%%%%%%%%%%%%%%%%%%%%%%%%%%%%%%%%%%%%%%%%%%%%%%%%%%%%%%%%%%%%%%%%%%%%%%%%%%%%%%%%%%%%%%%%%%%%%%%%%%%%%%%%
%%%%%%%%%%%%%%%%%%%%%%%%%%%%%%%%%%%%%%%%%%%%%%%%%%%%%%%%%%%%%%%%%%%%%%%%%%%%%%%%%%%%%%%%%%%%%%%%%%%%%%%%%
%%%%%%%%%%%%%%%%%%%%%%%%%%%%%%%%%%%%%%%%%%%%%%%%%%%%%%%%%%%%%%%%%%%%%%%%%%%%%%%%%%%%%%%%%%%%%%%%%%%%%%%%%
%             _ _     _     
%            (_) |   | |    
%  _   _  ___ _| | __| |___ 
% | | | |/ _ \ | |/ _` / __|
% | |_| |  __/ | | (_| \__ \
%  \__, |\___|_|_|\__,_|___/
%   __/ |                   
%  |___/                    
%%%%%%%%%%%%%%%%%%%%%%%%%%%%%%%%%%%%%%%%%%%%%%%%%%%%%%%%%%%%%%%%%%%%%%%%%%%%%%%%%%%%%%%%%%%%%%%%%%%%%%%%%
%%%%%%%%%%%%%%%%%%%%%%%%%%%%%%%%%%%%%%%%%%%%%%%%%%%%%%%%%%%%%%%%%%%%%%%%%%%%%%%%%%%%%%%%%%%%%%%%%%%%%%%%%
%%%%%%%%%%%%%%%%%%%%%%%%%%%%%%%%%%%%%%%%%%%%%%%%%%%%%%%%%%%%%%%%%%%%%%%%%%%%%%%%%%%%%%%%%%%%%%%%%%%%%%%%%
%%%%%%%%%%%%%%%%%%%%%%%%%%%%%%%%%%%%%%%%%%%%%%%%%%%%%%%%%%%%%%%%%%%%%%%%%%%%%%%%%%%%%%%%%%%%%%%%%%%%%%%%%
%%%%%%%%%%%%%%%%%%%%%%%%%%%%%%%%%%%%%%%%%%%%%%%%%%%%%%%%%%%%%%%%%%%%%%%%%%%%%%%%%%%%%%%%%%%%%%%%%%%%%%%%%
%%%%%%%%%%%%%%%%%%%%%%%%%%%%%%%%%%%%%%%%%%%%%%%%%%%%%%%%%%%%%%%%%%%%%%%%%%%%%%%%%%%%%%%%%%%%%%%%%%%%%%%%%
%%%%%%%%%%%%%%%%%%%%%%%%%%%%%%%%%%%%%%%%%%%%%%%%%%%%%%%%%%%%%%%%%%%%%%%%%%%%%%%%%%%%%%%%%%%%%%%%%%%%%%%%%
%%%%%%%%%%%%%%%%%%%%%%%%%%%%%%%%%%%%%%%%%%%%%%%%%%%%%%%%%%%%%%%%%%%%%%%%%%%%%%%%%%%%%%%%%%%%%%%%%%%%%%%%%
%%%%%%%%%%%%%%%%%%%%%%%%%%%%%%%%%%%%%%%%%%%%%%%%%%%%%%%%%%%%%%%%%%%%%%%%%%%%%%%%%%%%%%%%%%%%%%%%%%%%%%%%%
%%%%%%%%%%%%%%%%%%%%%%%%%%%%%%%%%%%%%%%%%%%%%%%%%%%%%%%%%%%%%%%%%%%%%%%%%%%%%%%%%%%%%%%%%%%%%%%%%%%%%%%%%



\clearpage
\section{Event categorization and yields}
\label{sec:categorization}

In order to increase the sensibility of the analysis, a categorization procedure was applied.
 % Table \ref{tab:categorization} present the 3 exclusive categories proposed. 
 They are based on the $\eta$ and R9 distribution of the reconstructed photon.

The photon R9 is a shower shape variable defined as the fraction of energy deposited in the 5x5 square surrounding the Super Cluster seed of the reconstructed photon. A photon that convert before reaching the ECAL tend to have lower values of R9, in comparison with unconverted photons. Converted photons have wider energy resolution and are more likely to be misidentified.

% \begin{table}[ht]
% \begin{center}

% \begin{tabular}{l|l}

% \textbf{Name} & \textbf{Definition} \\ \hline
% Cat0 & No categorization.                     \\ \hline
% EB High R9 & $|\eta_{SC}| < 1.444$ (Barrel)         \\
%      & $R9 >$ 0.94 (High R9)                           \\ \hline
% EB Low R9 & $|\eta_{SC}| < 1.444$ (Barrel)         \\
%      & $R9 <$ 0.94 (Low R9)                            \\ \hline
% EE & $1.566 < |\eta_{SC}| < 2.5$ (Endcap)   \\
%      & No R9 categorization.                  \\ 
% \end{tabular}


% \caption{Exclusive categories implemented based on the reconstructed photon. The uncategorized selection (Cat0) is the sum of the three categories and, for the \Z decay, it is kept only for reference. For the Higgs decay, it is actually used for the limits extraction.}
% \label{tab:categorization}
% \end{center}
% \end{table}

Selected events with the photon reconstructed inside the barrel and with $R9 > 0.94$ are categorized as "EB High R9"~\footnote{EB stands for Electromagnetic Barrel}, selected events with the photon reconstructed inside the barrel and with $R9 < 0.94$ are categorized as "EB Low R9" and selected events with the photon reconstructed inside the endcap, regardless of its R9 value, categorized as "EE". This categorization is done in view of increase the analysis sensitivity. 

This categorization is implemented only for the \Z decay. The Higgs does not present enough statistics to make it profitable, so only the inclusive one is used. 

\subsection{R9 reweighting}

As spotted by the $H \rightarrow \gamma\gamma$ at $\sqrt{13}$ TeV analysis~\cite{higgs_gammagamma_PAPPER}, there is a disagreement in the R9 distribution of photons in Data and MC. In order to mitigate this difference, a transformation factor is extracted and applied to the reconstructed photons before the categorization.

The same approach of the $H \rightarrow \gamma\gamma$ analysis is applied, in which the nominal photon selection of this analysis (see section \ref{sec:photon_id}) is used to select photons on Data and MC. Then the two distributions are remapped and the transformation factors are extracted. 

Figure \ref{fig:r9_reweighting} show the R9 distribution before and after the reweighting, for the Barrel and the Endcap.

%%%%%%%%% R9
% more info: https://eliza.web.cern.ch/eliza/public/ZGTo2MuG_MMuMu_EraDcutsR9/

\begin{figure}[!htbp]
\begin{center}
\includegraphics[width=0.45\textwidth]{figures_and_tables/R9/R9EB.png}\hspace*{1.cm}
\includegraphics[width=0.45\textwidth]{figures_and_tables/R9/R9EE.png}
\end{center}\vspace*{-.5cm}
\caption{Data and MC of the R9 variable, before and after the reweighting. Barrel (left) and Endcap (right).}
\label{fig:r9_reweighting}
\end{figure}

\subsection{Event counting and yields}
\label{sec:yields}

Tables \ref{tab:Z_yields} and \ref{tab:HIGGS_yields} show the total number of events before and after the full selection. Two things are important to notice.

\begin{table}[ht]
\begin{center}


\begin{tabular}{c|c|c|c|c|c}
%& \multicolumn{5}{c}{$Z \rightarrow \Upsilon(1S, 2S, 3S)+\gamma$}       \\
\hline
\hline

&  &  \multicolumn{3}{c|}{Signal $Z \rightarrow \Upsilon(nS)+\gamma$} &    \\
\cline{3-5}
& Data & $n=1$ & $n=2$ & $n=3$ &  $Z \rightarrow \mu\mu\gamma_{FSR}$  \\
\hline
Total & 169.84 M &  3.54 & 1.4 & 1.22 & $3.33 \times 10^{3}$  \\
\hline\hline
Inclusive & 447  &  0.393 &  0.157 &  0.136 &  176  \\
EB High R9 & 197  &  0.172 &  0.0682 &  0.0597 &  78  \\
EB Low R9 & 146  &  0.129 &  0.0519 &  0.0448 &  58.5  \\
EE & 104  &  0.0916 &  0.0365 &  0.032 &  39.8 \\

\end{tabular}


\caption{Number of events for the Z decay, before and after the full selection, per categorization scenarios.}
\label{tab:Z_yields}
\end{center}
\end{table}


\begin{table}[ht]
\begin{center}


\begin{tabular}{c|c|c|c|c|c}
%& \multicolumn{5}{c}{$H \rightarrow \Upsilon(1S, 2S, 3S)+\gamma$}       \\
\hline
\hline

&  &  \multicolumn{3}{c|}{Signal} &    \\
\cline{3-5}
& Data & $H \rightarrow \Upsilon(1S)+\gamma$ & $H \rightarrow \Upsilon(2S)+\gamma$ & $H \rightarrow \Upsilon(3S)+\gamma$ &  $H \rightarrow \gamma\gamma^{*}$  \\
\hline
Total (before selection) & 169.84 M &  0.000257 & $5.43 \times 10^{-5}$ & $3.93 \times 10^{-5}$ & 136  \\
\hline\hline
Inclusive & 231  &  $5.23 \times 10^{-5}$ &  $1.2 \times 10^{-5}$ &  $8.96 \times 10^{-6}$ &  1.22  \\

\end{tabular}


\caption{Number of events for the H decay, before and after the full selection.}
\label{tab:HIGGS_yields}
\end{center}
\end{table}


The signal selection efficiency is between 20\% and 21\% for all $\Upsilon$ states and categories. 

When one compares the fraction of selected peaking background, with respect to the selected data events for the Higgs decay (1.22/231), the fraction obtained ($\sim0.3\%$) is irrelevant. On the other hand, the same fraction for the \Z decay (176/447) is far from irrelevant ($\sim39\%$)\footnote{It is worth to keep in mind that this is a estimation based on MC}. The same relation is not found in the $H/Z \rightarrow J/\psi + \gamma$ analysis \cite{papper_jpsi}, where both decays (Higgs and \Z) show neglectable estimations of peaking background contribution to data. 
The very same behavior was found by ATLAS~\cite{atlas_paper:PhysRevLett.114.121801}. It can be explained by the relatively larger cross-section of the \Z peaking background ($Z \rightarrow \mu\mu\gamma_{FSR}$), with respect to the Higgs peaking background (Higgs Dalitz Decay). For the J/$\psi$ channel, it is not an issue since its  cross-section is way larger then the peaking background. The figures \ref{fig:dimuon_mass_ZtoUpsilon_Cat0} and \ref{fig:dimuon_mass_HtoUpsilon_Cat0} help to clarify these affirmations, for the \Z and Higgs decay, respectively. 
One can easily see how clear the J/$\psi$ peak is in both decays and how minor the Higgs Dalitz Decay contributions is to the $\Upsilon$ peak, with respect to the $Z \rightarrow \mu\mu\gamma_{FSR}$ contribution. It is important to keep in mind the different scaling of the peaking background distributions, x3 for the \Z and $\times 100$ for the Higgs.
The peaking background to the data due to $Z \rightarrow \Upsilon(1S,2S,3S) + \gamma$ channel  is the main motivation to use a 2-dimensional modeling fitting of the signal and background events, in order to add one more layer of differentiation between many backgrounds contributions which will be detailed in the next section.


%Since the proposal is to estimate the background from data, this considerable of the peaking background to the data, for the $Z \rightarrow \Upsilon(1S,2S,3S) + \gamma$ channel, is the main motivation to use a 2-dimensional modeling of the signal and background, in order to add one more layer of differentiation between many backgrounds contributions. It will be detailed in the next section.





\clearpage
%%%%%%%%%%%%%%%%%%%%%%%%%%%%%%%%%%%%%%
\section{Background modeling} \label{sec:background_modeling}
%\textbf {editors: Felipe Silva}

The background modeling proposed for this analysis is a two dimensional unbinned maximum likelihood fit on the $\mu\mu$ and the $\mu\mu\gamma$ invariant mass distributions. It is considered and modeled, as briefly discussed in \ref{sec:datasets}, three kinds of backgrounds:

% Concerning the main reducible background (combinatorial background), it is due to Drell-Yan process where the photon comes from the initial-state radiation (ISR) or final-state radiation (FSR) and background reducible events are produced from Drell-Yan with jet associate and inclusive quarkonium production, where the jet in both processes is misidentified as a photon in reconstruction level. For the analysis, besides the peaking background, most of the background contributions will be modeled from data. 

\begin{itemize}
  \item \textbf{Full Combinatorial}: any combination of two muon and one photon that pass all the object reconstruction and event selection criteria.
  \item \textbf{$\Upsilon$ Combinatorial}: a $\Upsilon(1S,2S,3S)$, that decays to a dimuon system, combined with a misidentified photon (misreconstructed, pileup photon, etc.), that pass all the object reconstruction, identification and event selection criteria.
  \item \textbf{Peaking background}: a \Z (or Higgs) that decays straight to a $\mu\mu\gamma$, that pass all the object reconstruction and event selection criteria, without passing through any intermediate state. The main contributions considered for this background are $Z \rightarrow \mu\mu\gamma_{\small{FSR}}$ (a Z decaying to a dimuon system with one of the muons irradiating a photon) or a Higgs Dalitz Decay.
\end{itemize}

All of them will be modeled from data, with some inputs from the MC (simulated) samples, as explained below. For both invariant mass spectra ($\mu\mu$ and $\mu\mu\gamma$) the full combinatorial background is expected to behave like a non-peaking distribution. The same behavior is expected for the $\mu\mu\gamma$ mass distribution of the $\Upsilon$ Combinatorial background and for the $\mu\mu$ mass distribution of the peaking background. 

On the other hand, the $\mu\mu$ distribution of the $\Upsilon$ Combinatorial background and the $\mu\mu\gamma$ mass distribution for the peaking background are expected to behave like a peaking distribution, centered around the $\Upsilon(1S, 2S, 3S)$ invariant mass (9.46 GeV, 10.02 GeV and 10.35 GeV)~\cite{pdg_2020} and the Z boson invariant mass (91.2 GeV)~\cite{pdg_2020}, respectively . Table \ref{tab:BckgModeling_Z} summarizes the background modeling proposed for this analysis.

% % bckg modeling
% \begin{table}[ht]
% \begin{center}
% \begin{tabular}{l|c|c}
%                          & $M_{\mu\mu}$                                          & $M_{\mu\mu\gamma}$       \\ \hline 
% \multirow{2}{*}{\textbf{Peaking background} }      & \multirow{2}{*}{Bernstein 1\textsuperscript{st} order}                 & Crystal Ball (Higgs decay)    \\ 
%                                   &                       & Double Crystal Ball (Z decay)    \\ \hline
% \textbf{$\Upsilon$ Combinatorial} & 3 Gaussians (one for each $\Upsilon$ state) & \multirow{2}{*}{Polynomial (F-Test)}  \\ \cline{1-2}
% \textbf{Full Combinatorial}       & Chebychev 1\textsuperscript{st} order                 &                          \\ 
% \end{tabular}

% \caption{Modeling for each background source and mass component.}
% \label{tab:BckgModeling_Z}
% \end{center}
% \end{table}


% bckg modeling
\begin{table}[ht]
\begin{center}
\caption{Modeling for each background source and mass component.}
\begin{tabular}{l|c|c}
                         & $m_{\mu\mu}$                                          & $m_{\mu\mu\gamma}$       \\ \hline 
\multirow{2}{*}{\textbf{Peaking background} }      & \multirow{2}{*}{Bernstein 1\textsuperscript{st} order}                 & Crystal Ball (Higgs decay)    \\ 
                                  &                       & Double Crystal Ball (Z decay)    \\ \hline
\textbf{$\Upsilon$ Combinatorial} & 3 Gaussians (one for each $\Upsilon$ state) & \multirow{2}{*}{Polynomial}  \\ \cline{1-2}
\textbf{Full Combinatorial}       & Chebychev 1\textsuperscript{st} order                 &                          \\ 
\end{tabular}

\label{tab:BckgModeling_Z}
\end{center}
\end{table}


For the $Z \rightarrow \Upsilon(1S,2S,3S) +\gamma$ analysis, the peaking background model parameters are extracted by performing a simultaneous 2-dimensional fit over the invariant masses, $m_{\mu\mu}$ and $m_{\mu\mu\gamma}$, of the simulated $Z \rightarrow \mu\mu\gamma_{\small{FSR}}$ MC sample of events that passes the selection described in Section~\ref{sec:selection}, as in figure \ref{fig:ZToUpsilon_PeakingBackground}. Once the parameters are extracted, they are fixed and the \textit{pdf} (Probability Distributions Function) of the 2-Dimensional modeling is stored, leaving only the normalization of the \textit{pdf} as a parameter free to float (this will be determined from data). 

In order to describe the 2-dimensional invariant mass distribution of the Peaking Background, as stated in Table~\ref{tab:BckgModeling_Z}, the $m_{\mu\mu}$ component is described by a Bernstein polynomial of 1\textsuperscript{st} order~\cite{Bernstein_pol}, which is used here just a representation of a linear function. The $m_{\mu\mu\gamma}$ component is described by Double Crystal Ball function~\cite{cb_function}. A Crystal Ball function is a \textit{pdf} composed by a gaussian distribution and a power-law tail to the left, before certain threshold. The Crystal Ball function was named after the Crystal Ball Collaboration (first to use this \textit{pdf}) and it is widely used in high-energy physics to describe mass distributions that incorporate FSR (final state radiation) effects, via the power-law tail. A Double Crystal Ball is a Crystal Ball function with the power-law tail on both sides.

A Crystal Ball function is defined as:


\begin{equation}
\label{eqn:cb_function}
CB(x;\alpha,n,\bar x,\sigma) = N \cdot \begin{cases} \exp(- \frac{(x - \bar x)^2}{2 \sigma^2}), & \mbox{for }\frac{x - \bar x}{\sigma} > -\alpha \\
 A \cdot (B - \frac{x - \bar x}{\sigma})^{-n}, & \mbox{for }\frac{x - \bar x}{\sigma} \leqslant -\alpha \end{cases},
\end{equation}
where,

\begin{equation}
    \begin{split}
        &A = \left(\frac{n}{\left| \alpha \right|}\right)^n \cdot \exp\left(- \frac {\left| \alpha \right|^2}{2}\right), \notag\\
        &B = \frac{n}{\left| \alpha \right|}  - \left| \alpha \right|, \notag\\
        &N = \frac{1}{\sigma (C + D)}, \notag\\
        &C = \frac{n}{\left| \alpha \right|} \cdot \frac{1}{n-1} \cdot \exp\left(- \frac {\left| \alpha \right|^2}{2}\right), \notag\\
        &D = \sqrt{\frac{\pi}{2}} \left(1 + \operatorname{erf}\left(\frac{\left| \alpha \right|}{\sqrt 2}\right)\right),
    \end{split}
\end{equation}

and $erf$ is the error function.



% Z To Upsilon - peaking background
\begin{figure}[!htbp]
\begin{center}


\includegraphics[width=0.45\textwidth]{figures_and_tables/fitPlotFiles2D/ZToUpsilonPhotonSignalAndBackgroundFit/mMuMNU_ZToUpsilon1SPhotonSignalAndBackgroundFit_PeakingBackground_Cat0}\hspace*{1.cm}
\includegraphics[width=0.45\textwidth]{figures_and_tables/fitPlotFiles2D/ZToUpsilonPhotonSignalAndBackgroundFit/mHZ_ZToUpsilon1SPhotonSignalAndBackgroundFit_PeakingBackground_Cat0}\hspace*{1.cm}

\includegraphics[width=0.45\textwidth]{figures_and_tables/fitPlotFiles2D/ZToUpsilonPhotonSignalAndBackgroundFit/mMuMNU_ZToUpsilon1SPhotonSignalAndBackgroundFit_PeakingBackground_Cat1}\hspace*{1.cm}
\includegraphics[width=0.45\textwidth]{figures_and_tables/fitPlotFiles2D/ZToUpsilonPhotonSignalAndBackgroundFit/mHZ_ZToUpsilon1SPhotonSignalAndBackgroundFit_PeakingBackground_Cat1}\hspace*{1.cm}

\includegraphics[width=0.45\textwidth]{figures_and_tables/fitPlotFiles2D/ZToUpsilonPhotonSignalAndBackgroundFit/mMuMNU_ZToUpsilon1SPhotonSignalAndBackgroundFit_PeakingBackground_Cat2}\hspace*{1.cm}
\includegraphics[width=0.45\textwidth]{figures_and_tables/fitPlotFiles2D/ZToUpsilonPhotonSignalAndBackgroundFit/mHZ_ZToUpsilon1SPhotonSignalAndBackgroundFit_PeakingBackground_Cat2}\hspace*{1.cm}

\includegraphics[width=0.45\textwidth]{figures_and_tables/fitPlotFiles2D/ZToUpsilonPhotonSignalAndBackgroundFit/mMuMNU_ZToUpsilon1SPhotonSignalAndBackgroundFit_PeakingBackground_Cat3}\hspace*{1.cm}
\includegraphics[width=0.45\textwidth]{figures_and_tables/fitPlotFiles2D/ZToUpsilonPhotonSignalAndBackgroundFit/mHZ_ZToUpsilon1SPhotonSignalAndBackgroundFit_PeakingBackground_Cat3}\hspace*{1.cm}


\end{center}\vspace*{-.5cm}
\caption{Peaking background for the $Z \rightarrow \Upsilon(1S,2S,3S) +\gamma$ analysis. $\mu\mu$ mass distribution (left) and $\mu\mu\gamma$ invariant mass distribution (right). From top to bottom categories: Inclusive, EB High R9, EB Low R9, EE.}
\label{fig:ZToUpsilon_PeakingBackground}
\end{figure}

For the three gaussian functions fits, which represent the three $\Upsilon$ states (1S, 2S and 3S) from the $\Upsilon$ Combinatorial background in the $m_{\mu\mu}$ component, we use a $\Upsilon$ control sample in order to extract the fit parameters, including the relative normalization between each $\Upsilon$ state. This sample is composed by dimuon candidates obtained from data, by selecting the events that passes the same trigger and dimuon selection of the nominal selection and with $p_{T}^{\mu\mu} > $ 35 GeV (this cut is done in order to keep this selected dimuon candidates compatibles with the $p_{T}^{\mu\mu}/M_{\mu\mu\gamma}$ cut applied in the nominal selection). No selection or cuts in the photon are required.


This control sample is fitted with a Chebychev 1\textsuperscript{st} order (linear polynomial) for the background support and 3 gaussian with the following constraints:

\begin{itemize}
  \item the mean of each state should be the ones in the PDG \cite{pdg_2020}, but allowed to shift by a float and common (the same for all states) value.
  \item the sigma should be based on the 1S fit of the MC. All other sigma should be the result of the 1S sigma times the state mass over the 1S mass ($\sigma_{2S,3S} = \frac{m_{2S,3S}}{m_{1S}} \sigma_{1S}$).
\end{itemize}

The idea behind this fit is that scale and resolution (mean and sigma, respectively, of the gaussians) over a sample without a photon selection should be the same as over a sample with photon selection, since these are detector only dependent effects. The fact that we exclude the photon from this control sample, improves the statistics and gives a better measurement of these variables.

The fit of the $\Upsilon$ control sample if shown in figure \ref{fig:upsilon_control_Fit}.


\begin{figure}[!htbp]
\begin{center}

\includegraphics[width=0.60\textwidth]{figures_and_tables/fitPlotFiles2D/UpsilonControlSample/upsilonControlSample_ZToUpsilonPhoton_Cat0}

\end{center}\vspace*{-.5cm}

\caption{$\Upsilon$ control sample fit with Chebychev 1\textsuperscript{st} order for the background support and 3 gaussian for the three $\Upsilon(1S,2S,3S)$ peaks.}
\label{fig:upsilon_control_Fit}
\end{figure}


% Once determined, the fit parameters above are fixed and used to compose the 2-Dimensional \textit{pdf} The $m_{\mu\mu}$ component of the full combinatorial background is derived fully from the data fit (described below). In the same sense, the $m_{\mu\mu\gamma}$ component of the full combinatorial and the $\Upsilon$ Combinatorial backgrounds are also fully derived from the data, but following a much complex procedure: a composition with the \textit{pdf} components described above, plus a F-Test within a Discrete Profiling (or "Envelope Method").


Once determined, the fit parameters are fixed, they are used to compose the 2-Dimensional \textit{pdf}. The $m_{\mu\mu}$ component of the full combinatorial background is derived fully from the data fit (described below). In the same sense, the $m_{\mu\mu\gamma}$ component of the full combinatorial and the $\Upsilon(nS)$ Combinatorial backgrounds are also fully derived from the data, but following a more complex procedure: a composition with the \textit{pdf} components described above, plus a statistical test, to avoid overfitting within a Discrete Profiling (or "Envelope Method"), as described in~\cite{DiscreteProfilingMethod} and also implemented in~\cite{higgs_gammagamma_PAPPER}. 

The statistical test consists of, for each category, different orders of a set of polynomial \textit{pdfs} families are tested: a sums of exponentials, sum of polynomials (in the Bernstein basis), a Laurent series and a sums of power-law functions. 


\begin{itemize}
\item Sums of exponentials: $$ f_{N}(x)= \sum^{N}_{i=1} p_{2i} e^{p_{2i+1} x} ,$$
\item Sums of polynomials (in the Bernstein basis): $$ f_{N}(x) = \sum^{N}_{i=0} p_{i} b_{(i,N)}, \text{ where } b_{(i,N)}:= \begin{pmatrix} N \\ i \end{pmatrix} x^i (1-x)^{N-i} ,$$
\item Laurent series: $$ f_{N}(x)= \sum^{N}_{i=1} p_{i} x^{-4 + \sum^{i}_{j=1} (-1)^{j} (j-1)},$$
\item Sums of power-law functions: $$ f_{N}(x)= \sum^{N}_{i=1} p_{2i} x^{-p_{2i+1}},$$
\end{itemize}
where for all $k$, the $p_k$ are a set of floating parameters in the fit.

Twice difference in the negative log-likelihood ($NLL$) between the $N^{th}$ and the $(N+1)^{th}$ order of the same polynomial ($\Delta NLL = 2 \times (NLL_{N} - NLL_{N+1})$) is expected to follow a $\chi^2$ distribution with $M$ degrees of freedom, where $M$ is the increase in degrees of freedom when going from $N^{th}$ to $(N+1)^{th}$. This can be shown with the help of the Wilks' theorem~\cite{wilks1938}. 
% In summary, a likelihood ratio test in the form of $\Lambda = \mathcal{L(\theta_{null})}/\mathcal{L(\theta_{alternative})}$, between the null hypotheses and an alternative one, , 

% \begin{equation}
% \label{eqn:wilks}
% -2log(\Lambda) \sim \chi^2_M,
% \end{equation}
% where $\Lambda$ is a likelihood ratio test in the form of $\Lambda = \mathcal{L(\theta_{null})}/\mathcal{L(\theta_{alternative})}$, between the null hypotheses and an alternative one.

Starting from the lowest order possible, the best choice of order, for each family, is determined when a increase in the order of the polynomial, does not brings a significant improvement in the quality of the fit. Since a model with more fit parameters (higher order polynomials) will always perform, if not the same, better than a simpler one, an optimal choice of the polynomial order, will be the one right before the model becomes too flexible for the data.

Consider a $p$-value defined as: 

\begin{equation}
\label{eqn:p-value_f_test}
\begin{split}
 p\text{-value} & = \int^{\infty}_{\Delta NLL} \chi^2_M(\Delta) \text{ } d\Delta\\
& = P(\chi^2_M > \Delta NLL)  ,
\end{split}
\end{equation}

In the same spirit as the Wilks' theorem, this is the $p$-value for a likelihood ratio test between a null hypotheses and an alternative model, where the null hypotheses is the $N^{th}$ order and $(N+1)^{th}$ order is the alternative one.

\begin{equation}
\label{eqn:likehood_ratio}
\begin{split}
 \Delta NLL & = 2 \times (NLL_{N} - NLL_{N+1}) \\
  & = -2 \times log(\frac{\mathcal{L}_N}{\mathcal{L}_{N+1}}),
\end{split}
\end{equation}
where $\mathcal{L}_N$ is the likelihood for the $N^{th}$ polynomial order.

The alternative will present a statistically significant improvement, with respect to the null hypotheses, if the $p$-value is smaller than 0.05, since the probability of obtaining, by chance, considering the null hypotheses is true, a even higher $\Delta NLL$ is less than 5\%. This will give support to chose $(N+1)^{th}$ over $N^{th}$.

If the $p$-value is greater than 0.05 a higher order is not supported, since the probability of obtaining a $\Delta NLL$ greater than the one observed is statistically significant (more than 5\%). A higher $\Delta NLL$ means that another data sample, collected and analyzed with strictly the same conditions, would have a probability of more than 5\% of giving a better fit improvement than the one observed, again assuming that the null hypotheses is true. This is an indication of overfitting, since the improvements are likely to come from just statistical fluctuations. When testing the $(N+1)^{th}$ order and this condition is reached, the optimal order should be the $N^{th}$.

At first, before any fit to data, the 2-Dimensional model is composed by the five components, as described in Table \ref{tab:BckgModeling_Z} (in which the $m_{\mu\mu\gamma}$ modeling for the Full Combinatorial Background and the $\Upsilon$ combinatorial are shared), then, the statistical test described before is ran for each family. It is important to stress that before the statistical test all the other fitting parameters have been fixed. This leaves only the normalizations of the model components and the polynomial coefficients free to float.

Once the optimal order for each \textit{pdf} family is obtained, the composed \textit{pdf} with each choice from statistical test is saved in the same model, providing a discrete variable that indexes the different polynomial \textit{pdf} families. This method is called Discrete Profiling (or \textit{"Envelope Method"}) and it allows the analysis algorithm to treat the choice of the \textit{pdf} as a systematics and incorporate its effect in the extracted upper limits. This model, with different choices of polynomial families is called envelope.

The implementation, used in the analysis, of the statistical test and the Discrete Profiling is based on the same algorithm used by the $H \rightarrow \gamma\gamma$ Run II analysis. An extensive documentation on these methods can be found in $H \rightarrow \gamma\gamma$ analysis note and physics analysis summary \cite{higgs_gammagamma_AN, higgs_gammagamma_PAS} and in the specific reference of the Discrete Profiling \cite{DiscreteProfilingMethod}. The figures \ref{fig:ZToUpsilon_mMuMU_Projection} and \ref{fig:ZToUpsilon_mHZ_Projection} show the projection for the $\mu\mu$ and $\mu\mu\gamma$ distribution after the statistical test.

% Z To Upsilon - mMuMU Projection
\begin{figure}[!htbp]
\begin{center}
\includegraphics[width=0.45\textwidth]{figures_and_tables/fitPlotFiles2D/ftestOutput2D/outdir_ZToUpsilonPhoton_Cat0/bkgfTest-Data/mMuMU_multipdf_UntaggedTag_0}\hspace*{1.cm}
\includegraphics[width=0.45\textwidth]{figures_and_tables/fitPlotFiles2D/ftestOutput2D/outdir_ZToUpsilonPhoton_Cat1/bkgfTest-Data/mMuMU_multipdf_UntaggedTag_0}
\includegraphics[width=0.45\textwidth]{figures_and_tables/fitPlotFiles2D/ftestOutput2D/outdir_ZToUpsilonPhoton_Cat2/bkgfTest-Data/mMuMU_multipdf_UntaggedTag_0}\hspace*{1.cm}
\includegraphics[width=0.45\textwidth]{figures_and_tables/fitPlotFiles2D/ftestOutput2D/outdir_ZToUpsilonPhoton_Cat3/bkgfTest-Data/mMuMU_multipdf_UntaggedTag_0}
\end{center}\vspace*{-.5cm}
\caption{$Z \rightarrow \Upsilon(1S,2S,3S) +\gamma$ Background Modeling: $\mu\mu$ distribution. Inclusive (top left); EB High R9 (top right), EB Low R9 (bottom left), EE (bottom right). The pdfs projections are plotted with respect to the overall best choice of the statistica test.}
\label{fig:ZToUpsilon_mMuMU_Projection}
\end{figure}

%%%%%%%%%%%%

% Z To Upsilon - mHZ Projection
\begin{figure}[!htbp]
\begin{center}
\includegraphics[width=0.45\textwidth]{figures_and_tables/fitPlotFiles2D/ftestOutput2D/outdir_ZToUpsilonPhoton_Cat0/bkgfTest-Data/mHZ_multipdf_UntaggedTag_0}\hspace*{1.cm}
\includegraphics[width=0.45\textwidth]{figures_and_tables/fitPlotFiles2D/ftestOutput2D/outdir_ZToUpsilonPhoton_Cat1/bkgfTest-Data/mHZ_multipdf_UntaggedTag_0}
\includegraphics[width=0.45\textwidth]{figures_and_tables/fitPlotFiles2D/ftestOutput2D/outdir_ZToUpsilonPhoton_Cat2/bkgfTest-Data/mHZ_multipdf_UntaggedTag_0}\hspace*{1.cm}
\includegraphics[width=0.45\textwidth]{figures_and_tables/fitPlotFiles2D/ftestOutput2D/outdir_ZToUpsilonPhoton_Cat3/bkgfTest-Data/mHZ_multipdf_UntaggedTag_0}
\end{center}\vspace*{-.5cm}
\caption{$Z \rightarrow \Upsilon(1S,2S,3S) +\gamma$ Background Modeling: $\mu\mu\gamma$ distribution. Inclusive (top left); EB High R9 (top right), EB Low R9 (bottom left), EE (bottom right). The plotted \textit{pdfs} corresponds to the best choice by the statistical test for each family. The signal region, from 80 GeV to 100 GeV was blinded.}
\label{fig:ZToUpsilon_mHZ_Projection}
\end{figure}


For the $H \rightarrow \Upsilon(1S,2S,3S) + \gamma$ analysis, the same procedure is implemented, except for the peaking background modeling. Since the MC prediction for the contribution of the background is too small, according to the comparison between the final selected events for data and the Higgs Dalitz Decay sample, in order to avoid fitting over statistical fluctuations of the data sample, the Peaking Background, for the Higgs channel, is fully modeled from the MC sample, including its normalization, as shown in figure \ref{fig:HToUpsilon_PeakingBackground}, hence it is not included the the statistical test, neither in the final background modeling envelope.

The results of the background modeling for the Full Combinatorial and $\Upsilon$ Combinatorial, can be found at Figures \ref{fig:HToUpsilon_mMuMU_Projection} and \ref{fig:HToUpsilon_mHZ_Projection}, for the $\mu\mu$ and $\mu\mu\gamma$ distribution, respectively. It is worth to remember that, for the Higgs channel, we are not implementing any categorization.


% H To Upsilon - peaking background
\begin{figure}[!htbp]
\begin{center}
\includegraphics[width=0.45\textwidth]{figures_and_tables/fitPlotFiles2D/HToUpsilonPhotonSignalAndBackgroundFit/mMuMNU_HToUpsilon1SPhotonSignalAndBackgroundFit_PeakingBackground_Cat0}\hspace*{1.cm}
\includegraphics[width=0.45\textwidth]{figures_and_tables/fitPlotFiles2D/HToUpsilonPhotonSignalAndBackgroundFit/mHZ_HToUpsilon1SPhotonSignalAndBackgroundFit_PeakingBackground_Cat0}\hspace*{1.cm}
\end{center}\vspace*{-.5cm}
\caption{Peaking Background for the $H \rightarrow \Upsilon(1S,2S,3S) +\gamma$ analysis. $m_{\mu\mu}$ invariant mass distribution (left) and $m_{\mu\mu\gamma}$ invariant mass distribution (right).}
\label{fig:HToUpsilon_PeakingBackground}
\end{figure}


% H To Upsilon - mMuMU Projection
\begin{figure}[!htbp]
\begin{center}
\includegraphics[width=0.45\textwidth]{figures_and_tables/fitPlotFiles2D/ftestOutput2D/outdir_HToUpsilonPhoton_Cat0/bkgfTest-Data/mMuMU_multipdf_UntaggedTag_0}\hspace*{1.cm}
\end{center}\vspace*{-.5cm}
\caption{$H \rightarrow \Upsilon(1S,2S,3S) +\gamma$ Background Modeling: $m_{\mu\mu}$ distribution. The \textit{pdfs} projections are plotted with respect to the overall best choice of the statistical test.}
\label{fig:HToUpsilon_mMuMU_Projection}
\end{figure}

%%%%%%%%%%%%

% H To Upsilon - mHZ Projection
\begin{figure}[!htbp]
\begin{center}
\includegraphics[width=0.45\textwidth]{figures_and_tables/fitPlotFiles2D/ftestOutput2D/outdir_HToUpsilonPhoton_Cat0/bkgfTest-Data/mHZ_multipdf_UntaggedTag_0}\hspace*{1.cm}
\end{center}\vspace*{-.5cm}
\caption{$H \rightarrow \Upsilon(1S,2S,3S) +\gamma$ Background Modeling: $\mu\mu\gamma$ distribution. The plotted \textit{pdfs} corresponds to the best choice by the statistical test for each family. The signal region, from 115 GeV to 135 GeV was blinded.}
\label{fig:HToUpsilon_mHZ_Projection}
\end{figure}



\clearpage
%%%%%%%%%%%%%%%%%%%%%%%%%%%%%%%%%%%%%
\section{Signal modeling}
%\textbf {editors: Felipe Silva}
%SIGNAL MODELING...  % to be commented

Along the same lines as the background modeling (Section \ref{sec:background_modeling}), the signal modeling is implemented as a two dimensional unbinned maximum likelihood fit on the $m_{\mu\mu}$ and the $m_{\mu\mu\gamma}$ invariant masses distributions, but this time, only using the signal simulated MC samples~\ref{sec:datasets}. Since, for the two spectra, it is expected a peak-like distribution, one centered in the \Z (or Higgs) boson mass and the other centered in the $\Upsilon$ mass, two also peak-like analytics \textit{pdfs} were chosen to compose the signal model. The modeling is summarized in table \ref{tab:SignalModeling}.


% % signal modeling
% \begin{table}[ht]
% \begin{center}
% \begin{tabular}{l|c|c}
%                          & \boldmath$M_{\mu\mu}$                                          & \boldmath$M_{\mu\mu\gamma}$       \\ \hline 
% \textbf{\boldmath$Z \rightarrow \Upsilon(1S,2S,3S) +\gamma$}       & Crystal Ball & Double Crystal Ball      \\ \hline
% \textbf{\boldmath$H \rightarrow \Upsilon(1S,2S,3S) +\gamma$} & Double Crystal Ball & Crystal Ball + Gaussian with the same mean  \\ 
% \end{tabular}

% \caption{Modeling for each signal source and mass component.}
% \label{tab:SignalModeling}
% \end{center}
% \end{table}


% signal modeling
\begin{table}[ht]
\begin{center}
\begin{tabular}{l|c|c}
                         & \boldmath$m_{\mu\mu}$                                          & \boldmath$m_{\mu\mu\gamma}$       \\ \hline 
\textbf{\boldmath$Z \rightarrow \Upsilon(1S,2S,3S) +\gamma$}       & Double Crystal Ball & Double Crystal Ball      \\ \hline
\textbf{\boldmath$H \rightarrow \Upsilon(1S,2S,3S) +\gamma$} & Double Crystal Ball & Crystal Ball + Gaussian with the same mean  \\ 
\end{tabular}

\caption{Modeling for each signal source and mass component.}
\label{tab:SignalModeling}
\end{center}
\end{table}




The projections of the modeling for the \Z boson decay channel analysis can be found at figures \ref{fig:ZToUpsilon_Signal_Cat0}, \ref{fig:ZToUpsilon_Signal_Cat1}, \ref{fig:ZToUpsilon_Signal_Cat2} and \ref{fig:ZToUpsilon_Signal_Cat3}, for Inclusive, EB High R9, EB Low R9 and EE, respectively. The projection on the modeling for the Higgs boson signal can be found at Figure \ref{fig:HToUpsilon_Signal_Cat0}. A deeper discussion on the systematics uncertainties associated to them, will be presented in the next section.



% Z To Upsilon - Signal Modeling
% Cat0
\begin{figure}[!htbp]
\begin{center}


\includegraphics[width=0.45\textwidth]{figures_and_tables/fitPlotFiles2D/ZToUpsilonPhotonSignalAndBackgroundFit/mMuMNU_ZToUpsilon1SPhotonSignalAndBackgroundFit_Signal_Cat0}\hspace*{1.cm}
\includegraphics[width=0.45\textwidth]{figures_and_tables/fitPlotFiles2D/ZToUpsilonPhotonSignalAndBackgroundFit/mHZ_ZToUpsilon1SPhotonSignalAndBackgroundFit_Signal_Cat0_default}\hspace*{1.cm}

\includegraphics[width=0.45\textwidth]{figures_and_tables/fitPlotFiles2D/ZToUpsilonPhotonSignalAndBackgroundFit/mMuMNU_ZToUpsilon2SPhotonSignalAndBackgroundFit_Signal_Cat0}\hspace*{1.cm}
\includegraphics[width=0.45\textwidth]{figures_and_tables/fitPlotFiles2D/ZToUpsilonPhotonSignalAndBackgroundFit/mHZ_ZToUpsilon2SPhotonSignalAndBackgroundFit_Signal_Cat0_default}\hspace*{1.cm}

\includegraphics[width=0.45\textwidth]{figures_and_tables/fitPlotFiles2D/ZToUpsilonPhotonSignalAndBackgroundFit/mMuMNU_ZToUpsilon3SPhotonSignalAndBackgroundFit_Signal_Cat0}\hspace*{1.cm}
\includegraphics[width=0.45\textwidth]{figures_and_tables/fitPlotFiles2D/ZToUpsilonPhotonSignalAndBackgroundFit/mHZ_ZToUpsilon3SPhotonSignalAndBackgroundFit_Signal_Cat0_default}\hspace*{1.cm}


\end{center}\vspace*{-.5cm}
\caption{Signal Modeling for the $Z \rightarrow \Upsilon(1S,2S,3S) +\gamma$ analysis for Inclusive category. $m_{\mu\mu}$ mass distribution (left) and $m_{\mu\mu\gamma}$ mass distribution (right). From top to bottom: $\Upsilon(1S)$, $\Upsilon(2S)$, $\Upsilon(3S)$.}
\label{fig:ZToUpsilon_Signal_Cat0}
\end{figure}

% Cat1
\begin{figure}[!htbp]
\begin{center}


\includegraphics[width=0.45\textwidth]{figures_and_tables/fitPlotFiles2D/ZToUpsilonPhotonSignalAndBackgroundFit/mMuMNU_ZToUpsilon1SPhotonSignalAndBackgroundFit_Signal_Cat1}\hspace*{1.cm}
\includegraphics[width=0.45\textwidth]{figures_and_tables/fitPlotFiles2D/ZToUpsilonPhotonSignalAndBackgroundFit/mHZ_ZToUpsilon1SPhotonSignalAndBackgroundFit_Signal_Cat1_default}\hspace*{1.cm}

\includegraphics[width=0.45\textwidth]{figures_and_tables/fitPlotFiles2D/ZToUpsilonPhotonSignalAndBackgroundFit/mMuMNU_ZToUpsilon2SPhotonSignalAndBackgroundFit_Signal_Cat1}\hspace*{1.cm}
\includegraphics[width=0.45\textwidth]{figures_and_tables/fitPlotFiles2D/ZToUpsilonPhotonSignalAndBackgroundFit/mHZ_ZToUpsilon2SPhotonSignalAndBackgroundFit_Signal_Cat1_default}\hspace*{1.cm}

\includegraphics[width=0.45\textwidth]{figures_and_tables/fitPlotFiles2D/ZToUpsilonPhotonSignalAndBackgroundFit/mMuMNU_ZToUpsilon3SPhotonSignalAndBackgroundFit_Signal_Cat1}\hspace*{1.cm}
\includegraphics[width=0.45\textwidth]{figures_and_tables/fitPlotFiles2D/ZToUpsilonPhotonSignalAndBackgroundFit/mHZ_ZToUpsilon3SPhotonSignalAndBackgroundFit_Signal_Cat1_default}\hspace*{1.cm}


\end{center}\vspace*{-.5cm}
\caption{Signal Modeling for the $Z \rightarrow \Upsilon(1S,2S,3S) +\gamma$ analysis for EB High R9 category. $m_{\mu\mu}$ mass distribution (left) and $m_{\mu\mu\gamma}$ mass distribution (right). From top to bottom: $\Upsilon(1S)$, $\Upsilon(2S)$, $\Upsilon(3S)$.}
\label{fig:ZToUpsilon_Signal_Cat1}
\end{figure}

% Cat2
\begin{figure}[!htbp]
\begin{center}


\includegraphics[width=0.45\textwidth]{figures_and_tables/fitPlotFiles2D/ZToUpsilonPhotonSignalAndBackgroundFit/mMuMNU_ZToUpsilon1SPhotonSignalAndBackgroundFit_Signal_Cat2}\hspace*{1.cm}
\includegraphics[width=0.45\textwidth]{figures_and_tables/fitPlotFiles2D/ZToUpsilonPhotonSignalAndBackgroundFit/mHZ_ZToUpsilon1SPhotonSignalAndBackgroundFit_Signal_Cat2_default}\hspace*{1.cm}

\includegraphics[width=0.45\textwidth]{figures_and_tables/fitPlotFiles2D/ZToUpsilonPhotonSignalAndBackgroundFit/mMuMNU_ZToUpsilon2SPhotonSignalAndBackgroundFit_Signal_Cat2}\hspace*{1.cm}
\includegraphics[width=0.45\textwidth]{figures_and_tables/fitPlotFiles2D/ZToUpsilonPhotonSignalAndBackgroundFit/mHZ_ZToUpsilon2SPhotonSignalAndBackgroundFit_Signal_Cat2_default}\hspace*{1.cm}

\includegraphics[width=0.45\textwidth]{figures_and_tables/fitPlotFiles2D/ZToUpsilonPhotonSignalAndBackgroundFit/mMuMNU_ZToUpsilon3SPhotonSignalAndBackgroundFit_Signal_Cat2}\hspace*{1.cm}
\includegraphics[width=0.45\textwidth]{figures_and_tables/fitPlotFiles2D/ZToUpsilonPhotonSignalAndBackgroundFit/mHZ_ZToUpsilon3SPhotonSignalAndBackgroundFit_Signal_Cat2_default}\hspace*{1.cm}


\end{center}\vspace*{-.5cm}
\caption{Signal Modeling for the $Z \rightarrow \Upsilon(1S,2S,3S) +\gamma$ analysis for EB Low R9 category. $m_{\mu\mu}$ mass distribution (left) and $m_{\mu\mu\gamma}$ mass distribution (right). From top to bottom: $\Upsilon(1S)$, $\Upsilon(2S)$, $\Upsilon(3S)$.}
\label{fig:ZToUpsilon_Signal_Cat2}
\end{figure}

% Cat3
\begin{figure}[!htbp]
\begin{center}


\includegraphics[width=0.45\textwidth]{figures_and_tables/fitPlotFiles2D/ZToUpsilonPhotonSignalAndBackgroundFit/mMuMNU_ZToUpsilon1SPhotonSignalAndBackgroundFit_Signal_Cat3}\hspace*{1.cm}
\includegraphics[width=0.45\textwidth]{figures_and_tables/fitPlotFiles2D/ZToUpsilonPhotonSignalAndBackgroundFit/mHZ_ZToUpsilon1SPhotonSignalAndBackgroundFit_Signal_Cat3_default}\hspace*{1.cm}

\includegraphics[width=0.45\textwidth]{figures_and_tables/fitPlotFiles2D/ZToUpsilonPhotonSignalAndBackgroundFit/mMuMNU_ZToUpsilon2SPhotonSignalAndBackgroundFit_Signal_Cat3}\hspace*{1.cm}
\includegraphics[width=0.45\textwidth]{figures_and_tables/fitPlotFiles2D/ZToUpsilonPhotonSignalAndBackgroundFit/mHZ_ZToUpsilon2SPhotonSignalAndBackgroundFit_Signal_Cat3_default}\hspace*{1.cm}

\includegraphics[width=0.45\textwidth]{figures_and_tables/fitPlotFiles2D/ZToUpsilonPhotonSignalAndBackgroundFit/mMuMNU_ZToUpsilon3SPhotonSignalAndBackgroundFit_Signal_Cat3}\hspace*{1.cm}
\includegraphics[width=0.45\textwidth]{figures_and_tables/fitPlotFiles2D/ZToUpsilonPhotonSignalAndBackgroundFit/mHZ_ZToUpsilon3SPhotonSignalAndBackgroundFit_Signal_Cat3_default}\hspace*{1.cm}


\end{center}\vspace*{-.5cm}
\caption{Signal Modeling for the $Z \rightarrow \Upsilon(1S,2S,3S) +\gamma$ analysis for EE category. $m_{\mu\mu}$ mass distribution (left) and $m_{\mu\mu\gamma}$ mass distribution (right). From top to bottom: $\Upsilon(1S)$, $\Upsilon(2S)$, $\Upsilon(3S)$.}
\label{fig:ZToUpsilon_Signal_Cat3}
\end{figure}

% H To Upsilon - Signal Modeling
% Cat0
\begin{figure}[!htbp]
\begin{center}


\includegraphics[width=0.45\textwidth]{figures_and_tables/fitPlotFiles2D/HToUpsilonPhotonSignalAndBackgroundFit/mMuMNU_HToUpsilon1SPhotonSignalAndBackgroundFit_Signal_Cat0}\hspace*{1.cm}
\includegraphics[width=0.45\textwidth]{figures_and_tables/fitPlotFiles2D/HToUpsilonPhotonSignalAndBackgroundFit/mHZ_HToUpsilon1SPhotonSignalAndBackgroundFit_Signal_Cat0_default}\hspace*{1.cm}

\includegraphics[width=0.45\textwidth]{figures_and_tables/fitPlotFiles2D/HToUpsilonPhotonSignalAndBackgroundFit/mMuMNU_HToUpsilon2SPhotonSignalAndBackgroundFit_Signal_Cat0}\hspace*{1.cm}
\includegraphics[width=0.45\textwidth]{figures_and_tables/fitPlotFiles2D/HToUpsilonPhotonSignalAndBackgroundFit/mHZ_HToUpsilon2SPhotonSignalAndBackgroundFit_Signal_Cat0_default}\hspace*{1.cm}

\includegraphics[width=0.45\textwidth]{figures_and_tables/fitPlotFiles2D/HToUpsilonPhotonSignalAndBackgroundFit/mMuMNU_HToUpsilon3SPhotonSignalAndBackgroundFit_Signal_Cat0}\hspace*{1.cm}
\includegraphics[width=0.45\textwidth]{figures_and_tables/fitPlotFiles2D/HToUpsilonPhotonSignalAndBackgroundFit/mHZ_HToUpsilon3SPhotonSignalAndBackgroundFit_Signal_Cat0_default}\hspace*{1.cm}


\end{center}\vspace*{-.5cm}
\caption{Signal Modeling for the $H \rightarrow \Upsilon(1S,2S,3S) +\gamma$. $m_{\mu\mu}$ mass distribution (left) and $m_{\mu\mu\gamma}$ mass distribution (right). From top to bottom: $\Upsilon(1S)$, $\Upsilon(2S)$, $\Upsilon(3S)$.}
\label{fig:HToUpsilon_Signal_Cat0}
\end{figure}

\clearpage

\section{Systematic uncertainties}

Two sources of systematics are considered: the ones that affect the predicted yields~\footnote{Number of events, per process, after full selection and corrected by the expected SM cross sections.} and the ones that affect the shape of the pdfs used to compose the signal and background model.

Those that affect the predicted yields, presented in Section~\ref{sec:yields}, it is considered integrated luminosity measurement~\cite{CMS-PAS-LUM-17-001}, the pile-up description in the Monte-Carlo simulations, the corrections applied to the simulated events in order to compensate for the differences in performance of the some selection criteria, such as trigger, object reconstruction and identification, the $\Upsilon$ polarization and the theoretical uncertainties, such as the effects of the \textit{parton density functions} (PDF) to the signal cross section~\cite{NNPDF3,deFlorian:2016spz,Butterworth:2015oua}, the variations of the renormalization and factorization scales~\cite{Martin:2009iq,Lai:2010vv,Alekhin:2011sk,Botje:2011sn,Ball:2011mu}, and the prediction of the decay branching ratios. 

For the systematics on the signal modeling, it is considered possible imprecisions of the momentum scale and resolution. They are measured on how they affect the mean ($\mu$) and the standard deviation ($\sigma$) of the signal model. For the background modeling, since it is derived from data, the choice of the \textit{pdf} (Probability distribution function) is the only systematic uncertainties considered. It is treated by the Discrete Profiling method, as described in section \ref{sec:background_modeling}. 

The two kinds of systematics uncertainties are described in details below.

\subsection{Uncertainties on the predicted yields}

The theoretical sources of uncertainties includes: parton distribution functions uncertainties, strong coupling constant ($\alpha_{s}$) uncertainty and uncertainty on the $\mathrm{H} \to \gamma\gamma$ branching fraction (used to derive the Higgs Dalitz Decays cross-section). The values for these theoretical uncertainties are taken from the Higgs Combination Group~\cite{CERNYellowReportPageAt13TeV} and also from~\cite{Passarino:2013nka,Botje:2011sn}.

An uncertainty value of $2.5\%$ is used on the integrated luminosity of the data samples, as recommended by CMS~\cite{CMS-PAS-LUM-17-001}. To evaluate the impact of the pile-up reweighting in the final result, the The total inelastic cross section of $69.2~mb$ is varied by $\pm~4.6\%$ and the analysis is ran with these extreme values. The systematic uncertainty quoted is the maximum difference in the yields with respect to nominal value, as recommended by CMS. 

The impact of the trigger scale factor is evaluated by running this analysis with $\pm~1\sigma$ on the Trigger Efficiency Scale factors (section \ref{sec:trigger}). The systematic uncertainty quoted is the maximum difference in the yields with respect to nominal value.

For the final state object identification and isolation associated uncertainty, the scale factors, provided by CMS, to match the performance of MC and Data samples are varied in  $\pm~1\sigma$. The systematic uncertainty quoted is the maximum difference in the yields with respect to nominal value. This procedure is applied on the scale factors for the Photon MVA ID and the Electron Veto (section \ref{sec:photon_id}) and for  Muon Identification and Isolation(section \ref{sec:muon_id}). 

Finally, the $\Upsilon$ Polarization is assessed applying the extremes scenarios of the $\Upsilon$ polarization (Transverse and Longitudinal Polarization to the signal samples (section \ref{sec:polarization}). The systematic uncertainty quoted is the maximum difference in the yields with respect to nominal (Unpolarized) yield. This procedure is applied only for the Z decay. For the Higgs decay, the only polarization considered is the transverse polarization.

The effect of all systematic uncertainties in the signal and peaking background yields are summarized on table \ref{tab:FinalZSystLatex}, for the \Z decay and table \ref{tab:FinalHSystLatex}, for the Higgs decay. Clearly, the main contribution to the systematics uncertainties on the yields is Polarization of the $\Upsilon(nS)$ (only for the Z decay), around 15\%.

\begin{table}[ht]
  \begin{center}
    
% ADD TO HEADER:
%\usepackage{multirow} %multirow
%\usepackage[table]{xcolor}    % loads also colortbl
%\renewcommand{\familydefault}{\sfdefault}
%\rowcolors{0}{gray!25}{white}
\begin{tabular}{c|c|c|c|c}
\cline{1-5}
\multirow{3}{*}{Source} & \multicolumn{4}{c}{Uncertainty} \\
\cline{2-5}
& \multicolumn{3}{c|}{Signal} & Peaking Background   \\
\cline{2-5}
& $Z \rightarrow \Upsilon(1S)  \gamma$ & $Z \rightarrow \Upsilon(2S)  \gamma$ & $Z \rightarrow \Upsilon(3S)  \gamma$ & $Z \rightarrow \mu\mu\gamma$  \\
\hline\hline
Integrated luminosity & \multicolumn{4}{l}{} \\ \hline
All Categories & \multicolumn{4}{c}{2.5\%} \\
\hline\hline
SM Z boson $\sigma$ (scale) & \multicolumn{4}{l}{} \\ \hline
All Categories & \multicolumn{3}{c|}{3.5\%}  & \multicolumn{1}{c}{5.0\%} \\
\hline\hline
SM Z boson $\sigma$ (PDF + $\alpha_s$)  & \multicolumn{4}{l}{} \\ \hline
All Categories & \multicolumn{3}{c|}{1.73\%}  & \multicolumn{1}{c}{5.0\%} \\
\hline\hline
Pileup Reweighting  & \multicolumn{4}{l}{} \\ \hline
Inclusive & 0.74\% & 0.33\% & 0.35\% & 0.68\% \\
EB High R9 & 1.22\% & 1.01\% & 0.59\% & 1.2\% \\
EB Low R9 & 0.02\% & 0.97\% & 0.89\% & 0.19\% \\
EE & 0.88\% & 0.75\% & 1.05\% & 0.8\% \\
\hline\hline
Trigger  & \multicolumn{4}{l}{} \\ \hline
Inclusive & 4.33\% & 4.3\% & 4.33\% & 4.6\% \\
EB High R9 & 3.52\% & 3.51\% & 3.51\% & 3.65\% \\
EB Low R9 & 3.52\% & 3.56\% & 3.56\% & 3.67\% \\
EE & 7.67\% & 7.61\% & 7.65\% & 7.83\% \\
\hline\hline
Muon ID/Isolation & \multicolumn{4}{l}{} \\ \hline
Inclusive & 4.71\% & 4.7\% & 4.73\% & 4.53\% \\
EB High R9 & 4.39\% & 4.41\% & 4.43\% & 4.2\% \\
EB Low R9 & 4.57\% & 4.55\% & 4.58\% & 4.34\% \\
EE & 5.69\% & 5.7\% & 5.71\% & 5.45\% \\
\hline\hline
Photon ID  & \multicolumn{4}{l}{} \\ \hline
Inclusive & 1.13\% & 1.11\% & 1.1\% & 1.09\% \\
EB High R9 & 1.13\% & 1.1\% & 1.09\% & 1.11\% \\
EB Low R9 & 1.14\% & 1.13\% & 1.11\% & 1.08\% \\
EE & 1.09\% & 1.09\% & 1.09\% & 1.09\% \\
\hline\hline
Electron Veto  & \multicolumn{4}{l}{} \\ \hline
Inclusive & 1.05\% & 1.06\% & 1.05\% & 1.03\% \\
EB High R9 & 1.2\% & 1.2\% & 1.2\% & 1.2\% \\
EB Low R9 & 1.2\% & 1.2\% & 1.2\% & 1.2\% \\
EE & 0.45\% & 0.45\% & 0.45\% & 0.45\% \\
\hline\hline
Polarization  & \multicolumn{4}{l}{} \\ \hline
Inclusive & 15.0\% & 14.99\% & 14.77\% & - \\
EB High R9 & 14.74\% & 15.13\% & 14.95\% & - \\
EB Low R9 & 14.91\% & 14.09\% & 14.41\% & - \\
EE & 15.77\% & 16.3\% & 14.99\% & - \\
\hline\hline

\end{tabular}



    \caption{ A summary table of systematic uncertainties in the \Z boson decaying in $\Upsilon(1S,2S,3S) + \gamma$, affecting the final yields of the MC samples.}
    \label{tab:FinalZSystLatex}
  \end{center}
\end{table}


\begin{table}[ht]
  \begin{center}
    
% ADD TO HEADER:
%\usepackage{multirow} %multirow
%\usepackage[table]{xcolor}    % loads also colortbl
%\renewcommand{\familydefault}{\sfdefault}
%\rowcolors{0}{gray!25}{white}

\begin{tabular}{c|c|c|c|c}
\cline{1-5}
\multirow{3}{*}{Source} & \multicolumn{4}{c}{Uncertainty} \\
\cline{2-5}
& \multicolumn{3}{c|}{Signal $H \rightarrow \Upsilon(nS)  \gamma$} & Res. Background   \\
\cline{2-5}
& $n=1$ & $n=2$ & $n=3$ & $H \rightarrow \gamma\gamma^{*}$  \\
\hline\hline
Integrated luminosity & \multicolumn{4}{c}{2.5\%} \\
\hline
SM Higgs $\sigma$ (scale) & \multicolumn{4}{c}{+4.6\% / -6.7\%}  \\
\hline
SM Higgs $\sigma$ (PDF + $\alpha_s$) & \multicolumn{4}{c}{3.2\%}  \\
\hline
SM BR $H \rightarrow \gamma\gamma^{*}$  & \multicolumn{3}{c|}{-}  & \multicolumn{1}{c}{6.0\%} \\
\hline
Pileup Reweighting & 0.61\% & 0.68\% & 0.56\% & 0.9\% \\
\hline
Trigger & 5.61\% & 5.47\% & 5.5\% & 6.12\% \\
\hline
Muon Identification & 4.39\% & 4.36\% & 4.34\% & 4.33\% \\
\hline
Photon Identification  & 1.21\% & 1.22\% & 1.22\% & 1.2\% \\
\hline
Electron Veto & 1.04\% & 1.04\% & 1.04\% & 1.04\% \\
\hline
\end{tabular}

    \caption{A summary table of systematic uncertainties in the Higgs boson decaying in $ \Upsilon(1S,2S,3S) + \gamma$, affecting the final yields of the MC samples.}
    \label{tab:FinalHSystLatex}
  \end{center}
\end{table}


\subsection{Uncertainties that affect the signal fits}

Smearing and scaling corrections are applied on simulated events since the resolution of Monte Carlo is better than that on data and the detector might not catch all the possible differences in the detector performance, with respect to the data observation. They need to be estimated and included on the systematics. The corrections are:

\begin{itemize}  
  \item \textbf{Muon Momentum Scale and Resolution}: extracted by running the analysis with different setups of the official CMS Muon scaling and smearing package~\cite{cms_muon_performance}. The deviations, with respect to the default correction are summed in quadrature. Once the nominal parameters (mean or sigma) are obtained, the default corrections are shifted by $\pm~1\sigma$ and the fits are re-done, with the parameters of interest free to float and all others fixed. The systematic uncertainty quoted is the maximum difference of the parameter with respect to nominal value.
  
  \item \textbf{Photon Energy Scale and Resolution}: extracted by running the analysis with different sets of corrections, provided by the CMS~\footnote{CMS has not published, yet, a paper on the Run2 performance of the Photon reconstruction (This document is under internal review process.). Just as a reference, we cite the Run1 paper~\cite{egamma_run1_papper}.}. Once the nominal mean is obtained, the sets are changed and the fits are re-done, with the mean free to float and all others parameters fixed. The corrections are shifted by $\pm~1\sigma$ on each source of systematics (following standard CMS recommendations). The quoted as systematic uncertainty is the quadrature sum of the maximum deviation within each set.
  
\end{itemize}

The effective systematic uncertainty associated with the scale and resolution are the quadrature sum of the muon and photon contributions. The effect of all systematic uncertainties in the Signal fits are summarized on table \ref{tab:FinalZSystShapeLatex}, for the \Z and Higgs decay. 

\begin{table}[ht]
  \begin{center}
    

\begin{tabular}{|l|c|c|c|c|c|}
\cline{2-6}
% \hline
\multicolumn{1}{c|}{} & \multicolumn{4}{c|}{Z$\rightarrow \Upsilon(nS) + \gamma$} & H$\rightarrow \Upsilon(nS) + \gamma$ \\ \cline{2-6}
\multicolumn{1}{c|}{} & \textbf{Inclusive}  & \textbf{EB High R9}  & \textbf{EB Low R9}  & \textbf{EE} & \textbf{Inclusive}        \\ \hline \hline

\multicolumn{6}{|c|}{\textbf{Mean - Scale ($n=1$)}} \\ \hline
Muon Unc.           & 0.06\% & 0.05\% & 0.06\% & 0.11\% & 0.11\% \\ \hline
Photon Unc.         & 0.21\% & 0.13\% & 0.19\% & 0.26\% & 0.28\% \\ \hline
\textbf{Total Unc.} & 0.22\% & 0.14\% & 0.2\% & 0.28\% & 0.3\% \\ \hline 

\multicolumn{6}{|c|}{\textbf{Sigma - Resolution ($n=1$)}}            \\ \hline 
Muon Unc.           & 1.12\% & 0.84\% & 1.55\% & 1.14\% & 2.62\% \\ \hline
Photon Unc.         & 2.14\% & 2.48\% & 1.95\% & 2.79\% & 4.27\% \\ \hline
\textbf{Total Unc.} & 2.42\% & 2.61\% & 2.49\% & 3.01\% & 5.01\% \\ \hline \hline 


\multicolumn{6}{|c|}{\textbf{Mean - Scale ($n=2$)}} \\ \hline 
Muon Unc.           & 0.07\% & 0.05\% & 0.06\% & 0.13\% & 0.1\% \\ \hline
Photon Unc.         & 0.25\% & 0.11\% & 0.2\% & 0.19\% & 0.26\% \\ \hline
\textbf{Total Unc.} & 0.26\% & 0.12\% & 0.21\% & 0.23\% & 0.28\% \\ \hline 

\multicolumn{6}{|c|}{\textbf{Sigma - Resolution ($n=2$)}}            \\ \hline 
Muon Unc.           & 1.21\% & 1.54\% & 2.65\% & 1.66\% & 1.02\% \\ \hline
Photon Unc.         & 1.85\% & 2.67\% & 3.56\% & 3.6\% & 6.6\% \\ \hline
\textbf{Total Unc.} & 2.21\% & 3.08\% & 4.44\% & 3.97\% & 6.68\% \\ \hline \hline 


\multicolumn{6}{|c|}{\textbf{Mean - Scale ($n=3$)}}  \\ \hline 
Muon Unc.           & 0.06\% & 0.06\% & 0.06\% & 0.09\% & 0.09\% \\ \hline
Photon Unc.         & 0.22\% & 0.14\% & 0.25\% & 0.17\% & 0.23\% \\ \hline
\textbf{Total Unc.} & 0.23\% & 0.15\% & 0.26\% & 0.19\% & 0.25\% \\ \hline 

\multicolumn{6}{|c|}{\textbf{Sigma - Resolution ($n=3$)}}            \\ \hline 
Muon Unc.           & 1.78\% & 2.38\% & 2.1\% & 2.25\% & 3.46\% \\ \hline
Photon Unc.         & 2.51\% & 4.14\% & 2.23\% & 4.08\% & 5.48\% \\ \hline
\textbf{Total Unc.} & 3.08\% & 4.77\% & 3.07\% & 4.66\% & 6.48\% \\ \hline  
\end{tabular}

    \caption{A summary table of systematic uncertainties in the \Z(H) decaying in $ \Upsilon(1S,2S,3S) + \gamma$, affecting the signal fits.}
    \label{tab:FinalZSystShapeLatex}
  \end{center}
\end{table}

%\begin{table}[ht]
  %\begin{center}
    %\input{figures_and_tables/tables/FinalHSystShapeLatex.tex}
    %\caption{A summary table of systematic uncertainties in the Higgs boson decaying in $ \Upsilon(1S,2S,3S) + \gamma$, affecting the signal fits.}
    %\label{tab:FinalHSystShapeLatex}
    %\end{center}
    %\end{table}
    
    
    %%%
    %The effects of the \textit{parton density functions}(PDF) choice on the signal cross section [REFs2] couse the theoretical uncertainties, the lack of higher-order calculations for the scale [REF3], and the prediction of the decay branching ratios [REF4].
    %The other uncertainty affect the shape of the signal model arises from the impreciseness of the momentum (energy) scale and resolution are varied, and the effect on the mean and sigma value of the signal model is introduced as a shape nuisance parameter in the estimation of the limit.
    %%%%%%%%%%%%%%%%%%%%%%%%%%%%%%%%%%%%%%%%%
    %REF1 -CMS Collaboration Collaboration, "CMS Luminosity Measurements for the 2016 Data Taking Period" , Technical Report CMS-PAS-LUM-17-001, CERN, Geneva, 2017.
    
    %REFs2 - LHC Higgs Cross Section Working Group Collaboration, "Handbook of LHC Higgs Cross Sections: 4. Deciphering the Nature of the Higgs Sector",
    %doi:10.23731/CYRM-2017-002, arXiv:1610.07922.
    %- NNPDF Collaboration, "Parton distributions for the LHC Run II" JHEP 04 (2015) 040,doi:10.1007/JHEP04(2015)040, arXiv:1410.8849.
    %- J. Butterworth et al., “PDF4LHC recommendations for LHC Run II”, J. Phys. G43 (2016) 023001, doi:10.1088/0954-3899/43/2/023001, arXiv:1510.03865.
    %REF3 
    %- A. D. Martin,W. J. Stirling, R. S. Thorne, and G.Watt, “Parton distributions for the LHC” Eur. Phys. J. C 63 (2009) 189, doi:10.1140/epjc/s10052-009-1072-5,arXiv:0901.0002.
    %- H. Lai et al., “New parton distributions for collider physics”, Phys. Rev. D 82 (2010) 74024, doi:10.1103/PhysRevD.82.074024, arXiv:1007.2241.
    %- S. Alekhin et al., “The PDF4LHC Working Group Interim Report”, (2011).arXiv:1101.0536.
    %- M. Botje et al., “The PDF4LHC Working Group Interim Recommendations”, (2011).arXiv:1101.0538.
    %- R. D. Ball et al., “Impact of Heavy Quark Masses on Parton Distributions and LHC Phenomenology”, Nucl. Phys. B 849 (2011) 296,doi:10.1016/j.nuclphysb.2011.03.021, arXiv:1101.1300.
    %REF4
    %“SM Higgs production cross sections at  s=13 TeV (update in CERN Report4 2016)”. https://twiki.cern.ch/twiki/bin/view/LHCPhysics/CERNYellowReportPageAt13TeV. Revision 23 of the page.
    %REF5
    %CMS Collaboration, “Search for diboson resonances in the 2l 2nu final state”, CMS Physics Analysis Note CMS-AN-16-352, 2016.
    %
    %REF6 G. Passarino, “Higgs boson production and decay: Dalitz sector”, Phys. Lett. B 727 (2013) 424, doi:10.1016/j.physletb.2013.10.052, arXiv:1308.0422.
    
    
    
    
    %They are evaluated by varying contributing sources within their corresponding uncertainties and propagating every uncertainty to 
    
    
    
    
    %\begin{table}[ht]
      %\begin{center}
        %%\resizebox{.5\width}{!}{\begin{tabular}{l|llll}
\multicolumn{4}{c}{95\% C.L. Upper Limit} \\
\hline
\hline
& \multicolumn{3}{c}{$\mathcal{B}(Z \rightarrow \Upsilon\gamma)$ $[\times10^{-6}]$}      \\
\cline{2-4}
&  $\Upsilon(1S)$ & $\Upsilon(2S)$ & $\Upsilon(3S)$  \\
\hline
Expected     & $6.4^{+3.1}_{-2.0}$ &  $8.3^{+4.0}_{-2.5}$  & $8.0^{+3.9}_{-2.4}$            \\
Observed     & 9.0 &  12.3  & 11.4      \\
\hline
SM Prediction $[\times10^{-8}]$ & 4.8  &  2.4  & 1.9      \\
\hline
\hline
& \multicolumn{3}{c}{$\mathcal{B}(H \rightarrow \Upsilon\gamma)$ $[\times10^{-4}]$}       \\
\cline{2-4}
&  $\Upsilon(1S)$ & $\Upsilon(2S)$ & $\Upsilon(3S)$ &   \\
\hline
Expected     & $12.5^{+6.1}_{-3.9}$ &  $14.6^{+7.1}_{-4.5}$  & $13.6^{+6.6}_{-4.2}$        \\
Observed     & 11.5 &  13.6  & 12.7     \\
\hline
SM Prediction $[\times10^{-9}]$ & 5.2  &  1.4  & 0.9      \\
\hline
\hline
\end{tabular}

}
        %

\begin{tabular}{c|c|c|c|c|c}
%& \multicolumn{5}{c}{$H \rightarrow \Upsilon(1S, 2S, 3S)+\gamma$}       \\
\hline
\hline

&  &  \multicolumn{3}{c|}{Signal} &    \\
\cline{3-5}
& Data & $H \rightarrow \Upsilon(1S)+\gamma$ & $H \rightarrow \Upsilon(2S)+\gamma$ & $H \rightarrow \Upsilon(3S)+\gamma$ &  $H \rightarrow \gamma\gamma^{*}$  \\
\hline
Total (before selection) & 169.84 M &  0.000257 & $5.43 \times 10^{-5}$ & $3.93 \times 10^{-5}$ & 136  \\
\hline\hline
Inclusive & 231  &  $5.23 \times 10^{-5}$ &  $1.2 \times 10^{-5}$ &  $8.96 \times 10^{-6}$ &  1.22  \\

\end{tabular}


        %\caption{A summary table of yields in the Higgs boson decaying in $\Upsilon(1S,2S,3S) + \gamma$}
        %\label{fig:FinalHYields}
        %\end{center}
        %\end{table}
        
        %%%%%%%%%% Z boson%%%%
        
        
        %\begin{table}[ht]
          %\begin{center}
            %%\resizebox{.5\width}{!}{\begin{tabular}{l|llll}
\multicolumn{4}{c}{95\% C.L. Upper Limit} \\
\hline
\hline
& \multicolumn{3}{c}{$\mathcal{B}(Z \rightarrow \Upsilon\gamma)$ $[\times10^{-6}]$}      \\
\cline{2-4}
&  $\Upsilon(1S)$ & $\Upsilon(2S)$ & $\Upsilon(3S)$  \\
\hline
Expected     & $6.4^{+3.1}_{-2.0}$ &  $8.3^{+4.0}_{-2.5}$  & $8.0^{+3.9}_{-2.4}$            \\
Observed     & 9.0 &  12.3  & 11.4      \\
\hline
SM Prediction $[\times10^{-8}]$ & 4.8  &  2.4  & 1.9      \\
\hline
\hline
& \multicolumn{3}{c}{$\mathcal{B}(H \rightarrow \Upsilon\gamma)$ $[\times10^{-4}]$}       \\
\cline{2-4}
&  $\Upsilon(1S)$ & $\Upsilon(2S)$ & $\Upsilon(3S)$ &   \\
\hline
Expected     & $12.5^{+6.1}_{-3.9}$ &  $14.6^{+7.1}_{-4.5}$  & $13.6^{+6.6}_{-4.2}$        \\
Observed     & 11.5 &  13.6  & 12.7     \\
\hline
SM Prediction $[\times10^{-9}]$ & 5.2  &  1.4  & 0.9      \\
\hline
\hline
\end{tabular}

}
            %

\begin{tabular}{c|c|c|c|c|c}
%& \multicolumn{5}{c}{$Z \rightarrow \Upsilon(1S, 2S, 3S)+\gamma$}       \\
\hline
\hline

&  &  \multicolumn{3}{c|}{Signal $Z \rightarrow \Upsilon(nS)+\gamma$} &    \\
\cline{3-5}
& Data & $n=1$ & $n=2$ & $n=3$ &  $Z \rightarrow \mu\mu\gamma_{FSR}$  \\
\hline
Total & 169.84 M &  3.54 & 1.4 & 1.22 & $3.33 \times 10^{3}$  \\
\hline\hline
Inclusive & 447  &  0.393 &  0.157 &  0.136 &  176  \\
EB High R9 & 197  &  0.172 &  0.0682 &  0.0597 &  78  \\
EB Low R9 & 146  &  0.129 &  0.0519 &  0.0448 &  58.5  \\
EE & 104  &  0.0916 &  0.0365 &  0.032 &  39.8 \\

\end{tabular}


            %\caption{A summary table of yields in the Z boson decaying in  $\Upsilon(1S,2S,3S) + \gamma$}
            %\label{fig:FinalZYieldsLatex}
            %\end{center}
            %\end{table}
\clearpage
            
\clearpage
%%%%%%%%%%%%%%%%%%%%%%%%%%%%%%%%%%%%%%%
\section{Modeling Cross checks} \label{modeling_xchecks}

In order to test the applicability of the statistical (signal and background) modeling proposed in this study, a cross-check procedure is performed by generating a set of pseudo-experiemnts (toys datasets) based on the  the signal plus background model, for each decay channel ($H/Z \rightarrow \Upsilon(1S,2S,3S,)+\gamma$) with some signal injected.

The procedure consists of resample from the signal plus background a number of events, including some extra (injected signal). The amount of injected signal is controlled by the $\mu_{true}$ variable, where $\mu_{true} = X$ means inject $X$ times the expected signal.

Once generated, the toy dataset is refitted to the signal plus background model and the signal strength ($\mu_{fit}$) and its error $\sigma_{fit}$ are extracted. This procedures is repeated 10000 times and only for the inclusive category. Figures~\ref{fig:fits_xchecks_mMuMNU_Z}, ~\ref{fig:fits_xchecks_mHZ_Z}, ~\ref{fig:fits_xchecks_mHZ_H} and \ref{fig:fits_xchecks_mMuMNU_H} show examples of those fits for the Higgs and Z decay.


\begin{figure}[!htbp]
\begin{center}
 %mu = 20
 \includegraphics[width=0.3\textwidth]{figures_and_tables/modeling_xchecks/plots/ZToUpsilon1SPhoton_Cat0_signalStrenght_20/Cat0_mMuMNU_fit_s}
\includegraphics[width=0.3\textwidth]{figures_and_tables/modeling_xchecks/plots/ZToUpsilon2SPhoton_Cat0_signalStrenght_20/Cat0_mMuMNU_fit_s}
\includegraphics[width=0.3\textwidth]{figures_and_tables/modeling_xchecks/plots/ZToUpsilon3SPhoton_Cat0_signalStrenght_20/Cat0_mMuMNU_fit_s}
 %mu = 50
\includegraphics[width=0.3\textwidth]{figures_and_tables/modeling_xchecks/plots/ZToUpsilon1SPhoton_Cat0_signalStrenght_50/Cat0_mMuMNU_fit_s}
\includegraphics[width=0.3\textwidth]{figures_and_tables/modeling_xchecks/plots/ZToUpsilon2SPhoton_Cat0_signalStrenght_50/Cat0_mMuMNU_fit_s}
\includegraphics[width=0.3\textwidth]{figures_and_tables/modeling_xchecks/plots/ZToUpsilon3SPhoton_Cat0_signalStrenght_50/Cat0_mMuMNU_fit_s}
 %mu = 100
\includegraphics[width=0.3\textwidth]{figures_and_tables/modeling_xchecks/plots/ZToUpsilon1SPhoton_Cat0_signalStrenght_100/Cat0_mMuMNU_fit_s}
\includegraphics[width=0.3\textwidth]{figures_and_tables/modeling_xchecks/plots/ZToUpsilon2SPhoton_Cat0_signalStrenght_100/Cat0_mMuMNU_fit_s}
\includegraphics[width=0.3\textwidth]{figures_and_tables/modeling_xchecks/plots/ZToUpsilon3SPhoton_Cat0_signalStrenght_100/Cat0_mMuMNU_fit_s}
 %mu = 1000
\includegraphics[width=0.3\textwidth]{figures_and_tables/modeling_xchecks/plots/ZToUpsilon1SPhoton_Cat0_signalStrenght_1000/Cat0_mMuMNU_fit_s}
\includegraphics[width=0.3\textwidth]{figures_and_tables/modeling_xchecks/plots/ZToUpsilon2SPhoton_Cat0_signalStrenght_1000/Cat0_mMuMNU_fit_s}
\includegraphics[width=0.3\textwidth]{figures_and_tables/modeling_xchecks/plots/ZToUpsilon3SPhoton_Cat0_signalStrenght_1000/Cat0_mMuMNU_fit_s}
\end{center}
\caption{Examples of the toy datasets fit ($M_{\mu\mu}$), for the Z decay analysis, after the toy dataset refit, for 1S, 2S and 3S (left to right), with $\mu_{true}$ equals to 20, 50, 100, 1000 (top to bottom). The red lines corresponds to the background model (B), the green lines to signal model (S), the blue lines to the total (S+B) and the dots is the toy dataset.}
\label{fig:fits_xchecks_mMuMNU_Z}
\end{figure}



\begin{figure}[!htbp]
\begin{center}
 %mu = 20
 \includegraphics[width=0.3\textwidth]{figures_and_tables/modeling_xchecks/plots/ZToUpsilon1SPhoton_Cat0_signalStrenght_20/Cat0_mHZ_fit_s}
\includegraphics[width=0.3\textwidth]{figures_and_tables/modeling_xchecks/plots/ZToUpsilon2SPhoton_Cat0_signalStrenght_20/Cat0_mHZ_fit_s}
\includegraphics[width=0.3\textwidth]{figures_and_tables/modeling_xchecks/plots/ZToUpsilon3SPhoton_Cat0_signalStrenght_20/Cat0_mHZ_fit_s}
 %mu = 50
\includegraphics[width=0.3\textwidth]{figures_and_tables/modeling_xchecks/plots/ZToUpsilon1SPhoton_Cat0_signalStrenght_50/Cat0_mHZ_fit_s}
\includegraphics[width=0.3\textwidth]{figures_and_tables/modeling_xchecks/plots/ZToUpsilon2SPhoton_Cat0_signalStrenght_50/Cat0_mHZ_fit_s}
\includegraphics[width=0.3\textwidth]{figures_and_tables/modeling_xchecks/plots/ZToUpsilon3SPhoton_Cat0_signalStrenght_50/Cat0_mHZ_fit_s}
 %mu = 100
\includegraphics[width=0.3\textwidth]{figures_and_tables/modeling_xchecks/plots/ZToUpsilon1SPhoton_Cat0_signalStrenght_100/Cat0_mHZ_fit_s}
\includegraphics[width=0.3\textwidth]{figures_and_tables/modeling_xchecks/plots/ZToUpsilon2SPhoton_Cat0_signalStrenght_100/Cat0_mHZ_fit_s}
\includegraphics[width=0.3\textwidth]{figures_and_tables/modeling_xchecks/plots/ZToUpsilon3SPhoton_Cat0_signalStrenght_100/Cat0_mHZ_fit_s}
 %mu = 1000
\includegraphics[width=0.3\textwidth]{figures_and_tables/modeling_xchecks/plots/ZToUpsilon1SPhoton_Cat0_signalStrenght_1000/Cat0_mHZ_fit_s}
\includegraphics[width=0.3\textwidth]{figures_and_tables/modeling_xchecks/plots/ZToUpsilon2SPhoton_Cat0_signalStrenght_1000/Cat0_mHZ_fit_s}
\includegraphics[width=0.3\textwidth]{figures_and_tables/modeling_xchecks/plots/ZToUpsilon3SPhoton_Cat0_signalStrenght_1000/Cat0_mHZ_fit_s}
\end{center}
\caption{Examples of the toy datasets fit ($M_{\mu\mu\gamma}$), for the Z decay analysis, after the toy dataset refit, for 1S, 2S and 3S (left to right), with $\mu_{true}$ equals to 20, 50, 100, 1000 (top to bottom). The red lines corresponds to the background model (B), the green lines to signal model (S), the blue lines to the total (S+B) and the dots is the toy dataset.}
\label{fig:fits_xchecks_mHZ_Z}
\end{figure}



\begin{figure}[!htbp]
\begin{center}
 %mu = 20
 \includegraphics[width=0.3\textwidth]{figures_and_tables/modeling_xchecks/plots/HToUpsilon1SPhoton_Cat0_signalStrenght_100000/Cat0_mMuMNU_fit_s}
\includegraphics[width=0.3\textwidth]{figures_and_tables/modeling_xchecks/plots/HToUpsilon2SPhoton_Cat0_signalStrenght_100000/Cat0_mMuMNU_fit_s}
\includegraphics[width=0.3\textwidth]{figures_and_tables/modeling_xchecks/plots/HToUpsilon3SPhoton_Cat0_signalStrenght_100000/Cat0_mMuMNU_fit_s}
\includegraphics[width=0.3\textwidth]{figures_and_tables/modeling_xchecks/plots/HToUpsilon1SPhoton_Cat0_signalStrenght_200000/Cat0_mMuMNU_fit_s}
\includegraphics[width=0.3\textwidth]{figures_and_tables/modeling_xchecks/plots/HToUpsilon2SPhoton_Cat0_signalStrenght_200000/Cat0_mMuMNU_fit_s}
\includegraphics[width=0.3\textwidth]{figures_and_tables/modeling_xchecks/plots/HToUpsilon3SPhoton_Cat0_signalStrenght_200000/Cat0_mMuMNU_fit_s}
\includegraphics[width=0.3\textwidth]{figures_and_tables/modeling_xchecks/plots/HToUpsilon1SPhoton_Cat0_signalStrenght_300000/Cat0_mMuMNU_fit_s}
\includegraphics[width=0.3\textwidth]{figures_and_tables/modeling_xchecks/plots/HToUpsilon2SPhoton_Cat0_signalStrenght_300000/Cat0_mMuMNU_fit_s}
\includegraphics[width=0.3\textwidth]{figures_and_tables/modeling_xchecks/plots/HToUpsilon3SPhoton_Cat0_signalStrenght_300000/Cat0_mMuMNU_fit_s}
\includegraphics[width=0.3\textwidth]{figures_and_tables/modeling_xchecks/plots/HToUpsilon1SPhoton_Cat0_signalStrenght_1000000/Cat0_mMuMNU_fit_s}
\includegraphics[width=0.3\textwidth]{figures_and_tables/modeling_xchecks/plots/HToUpsilon2SPhoton_Cat0_signalStrenght_1000000/Cat0_mMuMNU_fit_s}
\includegraphics[width=0.3\textwidth]{figures_and_tables/modeling_xchecks/plots/HToUpsilon3SPhoton_Cat0_signalStrenght_1000000/Cat0_mMuMNU_fit_s}
\end{center}
\caption{Examples of the toy datasets fit ($M_{\mu\mu}$), for the Higgs decay analysis, after the toy dataset refit, for 1S, 2S and 3S (left to right), with $\mu_{true}$ equals to 100000, 200000, 300000, 1000000 (top to bottom). The red lines corresponds to the background model (B), the green lines to signal model (S), the blue lines to the total (S+B) and the dots is the toy dataset.}
\label{fig:fits_xchecks_mMuMNU_H}
\end{figure}





\begin{figure}[!htbp]
\begin{center}
 \includegraphics[width=0.3\textwidth]{figures_and_tables/modeling_xchecks/plots/HToUpsilon1SPhoton_Cat0_signalStrenght_100000/Cat0_mHZ_fit_s}
\includegraphics[width=0.3\textwidth]{figures_and_tables/modeling_xchecks/plots/HToUpsilon2SPhoton_Cat0_signalStrenght_100000/Cat0_mHZ_fit_s}
\includegraphics[width=0.3\textwidth]{figures_and_tables/modeling_xchecks/plots/HToUpsilon3SPhoton_Cat0_signalStrenght_100000/Cat0_mHZ_fit_s}
\includegraphics[width=0.3\textwidth]{figures_and_tables/modeling_xchecks/plots/HToUpsilon1SPhoton_Cat0_signalStrenght_200000/Cat0_mHZ_fit_s}
\includegraphics[width=0.3\textwidth]{figures_and_tables/modeling_xchecks/plots/HToUpsilon2SPhoton_Cat0_signalStrenght_200000/Cat0_mHZ_fit_s}
\includegraphics[width=0.3\textwidth]{figures_and_tables/modeling_xchecks/plots/HToUpsilon3SPhoton_Cat0_signalStrenght_200000/Cat0_mHZ_fit_s}
\includegraphics[width=0.3\textwidth]{figures_and_tables/modeling_xchecks/plots/HToUpsilon1SPhoton_Cat0_signalStrenght_300000/Cat0_mHZ_fit_s}
\includegraphics[width=0.3\textwidth]{figures_and_tables/modeling_xchecks/plots/HToUpsilon2SPhoton_Cat0_signalStrenght_300000/Cat0_mHZ_fit_s}
\includegraphics[width=0.3\textwidth]{figures_and_tables/modeling_xchecks/plots/HToUpsilon3SPhoton_Cat0_signalStrenght_300000/Cat0_mHZ_fit_s}
\includegraphics[width=0.3\textwidth]{figures_and_tables/modeling_xchecks/plots/HToUpsilon1SPhoton_Cat0_signalStrenght_1000000/Cat0_mHZ_fit_s}
\includegraphics[width=0.3\textwidth]{figures_and_tables/modeling_xchecks/plots/HToUpsilon2SPhoton_Cat0_signalStrenght_1000000/Cat0_mHZ_fit_s}
\includegraphics[width=0.3\textwidth]{figures_and_tables/modeling_xchecks/plots/HToUpsilon3SPhoton_Cat0_signalStrenght_1000000/Cat0_mHZ_fit_s}
\end{center}
\caption{Examples of the toy datasets fit ($M_{\mu\mu\gamma}$), for the Higgs decay analysis, after the toy dataset refit, for 1S, 2S and 3S (left to right), with $\mu_{true}$ equals to 100000, 200000, 300000, 1000000 (top to bottom). The red lines corresponds to the background model (B), the green lines to signal model (S), the blue lines to the total (S+B) and the dots is the toy dataset.}
\label{fig:fits_xchecks_mHZ_H}
\end{figure}








It is expected that the pulls distribution for the fitted signal strength ($\frac{\mu_{fit} - \mu_{true}}{\sigma_{fit}}$) should follow a Gaussian distribution centered in 0 and with $\sigma$ around 1. Figures~\ref{fig:modeling_xchecks_Z} and \ref{fig:modeling_xchecks_H} present those pulls distributions for the Z and Higgs decays, respectively.






\begin{figure}[!htbp]
\begin{center}
 %mu = 20
 \includegraphics[width=0.3\textwidth]{figures_and_tables/modeling_xchecks/plots/ZToUpsilon1SPhoton_Cat0_signalStrenght_20/pulls}
\includegraphics[width=0.3\textwidth]{figures_and_tables/modeling_xchecks/plots/ZToUpsilon2SPhoton_Cat0_signalStrenght_20/pulls}
\includegraphics[width=0.3\textwidth]{figures_and_tables/modeling_xchecks/plots/ZToUpsilon3SPhoton_Cat0_signalStrenght_20/pulls}
 %mu = 50
\includegraphics[width=0.3\textwidth]{figures_and_tables/modeling_xchecks/plots/ZToUpsilon1SPhoton_Cat0_signalStrenght_50/pulls}
\includegraphics[width=0.3\textwidth]{figures_and_tables/modeling_xchecks/plots/ZToUpsilon2SPhoton_Cat0_signalStrenght_50/pulls}
\includegraphics[width=0.3\textwidth]{figures_and_tables/modeling_xchecks/plots/ZToUpsilon3SPhoton_Cat0_signalStrenght_50/pulls}
 %mu = 100
\includegraphics[width=0.3\textwidth]{figures_and_tables/modeling_xchecks/plots/ZToUpsilon1SPhoton_Cat0_signalStrenght_100/pulls}
\includegraphics[width=0.3\textwidth]{figures_and_tables/modeling_xchecks/plots/ZToUpsilon2SPhoton_Cat0_signalStrenght_100/pulls}
\includegraphics[width=0.3\textwidth]{figures_and_tables/modeling_xchecks/plots/ZToUpsilon3SPhoton_Cat0_signalStrenght_100/pulls}
 %mu = 1000
\includegraphics[width=0.3\textwidth]{figures_and_tables/modeling_xchecks/plots/ZToUpsilon1SPhoton_Cat0_signalStrenght_1000/pulls}
\includegraphics[width=0.3\textwidth]{figures_and_tables/modeling_xchecks/plots/ZToUpsilon2SPhoton_Cat0_signalStrenght_1000/pulls}
\includegraphics[width=0.3\textwidth]{figures_and_tables/modeling_xchecks/plots/ZToUpsilon3SPhoton_Cat0_signalStrenght_1000/pulls}
\end{center}
\caption{Distribution of pulls ($\frac{\mu_{fit} - \mu_{true}}{\sigma_{fit}}$), for the Z decay analysis, after the toy dataset refit, for 1S, 2S and 3S (left to right), with $\mu_{true}$ equals to 20, 50, 100, 1000 (top to bottom).}
\label{fig:modeling_xchecks_Z}
\end{figure}




\begin{figure}[!htbp]
\begin{center}
\includegraphics[width=0.3\textwidth]{figures_and_tables/modeling_xchecks/plots/HToUpsilon1SPhoton_Cat0_signalStrenght_100000/pulls}
\includegraphics[width=0.3\textwidth]{figures_and_tables/modeling_xchecks/plots/HToUpsilon2SPhoton_Cat0_signalStrenght_100000/pulls}
\includegraphics[width=0.3\textwidth]{figures_and_tables/modeling_xchecks/plots/HToUpsilon3SPhoton_Cat0_signalStrenght_100000/pulls}
\includegraphics[width=0.3\textwidth]{figures_and_tables/modeling_xchecks/plots/HToUpsilon1SPhoton_Cat0_signalStrenght_200000/pulls}
\includegraphics[width=0.3\textwidth]{figures_and_tables/modeling_xchecks/plots/HToUpsilon2SPhoton_Cat0_signalStrenght_200000/pulls}
\includegraphics[width=0.3\textwidth]{figures_and_tables/modeling_xchecks/plots/HToUpsilon3SPhoton_Cat0_signalStrenght_200000/pulls}
\includegraphics[width=0.3\textwidth]{figures_and_tables/modeling_xchecks/plots/HToUpsilon1SPhoton_Cat0_signalStrenght_300000/pulls}
\includegraphics[width=0.3\textwidth]{figures_and_tables/modeling_xchecks/plots/HToUpsilon2SPhoton_Cat0_signalStrenght_300000/pulls}
\includegraphics[width=0.3\textwidth]{figures_and_tables/modeling_xchecks/plots/HToUpsilon3SPhoton_Cat0_signalStrenght_300000/pulls}
\includegraphics[width=0.3\textwidth]{figures_and_tables/modeling_xchecks/plots/HToUpsilon1SPhoton_Cat0_signalStrenght_1000000/pulls}
\includegraphics[width=0.3\textwidth]{figures_and_tables/modeling_xchecks/plots/HToUpsilon2SPhoton_Cat0_signalStrenght_1000000/pulls}
\includegraphics[width=0.3\textwidth]{figures_and_tables/modeling_xchecks/plots/HToUpsilon3SPhoton_Cat0_signalStrenght_1000000/pulls}
\end{center}
\caption{Distribution of pulls ($\frac{\mu_{fit} - \mu_{true}}{\sigma_{fit}}$), for the Higgs decay analysis, after the toy dataset refit, for 1S, 2S and 3S (left to right), with $\mu_{true}$ equals to 100000, 200000, 300000, 1000000 (top to bottom).}
\label{fig:modeling_xchecks_H}
\end{figure}


As a general conclusion on this cross check, as long as the toy MC generation is able to inject enough signal to be fit, the final modeling of this analysis is able to recover a Gaussian pulls distribution. This, of course, depends on the $\Upsilon$ state to be considered. For the Z decay, between $\mu_{true} = 50$ and  $\mu_{true} = 100$ (around a hundred of events passing full selection), while for the Higgs decay, it is needed only a few events after full selection, even thought it means hundreds of thousands times the expected signal, since the very small cross sections for the decay, as shown in Table~\ref{tab:MC}.

\clearpage


%%%%%%%%%%%%%%%%%%%%%%%%%%%%%%%%%%%%%%%%%%
\section{Results and conclusion}
\label{chaper_results}

A two-dimensional (2D) unbinned maximum-likelihood fit to the $m_{\mu^{+}\mu^{-}\gamma}$ and $m_{\mu^{+}\mu^{-}}$ distributions was used to compare the data with background and signal predictions. Search has been performed for a SM Higgs and $\mathrm{Z}$ boson decaying into a $\Upsilon(1S,2S,3S)\gamma$, with $\Upsilon(1S,2S,3S)$ subsequently decaying into $\mu^{+}\mu^{-}$ using data obtained from $35.9~fb^{-1}$ of $pp$ colisions at $\sqrt{s}=13~\mathrm{TeV}$. 

Since no excess has been observed above the background, the \CLs formalism is applied, in order to establish an upper limit in the branching fractions for each channel.

\subsection{The $CL_{s}$ formalism for upper limits setting at CMS}
\label{sec:cls}

The \CLs formalism~\cite{cls_Read_2002} consist in a modified frequentist approach to obtain an upper limit for a certain parameter of a model, with respect to the data, when there is no significant excess that could justify an observation. It is based on the profile-likelihood-ratio test statistic~\cite{profiled_lh} and asymptotic approximations~\cite{asymptotic_cls}. It is a standard upper limit setting procedure for the LHC experiments~\cite{CMS-NOTE-2011-005}.

When searching for non-observed phenomena, it is often usual to derive the results as a function of the signal strength modifier $\mu$, which is a free parameter of the full model (signal + background). It can be defined such as, the expectation value for the number of events in a bin~\footnote{A set of common analysis criteria.} is:


\begin{equation}
\label{eqn:signal_strength}
E[n] = \mu s + b,
\end{equation}
where, $s$ and $b$ are the expected number of signal and background events, respectively.

The Neyman–Pearson lemma~\cite{profiled_lh} states the likelihood ratio is the optimal test between a null hypothesis and an alternative one (i.e. background-only and signal-plus-background models). On top on this, one could build a likelihood ratio test as:


\begin{equation}
  \label{profile_likelihood_ratio_lep}
  q(\mu)=-2ln \left(\frac{ \mathcal{L}(\text{data} \vert \mu s + b) }{\mathcal{L}(\text{data} \vert b)} \right),
\end{equation}
  where the denominator and numerator defines the likelihoods for the background-only and signal-plus-background models, respectively. The was the hypothesis test used by LEP and Tevatron experiments (the former one, with some modifications to include the nuisances effects).

With these two models, i.e. their \textit{pdfs}, one can throw toy MC events in order to construct a distribution of $q(\mu)$, namely $f(q(\mu) \vert \mu)$. The $p$-value of $f(q(\mu) \vert \mu)$, as below, can be used to chose between each model.

\begin{equation}
  \label{p_value_lep}
  p_{\mu}=\int^{\infty}_{q(\mu)_{\text{data}}} f(q(\mu) \vert \mu) \text{ } dq(\mu),
\end{equation}
  where $q(\mu)_{\text{data}}$ is the observed value of $q(\mu)$ on data, for a given $\mu$. 
  
If $p_{\mu}$ is less than $\alpha$ (usually 0.05 or 0.1) the background-only model can be excluded in favor of the signal-plus-background model. For the purpose of a confidence interval estimation, the argument can be reversed and one could look for all the values of $\mu$ that would not be excluded with Confidence Level (CL) $1-\alpha$.

The problem with this definition is that, when the expected signal strength is very small, e.g. a invariant mass distribution in the TeV scale, the null hypothesis and the alternative one are almost indistinguishable. In this situation, a downward fluctuation of the background might lead us to exclude the alternative hypothesis (signal) in a region of low experimental sensitivity region. Putting in number, if we expect 50 background events and 2 signal events, but we observe 40 events, the signal would be easily excluded.

In order to take this effect into account, a modified frequentist approach for upper limits setting, the $CL_s$ was proposed during the Higgs search era at LEP. Lets start by considering a profile likelihood ratio~\cite{kendall} as below:


\begin{equation}
  \label{profile_likelihood_ratio_def}
  \lambda(\mu)=\frac{ \mathcal{L}(\text{data} \vert \mu, \doublehat{\theta}) }{\mathcal{L}(\text{data} \vert \hat{\mu}, \hat{\theta})},
\end{equation}
  where, $\mathcal{L}(\text{data} \vert \mu, \doublehat{\theta})$ is the profile likelihood function. 

Defining $\mu$ and the investigated signal strength, $\doublehat{\theta}$ is the nuisances that maximizes the likelihood for a given $\mu$ (fixed) while $\hat{\mu}$ and $\hat{\theta}$ are the signal strength and nuisances that, overall, maximizes the likelihood. The advantage of the CMS and ATLAS have a common set of statistical guidelines~\cite{cms_atlas_statistical_guidelines} to ensure the compatibility of the published results. Following these recommendations, the statistics test based on~\ref{profile_likelihood_ratio_def} is:

\begin{equation}
  \label{statistics_test_lhc}
  \tilde{q_{\mu}}=-2 ln[\lambda(\mu)], \text{ with } 0 \leqslant \hat{\mu} \leqslant \mu.
\end{equation}

The left side restriction ($0 \leqslant \hat{\mu}$) ensure us the proper physical interpretation of $\mu$ as a positive defined signal strength, i.e., the observation a process would, for a given bin, increase the number of events. The right side restriction $\hat{\mu} \leqslant \mu$ secure the interpretation of $\tilde{q_{\mu}}$'s $p-$value as a one-sided confidence interval. This is required for a upper limit definition.

The advantage of using the profile likelihood ratio is that, even though it takes into account the effect of nuisances in the likelihood, it is possible to prove that, with the use of Wilk's Theorem~\cite{wilks1938}, that a statistic test defined as $\tilde{q_{\mu}}$, asymptotically follows a chi-square distribution with one degree of freedom (the signal strength)~\cite{asymptotic_cls}. Thus, $\tilde{q_{\mu}}$ is said to be approximately independent of any nuisance and allow a fast computation of its $p$-value without the need of toy MC strategies (which can computationally demanding, depending on the complexity of the models), event though this is not the standard CMS/ATLAS recommendation.

Based on $\tilde{q_{\mu}}$, defined at~\ref{statistics_test_lhc}, one should compute the $\tilde{q_{\mu}}^{\text{obs}}$, also the $\hat{\theta}_{\mu}^{\text{obs}}$ and $\hat{\theta}_{\mu = 0}^{\text{obs}}$, which corresponds to the observed value of $\tilde{q_{\mu}}$ on data, the maximum likelihood estimator for the nuisances assuming some signal strength $\mu$ and assuming a background-only model, respectively. Then, the distributions of $f(\tilde{q_{\mu}} \vert \mu, \hat{\theta}_{\mu}^{\text{obs}})$ and $f(\tilde{q_{\mu}} \vert \mu=0, \hat{\theta}_{\mu = 0}^{\text{obs}})$ are generated tossing pseudo-random toy MC. Figure~\ref{toy_mc_profile_likelihood_test} presents an example of these two distributions.

\begin{figure}[htbp]
  \centering
  \includegraphics[width=0.7\textwidth,keepaspectratio]{figures_and_tables/cls/q_tilde.png}
  \caption{Example of $f(\tilde{q_{\mu}} \vert \mu, \hat{\theta}_{\mu}^{\text{obs}})$ $f(\tilde{q_{\mu}} \vert \mu=0, \hat{\theta}_{\mu = 0}^{\text{obs}})$ distributions generated with toy MC. Source:~\cite{cms_atlas_statistical_guidelines}.}
  \label{toy_mc_profile_likelihood_test}
\end{figure}

The $CL_s$ value is defined as:

\begin{equation}
  \label{cls_def}
  CL_s(\mu)=\frac{p_{s+b}(\mu)}{1-p_{b}},
\end{equation}
where:

\begin{equation}
  \label{cls_p_def1}
  p_{s+b}(\mu) = \int_{\tilde{q_{\mu}}^{\text{obs}}}^{\infty} f(\tilde{q_{\mu}} \vert \mu, \hat{\theta}_{\mu}^{\text{obs}}) \text{ }d\tilde{q_{\mu}}
\end{equation}
and
\begin{equation}
  \label{cls_p_def2}
  p_{b} = \int^{\tilde{q_{\mu}}^{\text{obs}}}_{-\infty} f(\tilde{q_{\mu}} \vert \mu=0, \hat{\theta}_{\mu = 0}^{\text{obs}}) \text{ }d\tilde{q_{\mu}}
\end{equation}

Scanning different values of $\mu$, within $0 \leqslant \hat{\mu} \leqslant \mu$, one would exclude the ones which $CL_s < \alpha$. CMS and ATLAS recommends a CL level ($1-\alpha$) of 95\%.

The main advantage of the $CL_s$ approach is that the presence of the denominator $1-p_{b}$ in~\ref{cls_def} penalizes the exclusion of regions in which the experiment is no sensitive. Figure~\ref{qtevDist} helps to illustrate this. One can notice that a small value of $p_{s+b}$ (green area) is balanced by large value of $p_{b}$ (yellow area). When the experimental sensitivity is higher, the two distributions tend to be far away from each other. Thus leading to a smaller compensation factor ($p_{b}$) and enhancing the chance of a exclusive $CL_s$ value.

\begin{figure}[htbp]
  \centering
  \includegraphics[width=0.5\textwidth,keepaspectratio]{figures_and_tables/cls/qtevDist.png}
  \caption{Example of $f(\tilde{q_{\mu}} \vert \mu, \hat{\theta}_{\mu}^{\text{obs}})$ $f(\tilde{q_{\mu}} \vert \mu=0, \hat{\theta}_{\mu = 0}^{\text{obs}})$ distributions generated with toy MC. In the figure, $q$ must be read as $\tilde{q}$. The green area shows the $p_{s+b}$ defined in~\ref{cls_p_def1}, while the yellow one shows $p_{b}$ defined in~\ref{cls_p_def2}. Source:~\cite{asymptotic_cls}.}
  \label{qtevDist}
\end{figure}

The expected upper limit and its $\pm 1 \sigma$ and $\pm 2 \sigma$ are determined by generating a large number of toy MC events, for the background-only model ($\mu = 0$), with nuisances free to float, and for each simulation finding $\mu_{95\%}$, which defines the confidence level. Once enough samples are generated, one should scan, from left to right, the cumulative distribution of $\mu_{95\%}$. The median defines the expected value and the quantiles for 16\%, 84\% and 2.5\%, 97.5\% defines the $\pm 1 \sigma$ and $\pm 2 \sigma$, respectively.

\subsection{Branching fraction upper limits}
\label{sec:results}
The results are summarized on table \ref{tab:UpperLimits_Cat123}.

\begin{table}[ht]
\begin{center}
\caption{Summary table for the limits on branching ratio of $\mathrm{Z}\to\Upsilon(1S,2S,3S)\gamma$ and $\mathrm{H}\to\Upsilon(1S,2S,3S)\gamma$ decays.}
%\resizebox{.5\width}{!}{\begin{tabular}{l|llll}
\multicolumn{4}{c}{95\% C.L. Upper Limit} \\
\hline
\hline
& \multicolumn{3}{c}{$\mathcal{B}(Z \rightarrow \Upsilon\gamma)$ $[\times10^{-6}]$}      \\
\cline{2-4}
&  $\Upsilon(1S)$ & $\Upsilon(2S)$ & $\Upsilon(3S)$  \\
\hline
Expected     & $6.4^{+3.1}_{-2.0}$ &  $8.3^{+4.0}_{-2.5}$  & $8.0^{+3.9}_{-2.4}$            \\
Observed     & 9.0 &  12.3  & 11.4      \\
\hline
SM Prediction $[\times10^{-8}]$ & 4.8  &  2.4  & 1.9      \\
\hline
\hline
& \multicolumn{3}{c}{$\mathcal{B}(H \rightarrow \Upsilon\gamma)$ $[\times10^{-4}]$}       \\
\cline{2-4}
&  $\Upsilon(1S)$ & $\Upsilon(2S)$ & $\Upsilon(3S)$ &   \\
\hline
Expected     & $12.5^{+6.1}_{-3.9}$ &  $14.6^{+7.1}_{-4.5}$  & $13.6^{+6.6}_{-4.2}$        \\
Observed     & 11.5 &  13.6  & 12.7     \\
\hline
SM Prediction $[\times10^{-9}]$ & 5.2  &  1.4  & 0.9      \\
\hline
\hline
\end{tabular}

}
% \begin{tabular}{l|llll}
\multicolumn{4}{c}{95\% C.L. Upper Limit} \\
\hline
\hline
& \multicolumn{3}{c}{$\mathcal{B}(Z \rightarrow \Upsilon\gamma)$ $[\times10^{-6}]$}      \\
\cline{2-4}
&  $\Upsilon(1S)$ & $\Upsilon(2S)$ & $\Upsilon(3S)$  \\
\hline
Expected     & $6.4^{+3.1}_{-2.0}$ &  $8.3^{+4.0}_{-2.5}$  & $8.0^{+3.9}_{-2.4}$            \\
Observed     & 9.0 &  12.3  & 11.4      \\
\hline
SM Prediction $[\times10^{-8}]$ & 4.8  &  2.4  & 1.9      \\
\hline
\hline
& \multicolumn{3}{c}{$\mathcal{B}(H \rightarrow \Upsilon\gamma)$ $[\times10^{-4}]$}       \\
\cline{2-4}
&  $\Upsilon(1S)$ & $\Upsilon(2S)$ & $\Upsilon(3S)$ &   \\
\hline
Expected     & $12.5^{+6.1}_{-3.9}$ &  $14.6^{+7.1}_{-4.5}$  & $13.6^{+6.6}_{-4.2}$        \\
Observed     & 11.5 &  13.6  & 12.7     \\
\hline
SM Prediction $[\times10^{-9}]$ & 5.2  &  1.4  & 0.9      \\
\hline
\hline
\end{tabular}



\begin{tabular}{l|llll}
\multicolumn{4}{c}{95\% C.L. Upper Limit} \\
\hline
\hline
& \multicolumn{3}{c}{$\mathcal{B}(Z \rightarrow \Upsilon\gamma)$ $[\times10^{-6}]$}      \\
\cline{2-4}
&  $\Upsilon(1S)$ & $\Upsilon(2S)$ & $\Upsilon(3S)$  \\
\hline
Expected     & $1.6^{+0.8}_{-0.5}$ &  $2.0^{+1.0}_{-0.6}$  & $1.8^{+1.0}_{-0.6}$            \\
Observed     & 2.9 &  2.7  & 1.4      \\
\hline
SM Prediction $[\times10^{-8}]$ & 4.8  &  2.4  & 1.9      \\
\hline
\hline
& \multicolumn{3}{c}{$\mathcal{B}(H \rightarrow \Upsilon\gamma)$ $[\times10^{-4}]$}       \\
\cline{2-4}
&  $\Upsilon(1S)$ & $\Upsilon(2S)$ & $\Upsilon(3S)$ &   \\
\hline
Expected     & $7.3^{+4.0}_{-2.4}$ &  $8.1^{+4.6}_{-2.8}$  & $6.8^{+3.9}_{-2.3}$        \\
Observed     & 6.9 &  7.4  & 5.8     \\
\hline
SM Prediction $[\times10^{-9}]$ & 5.2  &  1.4  & 0.9      \\
\hline
\hline
\end{tabular}
	
\label{tab:UpperLimits_Cat123}
\end{center}
\end{table}

The observed(expected) exclusion limit at $95\%$ confidence level on the $\mathcal{B}(\mathrm{Z}\to\Upsilon(1S,2S,3S)\gamma)=$ 2.9, 2.7, 1.4 ($1.6^{+0.8}_{-0.5}$,  $2.0^{+1.0}_{-0.6}$, $1.8^{+1.0}_{-0.6}$)$\times 10^{-6}$, and on the $\mathcal{B}(\mathrm{H}\to\Upsilon(1S,2S,3S)\gamma)=$ 6.9, 7.4, 5.8 ($7.3^{+4.0}_{-2.4}$,  $8.1^{+4.6}_{-2.8}$, $6.8^{+3.9}_{-2.3}$)$\times 10^{-4}$.

As stated before, this analysis was done, for the Z decay, taking into account a mutually excludent categorization of events, based on the reconstructed photon properties ($\eta_{SC}$ and R9 value), as described in Section~\ref{sec:categorization}. 

At Table~\ref{tab:UpperLimits_Cat0} we present the results obtained when there is no categorization of events (Inclusive category).

\begin{table}[ht]
\begin{center}
\caption{Summary table for the limits on branching ratio of $\mathrm{Z}\to\Upsilon(1S,2S,3S)\gamma$, for the two possible categorization scenarios.}
%\resizebox{.5\width}{!}{\begin{tabular}{l|llll}
\multicolumn{4}{c}{95\% C.L. Upper Limit} \\
\hline
\hline
& \multicolumn{3}{c}{$\mathcal{B}(Z \rightarrow \Upsilon\gamma)$ $[\times10^{-6}]$}      \\
\cline{2-4}
&  $\Upsilon(1S)$ & $\Upsilon(2S)$ & $\Upsilon(3S)$  \\
\hline
Expected     & $6.4^{+3.1}_{-2.0}$ &  $8.3^{+4.0}_{-2.5}$  & $8.0^{+3.9}_{-2.4}$            \\
Observed     & 9.0 &  12.3  & 11.4      \\
\hline
SM Prediction $[\times10^{-8}]$ & 4.8  &  2.4  & 1.9      \\
\hline
\hline
& \multicolumn{3}{c}{$\mathcal{B}(H \rightarrow \Upsilon\gamma)$ $[\times10^{-4}]$}       \\
\cline{2-4}
&  $\Upsilon(1S)$ & $\Upsilon(2S)$ & $\Upsilon(3S)$ &   \\
\hline
Expected     & $12.5^{+6.1}_{-3.9}$ &  $14.6^{+7.1}_{-4.5}$  & $13.6^{+6.6}_{-4.2}$        \\
Observed     & 11.5 &  13.6  & 12.7     \\
\hline
SM Prediction $[\times10^{-9}]$ & 5.2  &  1.4  & 0.9      \\
\hline
\hline
\end{tabular}

}
% \begin{tabular}{l|llll}
\multicolumn{4}{c}{95\% C.L. Upper Limit} \\
\hline
\hline
& \multicolumn{3}{c}{$\mathcal{B}(Z \rightarrow \Upsilon\gamma)$ $[\times10^{-6}]$}      \\
\cline{2-4}
&  $\Upsilon(1S)$ & $\Upsilon(2S)$ & $\Upsilon(3S)$  \\
\hline
Expected     & $6.4^{+3.1}_{-2.0}$ &  $8.3^{+4.0}_{-2.5}$  & $8.0^{+3.9}_{-2.4}$            \\
Observed     & 9.0 &  12.3  & 11.4      \\
\hline
SM Prediction $[\times10^{-8}]$ & 4.8  &  2.4  & 1.9      \\
\hline
\hline
& \multicolumn{3}{c}{$\mathcal{B}(H \rightarrow \Upsilon\gamma)$ $[\times10^{-4}]$}       \\
\cline{2-4}
&  $\Upsilon(1S)$ & $\Upsilon(2S)$ & $\Upsilon(3S)$ &   \\
\hline
Expected     & $12.5^{+6.1}_{-3.9}$ &  $14.6^{+7.1}_{-4.5}$  & $13.6^{+6.6}_{-4.2}$        \\
Observed     & 11.5 &  13.6  & 12.7     \\
\hline
SM Prediction $[\times10^{-9}]$ & 5.2  &  1.4  & 0.9      \\
\hline
\hline
\end{tabular}



\begin{tabular}{l|llll}
\multicolumn{4}{c}{95\% C.L. Upper Limit - $\mathcal{B}(Z \rightarrow \Upsilon\gamma)$ $[\times10^{-6}]$} \\
\hline
\hline
& \multicolumn{3}{c}{without categorization}      \\
\cline{2-4}
&  $\Upsilon(1S)$ & $\Upsilon(2S)$ & $\Upsilon(3S)$  \\
\hline
Expected     & $1.7^{+0.9}_{-0.5}$ &  $2.1^{+1.1}_{-0.7}$  & $1.9^{+1.0}_{-0.6}$            \\
Observed     & 2.6 &  2.3  & 1.2      \\
\hline
\hline
& \multicolumn{3}{c}{with categorization}      \\
\cline{2-4}
&  $\Upsilon(1S)$ & $\Upsilon(2S)$ & $\Upsilon(3S)$  \\
\hline
Expected     & $1.6^{+0.8}_{-0.5}$ &  $2.0^{+1.0}_{-0.6}$  & $1.8^{+1.0}_{-0.6}$            \\
Observed     & 2.9 &  2.7  & 1.4      \\
\hline
\hline
\end{tabular}
	
\label{tab:UpperLimits_Cat0}
\end{center}
\end{table}

It is worth to remember that the categorization takes places only for the Z decay. For the Higgs decay, no categorization is imposed.

By taking, or not, into account any categorization, the numbers presented in both tables (\ref{tab:UpperLimits_Cat123} and \ref{tab:UpperLimits_Cat0}), are compatible within themselves and with the results published by the ATLAS collaboration~\cite{atlas_paper_2018:2018txb}. Our interpretation to the lack of improvement of the no categorization scenario with respect to the categorized one, is that, the collected statistics, after full selection, is so small, that the categorization just jeopardize the amount of events available. 



% conclusion
\chapter{Conclusion}
\label{chapter_conclusion}

\todo{FAZER!}

\printbibliography[heading=bibintoc]


% Os anexos, se houver, vêm depois das referências:
\appendix

% O comando a seguir inclui o arquivo apendices.tex
% que contém os apêndices. Observe o comando \appendix
% na linha anterior
% Detalhe: não precisa incluir a extensão .tex
\include{apendices}

\end{document}