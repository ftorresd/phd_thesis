\begin{resumo}
    Esse trabalho apresenta o estudo sobre decaimentos raros de bósons do Modelo Padrão em quarkonia. O mesmo é feito em dados coletados em 2016 pelo detector CMS, com energia de centro de massa de $\sqrt{s}=13$ TeV. Decaimentos de bósons Z e Higgs em $\Upsilon(1S,2S,3S)$ e um fóton, com o subsequente decaimento do $\Upsilon(1S,2S,3S)$ em $\mu^{+} \mu^{-}$ são estudados utilizando uma luminosidade integrada de 35.86 $fb^{-1}$ em colisões próton-próton. Nenhum excesso significativo foi observado além da suposição do modelo de somente-fundo. Um limite em $95\%$ de nível de confiança é colocado na razão de ramificação dos decaimentos $Z \rightarrow  \Upsilon(1S,2S,3S) + \gamma$ de 2.9, 2.7, 1.4 $\times 10^{-6}$ e em $H\rightarrow  \Upsilon(1S,2S,3S) + \gamma$ de 6.9, 7.4, 5.8 $\times 10^{-4}$, usando o método de $CL_s$. Contribuições dadas de 2016 até 2018 para a operação, manutenção e P\&D da Fase-2 de melhorias do sistema de Câmaras de Placas Resistivas do CMS também são apresentadas. Incluindo plantões para operação do sistema, certificação de dados para controle de qualidade, melhorias e manutenção do sistema \textit{online} e manutenção do detector durante o período de manutenção conhecido como LS2.
\end{resumo}
    